\documentclass[output=paper,colorlinks,citecolor=brown]{langscibook}
\ChapterDOI{10.5281/zenodo.15148194}
\author{Radu Craioveanu\orcid{0000-0002-2181-4949}\affiliation{University of Toronto}}
\title{Weighing preaspiration}
\abstract{This chapter addresses some long-standing contradictions in the phonology and typology of preaspiration, reassessing the prosodic status of this phenomenon and arguing that its traditional representation in the literature does not align with cross-linguistic phonetic and phonological patterns. Contra the commonly held assumption that preaspiration is a laryngeal property of the following segment (and thus a kind of mirrored postaspiration), preaspiration is shown to be prosodically associated to a preceding syllable in all attested cases, separately from the following segment. I argue that preaspiration consistently bears moraic weight, and that this property renders it indistinguishable in most cases from a discrete coda [h] segment. This is supported by a subset of a larger typological survey; 15 languages across four families are described, where preaspiration is shown to be consistently weight-bearing or otherwise associated with the preceding syllable. Given this prosodic patterning, I propose that preaspiration is a phonological length-preserving strategy akin to compensatory vowel lengthening, and that it is not necessarily dependent on the [spread glottis] status of the following consonant. The typology supports this, showing that preaspiration may sometimes appear before both aspirated and unaspirated consonants, or before a consonant with no contrastive laryngeal specification. This typological overview and reassessment of preaspiration requires us to revisit what is ``rare'' about preaspiration, if anything.

\keywords{preaspiration, prosody, moraic structure, phonological typology,}
}

\IfFileExists{../localcommands.tex}{
   \addbibresource{../localbibliography.bib}
   \usepackage{tabularx, multicol, multirow, longtable}
\usepackage{url}
\urlstyle{same}

\usepackage{orcidlink}
\definecolor{orcidlogocol}{cmyk}{0,0,0,1}
\RenewDocumentCommand{\LinkToORCIDinAffiliations}{ +m }
  {%
    \orcidlink{#1}\,%
  }
\SetupAffiliations{orcid placement=before}

\usepackage{siunitx}
\sisetup{detect-weight=true, detect-family=true, group-digits=none}

\usepackage{mathtools}
\usepackage{langsci-optional}
\usepackage{langsci-lgr}
\usepackage{langsci-gb4e}

\usepackage{stmaryrd}
\usepackage[capitalize]{cleveref}
\babelfont[macedonian]{rm}[Language=Macedonian,ItalicFont=LibertinusSerif-Italic.otf]{LibertinusSerif-Regular.otf}
\usepackage{eqparbox}
\usepackage[autostyle]{csquotes}
\usepackage[linguistics]{forest}

\usetikzlibrary{positioning, matrix}
\usepackage[glosses,inline]{leipzig}
\PassOptionsToPackage{xindy,toc,nopostdot}{glossaries}
\usepackage{glossary-inline}
\setglossarystyle{inline}
\makeglossaries

\usepackage{phonrule}
\usepackage{threeparttable}


\usepackage{textcomp,gensymb}


\usepackage[preservefont]{tipauni}

\usepackage[normalem]{ulem}

\usepackage{enumitem} %so lists aren't ugly
	
\usepackage{threeparttable}	%allows tables with tablenotes. note marks: †‡
	\makeatletter 
	\g@addto@macro\TPT@defaults{\footnotesize} 
	\makeatother

\usepackage{colortbl}
	\definecolor{Pink}{rgb}{0.96, 0.76, 0.76} 
	\definecolor{PaleBlue}{rgb}{0.67, 0.9, 0.93}
	\definecolor{carolinablue}{rgb}{0.6, 0.73, 0.89}
	\definecolor{goldenyellow}{rgb}{1.0, 0.87, 0.0}
	\definecolor{Orange}{rgb}{1.0, 0.66, 0.07}
	\definecolor{puce}{rgb}{0.8, 0.53, 0.6}
	\definecolor{turquoisegreen}{rgb}{0.63, 0.84, 0.71}


% add all extra packages you need to load to this file
\usepackage{langsci-textipa}
\usepackage{vowel}
\usepackage{textgreek}

% \usepackage{langsci-branding}
% \usepackage{subcaption}
\usepackage{subfigure}

\usepackage{tabto}


\usetikzlibrary{tikzmark}
\usepackage{pgfplots}


\newfontfamily\tibetan{NotoSerifTibetan-Regular.ttf}
\usepackage{langsci-branding}
\usepackage{hyphenat}

\usepackage{accents}

   \renewcommand{\lsChapterFooterSize}{\footnotesize}

\makeatletter
\let\thetitle\@title
\let\theauthor\@author
\makeatother

\newcommand{\togglepaper}[1][0]{
   \bibliography{../localbibliography}
   \papernote{\scriptsize\normalfont
     \theauthor.
     \titleTemp.
     To appear in:
     Natalia Kuznetsova, Cormac Anderson \& Shelece Easterday (ed.).
     Rarities in phonetics and phonology.tex.
     Berlin: Language Science Press. [preliminary page numbering]
   }
   \pagenumbering{roman}
   \setcounter{chapter}{#1}
   \addtocounter{chapter}{-1}
}

\newbool{bookcompile}
\booltrue{bookcompile}
\newcommand{\bookorchapter}[2]{\ifbool{bookcompile}{#1}{#2}}

\newcommand{\textarab}[1]{\RL{\arabicfont #1}}

\newcommand\mb[1]{\eqparbox[t]{examples}{#1}\hspace{1em}}
\newcommand\mbi[1]{\mb{#1}}
\newcommand{\twe}[3]{\mbi{#1}\eqparbox[t]{orths}{\emph{#2}}\hspace{1em}`#3'\hspace{1em}} % three-way example
\providecommand\glottocode[1]{[\href{https://glottolog.org/resource/languoid/id/#1}{#1}]}
\newcommand{\phonreal}[1]{\ensuremath{\llbracket}#1\ensuremath{\rrbracket}}

\DeclareRobustCommand\dash{\unskip\nobreak\thinspace\textendash\allowbreak\thinspace\ignorespaces}

\forestset{minus/.style={edge label={node[midway, left] {\ensuremath{-}\hspace*{2mm}}}},
plus/.style={edge label={node[midway, right] {\hspace*{2mm}\ensuremath{+}}}}}
\providecommand\ipa[1]{#1}


\newcommand{\tone}[1]{\textsuperscript{#1}}

\newcommand{\orthog}[1]{\textit{#1}}
\newcommand{\gloss}[1]{`#1'}

\newcommand{\glottolog}[1]{\texttt{\href{https://glottolog.org/resource/languoid/id/#1}{#1}}}

\newcolumntype{O}{>{\itshape }l<{}}
\newcolumntype{G}{>{`}l<{'}}

\newcounter{tabsubcounter}
\newcommand{\tablecounter}{\setcounter{tabsubcounter}{0}}
\newcommand{\TC}{\stepcounter{tabsubcounter}\alph{tabsubcounter}.}

\usetikzlibrary{chains,positioning,calc,decorations.markings}
\tikzset{
	seg/.style={text height=0.6em, text depth=0.3em},
	moraic-structure/.style={xscale=0.6,yscale=1.1, text height=0.65em,text depth=0.25em},
 }

%05_Culhane_Edwards
%%%%%%%%%%%%%%%%%%%%%%%%%%%%%%%%
%%	Symbols and Characters  	%%
%%%%%%%%%%%%%%%%%%%%%%%%%%%%%%%% αβσµ

\newcommand{\tl}{\char`~}						%middle tilde ~
\renewcommand{\Q}{\textquotesingle}		%straight apostrophe444
\newcommand{\ra}{→} 								%right arrow ->
\newcommand{\0}{∅} 									%zero symbol
\newcommand{\gap}{\textunderscore} 	%underscore
%\renewcommand{\j}{ʤ}								%dezh digraph
\newcommand{\syll}{σ}								%lowercase sigma medial form
\newcommand{\wrd}{ω}								%lowercase omega
\newcommand{\ft}{φ}									%lowercase phi
\newcommand{\gw}{gʷ}								%g with superscript w
\newcommand{\B}{β}									%voiced bilabial fricative
\newcommand{\hp}{\hphantom}					%space equal to width of argument
\newcommand{\it}{\textit}	%italics

%%%%%%%%%%%%%%%%%%%%%%%%%%%%%%%%
%%	Font Styles & Formatting	%%
%%%%%%%%%%%%%%%%%%%%%%%%%%%%%%%%

\definecolor{DarkBlue}{RGB}{0,0,130}										%dark blue colour
% \newcommand{\ve}[1]{\textcolor{DarkBlue}{\textit{#1}}}	%vernacular text
\newcommand{\ve}[1]{{\textit{#1}}}	%vernacular text
\definecolor{DarkRed}{RGB}{150,0,0}											%dark red colour
% \newcommand{\tbr}[1]{\textcolor{DarkRed}{\textbf{#1}}}	%Bold red text
\newcommand{\tbr}[1]{{\textbf{#1}}}	%Bold red text
%\renewcommand{\it}{\textit}																%italics
\newcommand{\tsc}{\textsc}															%small caps
\newcommand{\sub}{\textsubscript}												%subscript
\newcommand{\su}{\textsuperscript}											%superscript

%%%%%%%%%%%%%%%%%%%%%%%%%%%%%%%%%%%%%%%%%%%%%%%%%%%%
%% Tables %% Tables %% Tables %% Tables %% Tables %%
%%%%%%%%%%%%%%%%%%%%%%%%%%%%%%%%%%%%%%%%%%%%%%%%%%%%

% \newcommand{\mc}{\multicolumn}									%multicolumn
% \newcommand{\st}[1]{\setlength{\tabcolsep}{#1}}	%reduce column width in tables
%
%%%%%%%%%%%%%%%%%%%%%%%%%%%%%%%%
%%    Cross   References      %%
%%%%%%%%%%%%%%%%%%%%%%%%%%%%%%%%

% \def\Plus{\texttt{+}}
% \def\Minus{\texttt{-}}
% \newcommand{\GS}{ʔ}
% \def\SH{ʃ}
% \newcommand{\TSH}{ʧ}
% \def\ZH{ʒ}
% \def\DZH{ʤ}
% \def\:{ː}
% \def\UP{\textsuperscript}
% \def\rs{ʂ}
% \newcommand{\rn}{ɳ}
% \def\rt{ʈ}
% \def\tllr{ɺ}
% \newcommand{\Bb}{β}
% \def\Eps{ɛ}
% \def\Oo{ɔ}
% \def\Gm{ɣ}
% \def\NG{ŋ}
% \def\barU{ʉ}
\newcommand{\CM}{\ding{51}}
\newcommand{\XM}{\ding{53}}
% \newcommand{\tap}{ɾ}
% \def\darkL{ɫ}
% \def\schwa{ə}
%
% \def\BUL{\textbullet}


%%%%%%%%%%%%%%
%					%
%	Secondaries		%
%					%
%%%%%%%%%%%%%%
%	Post
\newcommand{\Post}[2]{#1\textsuperscript{#2}}
%	Pre
\newcommand{\Pre} [2] {\textsuperscript{#1}#2}
%	Undertilde
\newcommand{\utilde}[1]{\ensuremath{\smash{\underset{\mathclap{\sim}}{\text{#1}}}}}
%	Devoiced
% \newcommand{\VCLS}[1]{\textsubring{#1}}
%%%%%%%%%%%
%				%
%	Definitions		%
%	Markup		%
%				%
%%%%%%%%%%%
% \def\->{$\rightarrow$}
% \def\__{\underline{\hspace{1em}}}
\def\NoPoss{\cellcolor{gray!30}}

\newcommand{\VOICELESS}{\textsc{voiceless}}
\newcommand{\VOICED}{\textsc{voiced}}
\newcommand{\tablenote}[2][1]{\parbox{#1\textwidth}{\footnotesize\raggedright #2}}

\newcommand{\appref}[1]{Appendix~\ref{#1}}
\renewcommand{\sectref}[1]{Section~\ref{#1}}


\newcommand{\dobuibox}[5]{#1\\[-1.1em]
\hspace*{-.8cm}
 \begin{tabularx}{.9\textwidth}{@{}lQQ@{}}
       &  {oral} &  {nasal} \\
       \midrule
     {controlled} &\parbox[t]{4cm}{\raggedright  #2} & \parbox[t]{4cm}{\raggedright #3} \\
     \tablevspace
     {ballistic} &\parbox[t]{4cm}{\raggedright  #4} & \parbox[t]{4cm}{\raggedright  #5} \\
 \end{tabularx}
}

\newfontfamily\VdottildeFont{LibertinusVdottilde.otf}

\newcommand{\Vdottilde}{{\VdottildeFont V̰̣}}

% \renewcommand{\keywords}[1]{\textbf{#1}}

   %% hyphenation points for line breaks
%% Normally, automatic hyphenation in LaTeX is very good
%% If a word is mis-hyphenated, add it to this file
%%
%% add information to TeX file before \begin{document} with:
%% %% hyphenation points for line breaks
%% Normally, automatic hyphenation in LaTeX is very good
%% If a word is mis-hyphenated, add it to this file
%%
%% add information to TeX file before \begin{document} with:
%% %% hyphenation points for line breaks
%% Normally, automatic hyphenation in LaTeX is very good
%% If a word is mis-hyphenated, add it to this file
%%
%% add information to TeX file before \begin{document} with:
%% \include{localhyphenation}
\hyphenation{
    af-fri-cates
    al-ve-o-pal-a-tal
    Ama-nu-ban
    Ara-wak-an
    Árna-son
    Ber-ber
    can-di-dates
    Cam-er-oon
    Chi-nan-tec
    Chir-ko-va
    Crai-o-ve-a-nu
    di-chot-o-my
    Ec-ua-do-rian
    Ec-ua-dor
    elec-tro-glot-to-gra-phy
    Faro-ese
    Ike-ma
    Kuznet-sova
    Mes-kwa-ki
    Mio-ma-fo
    mono-mor-aic
    Ne-ca-xa
    Oto-man-gue-an
    par-a-digm
    post-as-pi-rat-ed
    post-as-pi-ra-tion
    pre-as-pi-rat-ed
    pre-as-pi-ra-tion
    pros-o-dic
    pros-o-dies
    re-con-struc-table
    Sheh-ret
    Svan-tes-son
    Ta-ras-can
    Tórs-havn
    Ural-ic
    epen-the-sis
    Anin-dil-yak-wa
    Mi-nyag
    Na-ka-ma
}

\hyphenation{
    af-fri-cates
    al-ve-o-pal-a-tal
    Ama-nu-ban
    Ara-wak-an
    Árna-son
    Ber-ber
    can-di-dates
    Cam-er-oon
    Chi-nan-tec
    Chir-ko-va
    Crai-o-ve-a-nu
    di-chot-o-my
    Ec-ua-do-rian
    Ec-ua-dor
    elec-tro-glot-to-gra-phy
    Faro-ese
    Ike-ma
    Kuznet-sova
    Mes-kwa-ki
    Mio-ma-fo
    mono-mor-aic
    Ne-ca-xa
    Oto-man-gue-an
    par-a-digm
    post-as-pi-rat-ed
    post-as-pi-ra-tion
    pre-as-pi-rat-ed
    pre-as-pi-ra-tion
    pros-o-dic
    pros-o-dies
    re-con-struc-table
    Sheh-ret
    Svan-tes-son
    Ta-ras-can
    Tórs-havn
    Ural-ic
    epen-the-sis
    Anin-dil-yak-wa
    Mi-nyag
    Na-ka-ma
}

\hyphenation{
    af-fri-cates
    al-ve-o-pal-a-tal
    Ama-nu-ban
    Ara-wak-an
    Árna-son
    Ber-ber
    can-di-dates
    Cam-er-oon
    Chi-nan-tec
    Chir-ko-va
    Crai-o-ve-a-nu
    di-chot-o-my
    Ec-ua-do-rian
    Ec-ua-dor
    elec-tro-glot-to-gra-phy
    Faro-ese
    Ike-ma
    Kuznet-sova
    Mes-kwa-ki
    Mio-ma-fo
    mono-mor-aic
    Ne-ca-xa
    Oto-man-gue-an
    par-a-digm
    post-as-pi-rat-ed
    post-as-pi-ra-tion
    pre-as-pi-rat-ed
    pre-as-pi-ra-tion
    pros-o-dic
    pros-o-dies
    re-con-struc-table
    Sheh-ret
    Svan-tes-son
    Ta-ras-can
    Tórs-havn
    Ural-ic
    epen-the-sis
    Anin-dil-yak-wa
    Mi-nyag
    Na-ka-ma
}

   \boolfalse{bookcompile}
   \togglepaper[17]%%chapternumber
}{}

\begin{document}
\maketitle

\section{Introduction}
\label{sec-introduction}

Preaspiration, generally described as glottal frication preceding the onset of a particular set of segments, is considered much more unusual than postaspiration, and some work has been dedicated to describing the typology of the languages displaying this more marked property (e.g., \citealt{nichasaide1985,Helgason2002,Silverman2003,Clayton:2010,chapters/16_Hejná}). Based on nomenclature and brief phonetic description, it is easy to assume a symmetry between pre- and postaspiration: due to the ubiquity of [spread glottis] as a featural component of oral segments, it seems intuitive to consider preaspiration and postaspiration as different phonetic implementations of aspiration, in which the relative timing of the oral and glottal articulations is reversed, but the underlying phonological status is the same.
This is reflected in definitions of aspiration like that of \citeauthor{LadefogedMaddieson1996}, who say it can occur ``before or after a stricture'', and explicitly follow the gestural alignment approach: ``it is obvious that glottal gestures in these two categories differ in their timing relationships with the associated oral gestures'' ({\citeyear{LadefogedMaddieson1996}: 48, 72}). Similar phonetic definitions and analyses have been provided by {\citeauthor{catford1968}} (\citeyear{catford1968}: 332) and {\citeauthor{stevens1975}} (\citeyear{stevens1975}: 19), among others. In line with this perspective, virtually all previous work on preaspiration implies, assumes, or concludes that preaspiration must be a phonological property of the following stop, and thus be distinct from a heterosegmental sequence of \textit{h}+C \citep[e.g.,][]{nichasaide1985,Helgason2002,Clayton:2010}. This means that the common received view holds pre- and postaspiration to be either phonologically equivalent or phonologically symmetrical, depending on the level of subsegmental detail of the model. For instance, {\citeauthor{kehreingolston2004}'s (\citeyear{kehreingolston2004})} proposal that laryngeal features are properties of prosodic constituents rather than individual segments is intended to permit a flexible ordering of oral and laryngeal articulations under the same phonological representation.

This (implicit or explicit) assumption of equivalence or symmetry in aspiration is supported by an apparent lack of phonological contrast between pre- and postaspiration: ``preaspirating'' languages are generally reported to have postaspirated obstruents as well, and the alignment of the aspiration noise is claimed to be phonologically predictable \citep[e.g.,][]{LadefogedMaddieson1996,Silverman2003, kehreingolston2004}. Among the most frequently reported conditioning factors is position within the word or utterance: only medial and final consonants are preaspirated, while initial ones are postaspirated. 
This distribution is argued by \citet{golstonkehrein2013} to be determined largely by sonority sequencing.  
Some typical examples of data used as support for this view are given in~(\ref{ex-alleged-symmetry-data}).%
\footnote{Linguistic data in this chapter is provided in the IPA as much as possible. Transcriptions and glosses are largely those given by the authors cited, and consequently there may be minor variations in transcription style or amount of morphological detail glossed from language to language. In cases where the cited source does not use IPA transcription, I have represented the data in IPA as faithfully as possible; this adaptation of the transcription is noted where applicable. Apart from the example of alleged laryngeal symmetry in (\ref{ex-alleged-symmetry-data}), I render preaspiration consistently as [h] rather than [ʰ] throughout this chapter, regardless of the source's transcription.}

\ea
\label{ex-alleged-symmetry-data}
\tablecounter
\begin{tabular}[t]{l lG @{\hspace{2em}} l lG}
\TC & \multicolumn{2}{l}{Scottish Gaelic \citep{nichasaide1985}} & \TC & \multicolumn{2}{l}{Faroese \citep{arnason2011}} \\
& [pʰaːʰpə] & Pope & & [pʰɔɑʰpɪ] & daddy \\
& [kʰaʰpəɫ] & mare & & [tʰaʰkːa] & to thank \\
\TC & \multicolumn{2}{l}{Mongolian \citep{svantesson2005}} & \TC & \multicolumn{2}{l}{Pur\'epecha \citep{foster1965}} \\
& [tʰaɮ] & steppe & & [kʰaʰtsɨkwa] & hat \\
& [aʰta] & camel gelding.\textsc{refl} & & [teʰpar] & canoe \\
& [aʰt] & camel gelding & & [iʰkʷan] & to bathe \\
\end{tabular}
\z

This positional allophony account goes hand in hand with the gestural alignment theory mentioned above: aspiration is phonetically realized before a constriction or after a constriction based on when the vocal folds open, depending on the surrounding phonological environment. Although this description is conceptually and representationally convenient, examination of the phonetic and phonological patterns of preaspiration across languages immediately shows some profound asymmetries between pre- and postaspiration: most notably, preaspiration has a consistent prosodic link to a preceding syllable that is not seen for postaspiration. Asymmetries have been reported previously for pre- and postaspiration (e.g., {\citealp[13--15]{Clayton:2010}}), and are visible in all relevant language descriptions, but nevertheless there has been no broad-scale work on the prosodic or segmental status of preaspiration to date. In many languages, the inherently moraic nature of preaspiration results in it being fundamentally indistinguishable from a discrete coda [h] segment, and a unified conception of pre- and postaspiration is thus a misrepresentation of the phonological facts.

This chapter therefore revisits the commonly cited examples of preaspiration and surveys others from the literature in order to illustrate a consistent association of preaspiration to phonological weight, and thus a prosodic affiliation with the preceding syllable. This moraic status is a defining property of preaspiration cross-linguistically, and means that preaspiration should not be viewed as a purely featural pre-consonantal phenomenon,%
\footnote{Of course, a subsegmental analysis of preaspiration as a consonant feature is perfectly functional in some languages, and may be phonologically appealing for phonotactic or other reasons. However, in all of the cases seen in a wider typological survey \citep{craioveanu-thesis}, there is also a plausible analysis available in which preaspiration is synchronically associated to a mora of the preceding syllable, independently of the consonant that it supposedly ``belongs'' to. It is not always possible to prove that preaspiration in a given language \textit{cannot} be underlyingly a feature of a consonant: this becomes a diachronic question of whether the moraic association is formed through an active phonological process or is the result of historical change (cf.\ \citealp{balsbaal2012} for an argument that preaspiration is underlyingly featural but undergoes fission from the consonant during the phonological derivation). The crucial observation presented here is that we never see preaspiration that is not affiliated with a mora or that is not syllabified separately from the following consonant.}
but rather as part of the prosodic system. In \S\ref{sec-assumptions}, I outline some foundational assumptions about phonology generally and preaspiration specifically that will be important to this discussion, followed by a typological survey of preaspiration in \S\ref{sec-typology}. In \S\ref{sec-discussion}, I discuss patterns seen across the languages, proposing that preaspiration can serve as a phonological length-preserving strategy, along the lines of compensatory vowel lengthening but in a different domain. 


\section{Assumptions \& definitions}
\label{sec-assumptions}

Past research approaches preaspiration as a fundamentally rare phenomenon, with the reasons for its rarity being important to investigate \citep[e.g.,][]{Silverman2003}. However, suggestions that preaspiration might be less rare than generally reported date back to at least \citet{nichasaide1985}, who speculates that some languages that are broadly described as having a plain vs.~aspirated laryngeal contrast might actually feature preaspiration. \citeauthor{hejna2015} (\citeyear{hejna2015, hejna2019, chapters/16_Hejná}) and \citeauthor{iosad2017-mfm} (\citeyear{iosad2017-mfm,iosad2018,chapters/02_Iosad}) also express the opinion that the rarity of preaspiration may be overstated, possibly due to fieldworkers' expectations or phonological awareness. Either preaspiration went unnoticed because the researcher was not expecting to find it and did not hear it, or the researcher heard it but judged it (consciously or subconsciously) to not be phonologically relevant.

\largerpage[-1]
In any discussion of typology or rarity, it must be clear exactly what is being described as rare, be it a combination of articulatory gestures or a more abstract phonological representation like a segment, and the theoretical approach should be explicit and consistent. Thus, I argue that in research on preaspiration, its phonological and prosodic status should be either an explicit area of investigation or a core assumption underlying the theoretical analysis. In past work on preaspiration, this has not always been the case: some authors have vague or unclear assumptions, make conflicting statements, or remain deliberately agnostic to the phonological status of preaspiration. \citet{Helgason2002} and \citet{Silverman2003} both state that they are not following a segmental theory of phonology, and claim that therefore the distinction between preaspiration and an \textit{hC} cluster is not relevant. Despite this declared position, their arguments and discussion of preaspiration align with a segmentally unary conception of preaspiration. They discuss alternations of preaspirated and postaspirated stops, as if these had an allophonic relationship, and describe preaspiration as a component of the following stop. \citet{Clayton:2010} presents arguments in favour of viewing preaspiration as a distinct segment but also in favour of it as a feature of the following stop, finding merits for both positions before settling implicitly on a segmentally unary conception of preaspirated stops.

Part of this equivocation is a result of the origins of preaspiration in the languages that have been most carefully studied -- where it arises from loss of historical geminates and thus is still phonotactically and orthographically associated with them --  in combination with a persistent association with prosodic weight that leaves it patterning like a coda segment in most cases. Thus, it is important to be clear on what we consider preaspiration to be, what its role is in the phonology, and what kind of phonetic and phonological evidence is important when discussing it.

In this chapter, I follow a generative approach to phonology that is strictly segmental. Segments can be expressed as bundles of features, which are generally based in articulation, although a phonetic basis is not strictly necessary. Phonemes are specified only for the features necessary to contrast them from other phonemes in the language's inventory. I specifically adopt the approach of Modified Contrastive Specification \citep{dresher2009,hall2011}, in which contrasts within an inventory are established through repeated featural division of sets of phonemes into smaller natural classes, until all phonemes are featurally distinct (the Successive Division Algorithm). Although a detailed description of this algorithm is not relevant here, the crucial proposal is that the phonemic contrasts within an inventory determine what properties of the segments are phonologically relevant. In other words, features that are not contrastive are not accessible to the phonological computation. To give a specific example, a language with the stop inventory /p,~pʰ,~tʰ,~k,~kʰ/ requires a single feature like [spread glottis] to contrast its two laryngeal series of stops, but a language with the more limited inventory /p,~t,~k/ would not make use of any laryngeal features in its phonological system.

\largerpage[-1]
In the suprasegmental domain, I follow a version of Moraic Theory \citep{hyman1985,mccarthyprince1986,hayes1989}, assuming a moraic tier that segments associate to directly without an intervening skeletal tier, and higher organization into syllables. As I am arguing for a weight-based prosodic analysis of preaspiration, I treat all of the languages discussed here as having moraic structure.
Syllable onsets are considered to be non-moraic and syllable nuclei and codas are considered to be moraic, unless evidence shows otherwise in a particular language (e.g., onset geminates, \citealp{topintzi2008}; extrametrical final consonants, \citealp{kristoffersen1999}). I assume a moraic theory of geminates, in which these are single segments associated both to a syllable onset and to a moraic preceding coda \citep{hayes1989,davis2011}.%
\footnote{Although it is possible to have long consonants that are not underlyingly a single segment (e.g., in multimorphemic words like English \textit{bookcase}), I consider these sequences of identical consonants rather than ``true'' geminates.}
Syllables may be maximally bimoraic; even in languages with ternary length distinctions, there is no clear evidence that ``superheavy'' trimoraic syllables are theoretically necessary \citep[e.g.,][]{bye2005,balsbaal2012,prehn2012}. Instead, I rely on a prosodic framework making use of mora sharing \citep{maddieson1993,hubbard1994,broselowetal1997}, and this structural assumption in combination with the proposal that preaspiration is always moraic predicts that we should be able to capture all preaspiration patterns with one of the three structures shown in \figref{fig:ex-preaspiration-moraic-structures}.

\begin{figure}
\caption{Typology of predicted moraic structures for preaspiration}
\label{fig:ex-preaspiration-moraic-structures}
\subfigure{
\begin{tikzpicture}[moraic-structure]
    \node (label) at (0.5,2) {\small (a)};	
	\node (c1) at (1,0) {C};	
	\node (v1) at (2,0) {V};	
	\node (c2) at (3,0) {h};	
	\node (c3) at (4,0) {C};	
	\node (v2) at (5,0) {V};	
	%
	\node (m1) at (2,1) {μ};
	\node (m2) at (3,1) {μ};
	\node (m3) at (5,1) {μ};
	%
	\node (s1) at (2,2) {σ};
	\node (s2) at (5,2) {σ};
	%
	\draw (c1.north) -- (s1.south);
	\draw (v1.north) -- (m1.south);
	\draw (c2.north) -- (m2.south);
	%
	\draw (c3.north) -- (s2.south);
	\draw (v2.north) -- (m3.south);
	%
	\draw (m1.north) -- (s1.south);
	\draw (m2.north) -- (s1.south);
	\draw (m3.north) -- (s2.south);
\end{tikzpicture}
}
\hfill
\subfigure{
\begin{tikzpicture}[moraic-structure]
    \node (label) at (0.5,2) {\small (b)};	
	\node (c1) at (1,0) {C};	
	\node (v1) at (2,0) {Vː};	
	\node (c2) at (3,0) {h};	
	\node (c3) at (4,0) {C};	
	\node (v2) at (5,0) {V};	
	%
	\node (m1) at (2,1) {μ};
	\node (m2) at (3,1) {μ};
	\node (m3) at (5,1) {μ};
	%
	\node (s1) at (2,2) {σ};
	\node (s2) at (5,2) {σ};
	%
	\draw (c1.north) -- (s1.south);
	\draw (v1.north) -- (m1.south);
	\draw (v1.north) -- (m2.south);
	\draw (c2.north) -- (m2.south);
	%
	\draw (c3.north) -- (s2.south);
	\draw (v2.north) -- (m3.south);
	%
	\draw (m1.north) -- (s1.south);
	\draw (m2.north) -- (s1.south);
	\draw (m3.north) -- (s2.south);
\end{tikzpicture}
}
\hfill
\subfigure{
\begin{tikzpicture}[moraic-structure]
    \node (label) at (0.5,2) {\small (c)};	
	\node (c1) at (1,0) {C};	
	\node (v1) at (2,0) {V};	
	\node (c2) at (3,0) {h};	
	\node (c3) at (4,0) {Cː};	
	\node (v2) at (5,0) {V};	
	%
	\node (m1) at (2,1) {μ};
	\node (m2) at (3,1) {μ};
	\node (m3) at (5,1) {μ};
	%
	\node (s1) at (2,2) {σ};
	\node (s2) at (5,2) {σ};
	%
	\draw (c1.north) -- (s1.south);
	\draw (v1.north) -- (m1.south);
	\draw (c2.north) -- (m2.south);
	%
	\draw (c3.north) -- (s2.south);
	\draw (c3.north) -- (m2.south);
	\draw (v2.north) -- (m3.south);
	%
	\draw (m1.north) -- (s1.south);
	\draw (m2.north) -- (s1.south);
	\draw (m3.north) -- (s2.south);
\end{tikzpicture}
}
\end{figure}

These combinations capture the range of environments in which preaspiration can appear cross-linguistically, which always involve quantity in some way. This can be seen either through categorical alternation of preaspiration with long vowels or consonants, or through phonetic length complementarity (e.g., shorter vowels before longer preaspiration). Notably, I use the term ``moraic'' with reference to preaspiration to indicate that the [h] has some association to a mora, even if it is sharing this mora with another segment; a non-moraic consonant would be associated directly to the syllable tier, as an onset or an extrametrical consonant. 

Following the assumption that moraic structure is reflected in the phonetic durations of segments \citep{broselowetal1997}, we expect that different moraic association patterns would result in different ratios of segmental duration. For instance, the [h] in a structure like \figref{fig:ex-preaspiration-moraic-structures}a is expected to be phonetically longer than in \figref{fig:ex-preaspiration-moraic-structures}b and \figref{fig:ex-preaspiration-moraic-structures}c, and the long vowel that shares its second mora in \figref{fig:ex-preaspiration-moraic-structures}b is expected to be longer than a short (monomoraic) vowel but shorter than a fully long (bimoraic) vowel. Conversely, this assumption also means that we may interpret phonetic differences in segmental duration ratios as evidence for different underlying moraic structures. The use of phonetic evidence to support phonological structure allows us to assess the nature of preaspiration in languages that do not have phonological processes that are informative about preaspiration or prosodic structure, or in varieties that have limited published work. Crucially, we predict that prosodic structures with non-moraic preaspiration as in \figref{fig:ex-unattested-moraic-structures} are unattested.

\begin{figure}
\caption{Possible but unattested prosodic structures in which preaspiration is non-moraic}
\label{fig:ex-unattested-moraic-structures}
\subfigure{
\begin{tikzpicture}[moraic-structure]
    \node (label) at (0.5,2) {\small (a)};	
	\node at (0.5,0) {*};	
	\node (c1) at (1,0) {C};	
	\node (v1) at (2,0) {V};	
	\node (c3) at (3.25,0) {hC};	
	\node (v2) at (4.25,0) {V};	
	%
	\node (m1) at (2,1) {μ};
	\node (m3) at (4.25,1) {μ};
	%
	\node (s1) at (2,2) {σ};
	\node (s2) at (4.25,2) {σ};
	%
	\draw (c1.north) -- (s1.south);
	\draw (v1.north) -- (m1.south);
	%
	\draw (c3.north) -- (s2.south);
	\draw (v2.north) -- (m3.south);
	%
	\draw (m1.north) -- (s1.south);
	\draw (m3.north) -- (s2.south);
\end{tikzpicture}
}
\hspace{3em}
\subfigure{
\begin{tikzpicture}[moraic-structure]
    \node (label) at (0.5,2) {\small (b)};	
	\node at (0.5,0) {*};	
	\node (c1) at (1,0) {C};	
	\node (v1) at (2,0) {Vː};	
	\node (c3) at (3.25,0) {hC};	
	\node (v2) at (4.25,0) {V};	
	%
	\node (m1) at (2,1) {μ};
	\node (m2) at (2.75,1) {μ};
	\node (m3) at (4.25,1) {μ};
	%
	\node (s1) at (2,2) {σ};
	\node (s2) at (4.25,2) {σ};
	%
	\draw (c1.north) -- (s1.south);
	\draw (v1.north) -- (m1.south);
	\draw (v1.north) -- (m2.south);
	%
	\draw (c3.north) -- (s2.south);
	\draw (v2.north) -- (m3.south);
	%
	\draw (m1.north) -- (s1.south);
	\draw (m2.north) -- (s1.south);
	\draw (m3.north) -- (s2.south);
\end{tikzpicture}
}


\subfigure{
\begin{tikzpicture}[moraic-structure]
    \node (label) at (0.5,2) {\small (c)};	
	\node at (0.5,0) {*};	
	\node (c1) at (1,0) {C};	
	\node (v1) at (2,0) {V};	
	\node (c3) at (3.25,0) {hC};	
	\node (v2) at (4.25,0) {V};	
	%
	\node (m1) at (2,1) {μ};
	\node (m2) at (2.75,1) {μ};
	\node (m3) at (4.25,1) {μ};
	%
	\node (s1) at (2,2) {σ};
	\node (s2) at (4.25,2) {σ};
	%
	\draw (c1.north) -- (s1.south);
	\draw (v1.north) -- (m1.south);
	\draw (c3.north) -- (m2.south);
	%
	\draw (c3.north) -- (s2.south);
	\draw (v2.north) -- (m3.south);
	%
	\draw (m1.north) -- (s1.south);
	\draw (m2.north) -- (s1.south);
	\draw (m3.north) -- (s2.south);
\end{tikzpicture}
}%
\hspace{3em}
\subfigure{
\begin{tikzpicture}[moraic-structure]
    \node (label) at (0.5,2) {\small (d)};	
	\node at (0.5,0) {*};	
	\node (c1) at (1,0) {C};	
	\node (v1) at (2,0) {V};	
	\node (h) at (3,0) {h};	
	\node (c3) at (3.75,0) {C};	
	\node (v2) at (4.75,0) {V};	
	%
	\node (m1) at (2,1) {μ};
	\node (m3) at (4.75,1) {μ};
	%
	\node (s1) at (2,2) {σ};
	\node (s2) at (4.75,2) {σ};
	%
	\draw (c1.north) -- (s1.south);
	\draw (v1.north) -- (m1.south);
	%
	\draw (c3.north) -- (s2.south);
	\draw (v2.north) -- (m3.south);
	%
	\draw (m1.north) -- (s1.south);
	\draw (m3.north) -- (s2.south);
	\draw (h.north) -- (s1.south);
\end{tikzpicture}
}
\end{figure}

It is worth clarifying here that the segmentally separate representation of preaspiration in the structures in \figref{fig:ex-preaspiration-moraic-structures} is not strictly dependent on its prosodic separation into a separate syllable from the following segment. Notably, the structure in \figref{fig:ex-preaspiration-moraic-structures}c features preaspiration before a geminate, and this preaspiration is not treated differently because it shares a mora with the following consonant. Likewise, I do not consider the geminate to be two segments despite its association to two syllables.%
\footnote{It is worth mentioning here that the stop and fricative components of geminate affricates like [tʃː] may also not lengthen proportionally. Although this is outside the scope of the current chapter, it would be worth comparing the phonological and phonetic patterns observed for geminate affricates as a potential area of contrast to preaspiration.}
In structures like \figref{fig:ex-preaspiration-moraic-structures}a--b, preaspiration bears moraic weight separately from the following consonant, and thus may be expected to lengthen independently of that consonant under stress, or to interact with preceding vowel length separately from the following consonant. In the structure in \figref{fig:ex-preaspiration-moraic-structures}c, the discrete nature of preaspiration would be harder to establish uncontroversially due to the shared mora. However, the languages discussed here that show evidence for this moraic structure (Faroese, \S\ref{sec-faroese}; Central Standard Swedish, \S\ref{sec-swedish-norwegian}) do not use it exclusively, and show evidence of structure \figref{fig:ex-preaspiration-moraic-structures}a or \figref{fig:ex-preaspiration-moraic-structures}b in other words. The primary stance I am taking on preaspiration is that it is always moraic: although this strongly suggests it is a discrete segment, and I will treat it as such in all cases for consistency, the issue of segmenthood is secondary in my mind.

Finally, it is necessary to be explicit about the meaning of the term ``preaspiration'', as it is often a term of convenience for a diverse range of surface phonetic phenomena which can arise in a variety of ways diachronically. The phonetic realization of reported preaspiration often spans vowel breathiness, glottal frication, and oral frication (\citealt{Silverman2003,hejna2016,chapters/16_Hejná}), and its phonological status is heterogeneous as well, with preaspiration sometimes analyzed as a coda /h/ segment (Icelandic: \citealp{arnason2011}; North Sámi: \citealp{balsbaal2012}; North Argyll Gaelic: \citealp{iosadetal2015}) but more often analyzed as a featural component of a following segment. Past typologies of preaspiration have limited the scope of their survey to preaspiration before stops, often with the stipulation that this narrow distribution constitutes ``true'' preaspiration \citep{Helgason2002,Silverman2003,Clayton:2010}. In this view, the distribution of [h] is a diagnostic for its status as preaspiration, with presence of [h] before a more diverse set of consonants suggesting that it is better analyzed as a coda fricative. The typological survey of preaspiration in \S\ref{sec-typology} does not support this restrictive definition of preaspiration, showing that the distribution of preaspiration is indisputably wider than just pre-stop contexts, and as is discussed later in \S\ref{sec-discussion}, the reasoning behind this restriction is not well motivated. Preaspiration found before stops does not have a special status, and it does not depend in any (synchronic) way on the laryngeal properties of the following segment: preaspiration may be found before segments that do not have phonological laryngeal contrasts. \textcitetv{chapters/16_Hejná} takes a similarly broad approach to the question of what might be considered preaspiration, and discusses the breadth of the related phenomena.  % : even in   as examples of preaspiration, preaspiration also occurs before fricatives \citep{helgason2002}.

For convenience and continuity with previous literature, I continue to use the term ``preaspiration'' here, despite the implication that the aspiration is inherently dependent on or affiliated with the following segment. For the purposes of the typology in \S\ref{sec-typology}, any instance of [h] directly before a consonant articulation may be considered ``preaspiration'', and in some cases voiceless dorsal fricatives in coda position may be considered aspiration as well, due to rampant fortition of [h] cross-linguistically. Although this may seem at first to cast a very broad net, and to conflate preaspiration as a phonological phenomenon with independent and unrelated coda /h/, this is in fact one of my core claims: synchronically, as a result of the moraic status of [h], most instances of preaspiration are indistinguishable from heterosegmental [hC] sequences.%
\footnote{Although I do not discuss specific examples here where it might be possible to distinguish preaspiration from a coda /h/, this issue will depend on the definition of preaspiration and the boundary between phonology and phonetics. \textcitetv{chapters/02_Iosad} argues that a distinction between phonological preaspiration and phonetic preaspiration is desirable, and this distinction is not fundamentally at odds with the position I take here. As phonological phenomena typically begin their life cycle as phonetic trends, there is inevitably a stage at which glottal frication could be acoustically discernible but not available to the phonology as an independent moraic segment. I would anticipate that phonetic preaspiration of this kind would appear associated to a moraic segment, making it difficult to discern where it is purely phonetic and where it is accessible to the phonology. However, further research on the life cycle and emergence of preaspiration may clarify this.}

However, I avoid the terms ``normative'' vs.\ ``non-normative'', which are commonly used in some previous work to refer to phonologized vs.\ variable preaspiration patterns \citep[e.g.,][]{Helgason2002,Clayton:2010,gordeevascobbie2007}. These terms are used inconsistently and set up a false equivalence between invariance and phonologization.
This concern is shared by \citeauthor{hejna2015} (\citeyear{hejna2015,hejna2019}) and \citeauthor{iosad2017-mfm} (\citeyear{iosad2017-mfm, chapters/02_Iosad}), and I align with their stance that the true phonological status of a pattern depends on its interactions with elements of phonological computation, and not on gradience or categoricity. Patterns in metrical constituency and prosodic association are part of the phonological ``module'' of the grammar \citep[e.g.,][]{bermudezotero2015}, and if preaspiration involves these elements, it should be considered part of the phonological computation, regardless of variance. In the typological survey presented below, both variant and invariant preaspiration are shown to interact with phonological quantity, syllable structure, and phonemic length, challenging the idea that there is a fundamental difference in the phonological status of preaspiration in ``normative'' vs.\ ``non-normative'' cases.

% \clearpage

\section{Typology}
\label{sec-typology}



\subsection{Icelandic}
\label{sec-icelandic}

Icelandic (\glottolog{icel1247}; North Germanic; Iceland) is perhaps the best-known reported case of preaspiration, with linguistic discussion of the phenomenon going back more than a century (\citealp{sweet1877,ofeigsson1920, malone1923, einarsson1931, einarsson1949, haugen1941, haugen1958, thrainsson1978, arnason1986, arnason2011}, \textit{inter alia}). Arguments that preaspiration is a distinct coda [h] in the language have been made for just as long, but early work focuses on the duration of preaspiration alone. 

There are two laryngeal series of stops in Icelandic, often labelled ``fortis'' and ``lenis''. The fortis stops are usually phonetically aspirated, while the lenis stops are unaspirated and sometimes voiced, which suggests that the relevant phonological contrast is in aspiration. Phonetic aspiration of the fortis consonants has been described as alternating between postaspiration and preaspiration. Word-initial fortis stops are always post-aspirated, and non-initial stops that either derive from geminates or are followed by /l/ or /n/ are preceded by preaspiration. This is illustrated below: data in (\ref{ex-icelandic-fortis-lenis}a) show the laryngeal contrast in initial position, (\ref{ex-icelandic-fortis-lenis}b) show contrasting historical fortis and lenis geminates, and (\ref{ex-icelandic-fortis-lenis}c) show preaspiration before stop-nasal sequences. Examples are presented in pairs, with the fortis stop listed first and lenis stop second.%
\footnote{\label{fn-icelandic-fortis-lenis}However, it is important to note that the fortis/lenis contrast only really remains in initial position, and is largely historical or orthographic elsewhere. The distribution of preaspiration signals the historical identity of non-initial stops, but if it is analyzed as synchronically unproductive or a distinct segment \citep[e.g.,][]{arnason2011}, this means the stop contrast itself is now limited to word-initial position. This is also the case for Danish \citep{basboll2005}.
}

\ea Icelandic fortis\slash lenis consonant pairs \citep[99, 106, 221]{arnason2011}\label{ex-icelandic-fortis-lenis}
\ea\NumTabs{5}	
	\ea 
	    \textit{par} \tab [pʰaːr̥] \tab `pair' \\
	    \textit{bar} \tab [paːr̥] \tab  `bar' 
	\ex
	    \textit{tal} \tab [tʰaːl̥] \tab `talk, speech' \\
	    \textit{dal} \tab [taːl̥] \tab  `valley' 
	\z
\ex	\ea \textit{koppi} \tab [kʰɔhpɪ] \tab `chamber pot.\textsc{dat}'\\
        \textit{kobbi} \tab [kʰɔpːɪ] \tab `young seal' 
	\ex \textit{hattur} \tab [hahtʏr̥]\tab `hat' \\
	    \textit{haddur} \tab [hatːʏr̥]\tab `hair' 
	\ex  \textit{bakka} \tab [pahka] \tab `bank.\textsc{acc}' \\
	     \textit{bagga} \tab [pakːa] \tab `bundle.\textsc{acc}' 
	\z
\ex	\ea \textit{opna} \tab [ɔhpna]   \tab `open.\textsc{inf}' \\
        \textit{ofna} \tab [ɔpna]    \tab `radiator.\textsc{pl}' 
	\ex \textit{batni} \tab [pahtnɪ] \tab `improve.\textsc{subj.sg}' \\
	    \textit{barni} \tab [patnɪ]  \tab `child.\textsc{dat}' 
	\ex \textit{sakna} \tab [sahkna] \tab `miss.\textsc{inf}' \\
	    \textit{sagna} \tab [sakna]  \tab `story.\textsc{gen.pl}' 
	\z
	\z
\z

In addition to the stops, Icelandic also has a laryngeal contrast in the sonorant series, with all liquids and nasals having voiceless counterparts. These voiceless sonorants are phonemic in onset position in Icelandic \citep{jessenpetursson1998}; they appear in codas as well, both word-finally and word-medially, where their phonemic status is contested. Word-medially, they are found in contexts where preaspiration would be expected, which has led some authors to argue that they should be considered an instance of preaspiration \citep[e.g.,][]{thrainsson1978,ringen1999}. In the interests of brevity, I do not discuss voiceless sonorants in this chapter, as they are clearly syllabified in the syllable before the stop.

Icelandic has length contrasts in both vowels and consonants, but these surface in a complementary way: word-initial syllables have primary stress and are phonologically heavy, requiring either a long vowel or a coda consonant. In other words, stressed vowels are long unless there is a moraic coda (e.g., a following geminate). In words of two or more syllables, this means that stressed vowels are long in open syllables. In monosyllabic words, vowels are long if there is no coda or if there is a simple coda, which is non-moraic word-finally in Icelandic.  This pattern is illustrated in (\ref{ex-icelandic-weight-complementarity}).%
\footnote{Data and transcriptions in (\ref{ex-icelandic-weight-complementarity}b--d) are from \citet{arnason2011}. The word pair in (\ref{ex-icelandic-weight-complementarity}a) is from \citet{garnes1976}, with my own IPA transcription matching \citeauthor{arnason2011}'s Icelandic conventions. Appropriate syllable boundaries have been added to all data.}

\ea Weight complementarity in heavy syllables \citep{garnes1976,arnason2011}\label{ex-icelandic-weight-complementarity}
\NumTabs{5}
\ea  \textit{völur}	 \tab  [ˈvøː.lʏr̥]	\tab  \gloss{rod} (archaic) \\
     \textit{völlur}\tab  [ˈvøt.lʏr̥]	\tab  `field, airport'
\ex  \textit{mani}	\tab  [ˈmaː.ni]	\tab  `young lady.\textsc{dat}'	\\
     \textit{manni}	\tab  [ˈman.ni] \tab  `man.\textsc{dat}' 
\ex  \textit{man}	\tab  [maːn]	\tab  `young lady.\textsc{acc}'	\\
     \textit{mann}	\tab  [manː] 	\tab  `man.\textsc{acc}' 
\ex  \textit{ein}	\tab  [eiːn] 	\tab  `one.\textsc{f}'		\\
     \textit{einn}	\tab  [eitn̥]	\tab  `one.\textsc{m}' 
\z
\z

In the examples in (\ref{ex-icelandic-weight-complementarity}a--b), medial clusters or geminates require the preceding stressed vowel to be short. In the examples in (\ref{ex-icelandic-weight-complementarity}c--d), vowels are required to be short in monosyllables with moraic codas or coda consonant clusters, but may be long before simple non-moraic codas. The pattern in monosyllables is seen more generally across the Germanic languages, where word-final consonants have been analyzed as extrametrical. This creates a prosodic distinction between single final consonants and final clusters or geminates (Icelandic: \citealp{iversonkesterson1989}; Swedish: \citealp{riad2014}; Norwegian: \citealp{kristoffersen2000}; English, Dutch: \citealp{kager1989}). 

I make use of the moraic structures in \figref{fig:ex-icelandic-moraic-structures} to represent this prosodic pattern. These representations are similar to those of \citet{ringen1999} and \citet{moren2001}, and build on previous models of foot and syllable structure expressing the same insight \citep[e.g.,][]{iversonkesterson1989,arnason2011}.

\begin{figure}[p]
\caption{Moraic structures showing the weight complementarity pattern in heavy syllables in Icelandic}
\label{fig:ex-icelandic-moraic-structures}
\subfigure[\orthog{völur} `rod']{
	\begin{tikzpicture}[moraic-structure]
		\node (c1) at (1,0) {v};	
		\node (v1) at (2,0) {øː};	
		\node (c2) at (3,0) {l};	
		\node (v2) at (4,0) {ʏ};	
		\node (c3) at (5,0) {r̥};	
		%
		\node (m1) at (2,1) {μ};
		\node (m2) at (2.75,1) {μ};
		\node (m3) at (4,1) {μ};
		%
		\node (s1) at (2,2) {σ};
		\node (s2) at (4,2) {σ};
		%
		\draw (v1.north) -- (m1.south);
		\draw (v1.north) -- (m2.south);
		\draw (v2.north) -- (m3.south);
		%
		\draw (m1.north) -- (s1.south);
		\draw (m2.north) -- (s1.south);
		\draw (m3.north) -- (s2.south);
		\draw (c1.north) -- (s1.south);
		\draw (c2.north) -- (s2.south);
		\draw (c3.north) -- (s2.south);
	\end{tikzpicture}
}%
\hspace{3em}
\subfigure[\orthog{völlur} `field']{
\begin{tikzpicture}[moraic-structure]
	\node (c1) at (1,0) {v};	
	\node (v1) at (2,0) {ø};	
	\node (c2) at (3,0) {t};	
	\node (c3) at (4,0) {l};	
	\node (v2) at (5,0) {ʏ};	
	\node (c4) at (6,0) {r̥};	
	%
	\node (m1) at (2,1) {μ};
	\node (m2) at (3,1) {μ};
	\node (m3) at (5,1) {μ};
	%
	\node (s1) at (2,2) {σ};
	\node (s2) at (5,2) {σ};
	%
	\draw (v1.north) -- (m1.south);
	%\draw (v1.north) -- (m2.south);
	\draw (v2.north) -- (m3.south);
	%
	\draw (m1.north) -- (s1.south);
	\draw (m2.north) -- (s1.south);
	\draw (m3.north) -- (s2.south);
	\draw (c1.north) -- (s1.south);
	\draw (c3.north) -- (s2.south);
	\draw (c2.north) -- (m2.south);
	\draw (c4.north) -- (s2.south);
	\end{tikzpicture}
}
\hspace{3em}
\subfigure[\orthog{manni} `man.\textsc{dat}']{
\hspace*{1em}
\begin{tikzpicture}[moraic-structure]
	\node (c1) at (1,0) {m};	
	\node (v1) at (2,0) {a};	
	\node (c2) at (3,0) {nː};
	\node (v2) at (4,0) {i};	
	%
	\node (m1) at (2,1) {μ};
	\node (m2) at (3,1) {μ};
	\node (m3) at (4,1) {μ};
	%
	\node (s1) at (2,2) {σ};
	\node (s2) at (4,2) {σ};
	%
	\draw (v1.north) -- (m1.south);
	\draw (v2.north) -- (m3.south);
	%
	\draw (m1.north) -- (s1.south);
	\draw (m2.north) -- (s1.south);
	\draw (m3.north) -- (s2.south);
	%
	\draw (c1.north) -- (s1.south);
	\draw (c2.north) -- (m2.south);
	\draw (c2.north) -- (s2.south);
\end{tikzpicture}
\hspace*{1ex}
}

\subfigure[\orthog{völ} `choice.\textsc{pl}']{
\begin{tikzpicture}[moraic-structure]
	\node (c1) at (1,0) {v};	
	\node (v1) at (2.5,0) {øː};	
	\node (c2) at (4,0) {l};	
	%
	\node (m1) at (2,1) {μ};
	\node (m2) at (3,1) {μ};
	%
	\node (s1) at (2.5,2) {σ};
	%
	\draw (v1.north) -- (m1.south);
	\draw (v1.north) -- (m2.south);
	%
	\draw (m1.north) -- (s1.south);
	\draw (m2.north) -- (s1.south);
	\draw [out=90,in=-160] (c1.north) to (s1.south);
	\draw [out=90,in=-20] (c2.north) to (s1.south);
\end{tikzpicture}
}
\hspace{3em}
\subfigure[\orthog{völl} `field.\textsc{acc}']{
\begin{tikzpicture}[moraic-structure]
	\node (c1) at (1,0) {v};	
	\node (v1) at (2,0) {ø};	
	\node (c1r) at (3,0) {t};	
	\node (c2) at (4,0) {l̥};	
	%
	\node (m1) at (2,1) {μ};
	\node (m2) at (3,1) {μ};
	%
	\node (s1) at (2.5,2) {σ};
	%
	\draw (v1.north) -- (m1.south);
	\draw (c1r.north) -- (m2.south);
	%
	\draw (m1.north) -- (s1.south);
	\draw (m2.north) -- (s1.south);
	\draw [out=90,in=-160] (c1.north) to (s1.south);
	\draw [out=90,in=-20] (c2.north) to (s1.south);
\end{tikzpicture}
}
\hspace{3em}
\subfigure[\orthog{mann} `man.\textsc{acc}']{
\hspace*{1em}
\begin{tikzpicture}[moraic-structure]
	\node (c1) at (1,0) {m};	
	\node (v1) at (2,0) {a};	
	\node (c2) at (3,0) {nː};	
	%
	\node (m1) at (2,1) {μ};
	\node (m2) at (3,1) {μ};
	%
	\node (s1) at (2,2) {σ};
	%
	\draw (v1.north) -- (m1.south);
	\draw (c1.north) -- (s1.south);
	\draw (c2.north) -- (m2.south);
	%
	\draw (m1.north) -- (s1.south);
	\draw (m2.north) -- (s1.south);
\end{tikzpicture}
\hspace*{1em}
}
\end{figure}

Comparing the vowel length patterns of words with preaspiration, it is immediately clear that these behave like words with medial or final clusters, as long vowels never co-occur with preaspiration. The examples in (\ref{ex-icelandic-preaspiration-weight}) show that preaspiration is moraic and takes up the entire second mora of a stressed syllable in all cases.%
\footnote{According to the phonotactics of Icelandic as generally described in the literature, \{pʰ,tʰ,kʰ\}+\{l,n\} consonant sequences as seen in \orthog{vakna} in (\ref{ex-icelandic-preaspiration-weight}c) are not word-medial onset clusters \citep[e.g.,][]{arnason2011}. However, the traditional diagnostic for the set of word-medial onsets in the Icelandic literature is the ability to appear after a long vowel, where it is clear that the preceding syllable has no coda. Since \{pʰ,tʰ,kʰ\}+\{l,n\} sequences always have a preceding [h], the preceding syllable is always closed. There is compelling evidence elsewhere in the language that these also form medial clusters, an analysis which is acknowledged by \citet[172]{arnason2011} as well; however, this is not critical to the arguments here. This syllabification is confirmed by the lengthening patterns illustrated below in (\ref{ex-arnason-preaspiration-focus}b), where the /h/ is moraic and not the /kʰ/.} 
Corresponding moraic structures are shown in \figref{ex-icelandic-preasp-moraic-structures}, illustrating the moraic association of preaspiration.

\ea Icelandic vowel length and preaspiration \citep{thrainsson1978,arnason2011}\footnote{As can be seen in this data, monophthongs and diphthongs show the same length alternations in open~vs.~closed syllables in Icelandic. Phonetic measurements of vowels and preaspiration in these contexts confirm that there are no differences between monophthongs and diphthongs in either vocalic duration or following preaspiration duration \citep{garnes1976,thrainsson1978}.}
\label{ex-icelandic-preaspiration-weight}
	\ea	\NumTabs{5} 
		 \textit{leit}	\tab\relax [leiːt]	    \tab `search'\\
	     \textit{heitt}	\tab\relax [heiht]		\tab `hot.\textsc{n}'
	\ex	 \textit{hatur} \tab\relax [haː.tʰʏr̥]	\tab `hate'\\
	     \textit{hattur}\tab\relax [hah.tʏr̥]	\tab `hat'
	\ex	 \textit{vekja}	\tab\relax [vɛː.kʰja]	\tab `wake s.o.\ up.\textsc{inf}'\\
	     \textit{vakna}	\tab\relax [vah.kna]	\tab `wake up.\textsc{inf}'
	\ex	 \textit{líta}  \tab\relax [liː.tʰa]	\tab `look.\textsc{inf}'\\
	     \textit{líttu}	\tab\relax [lih.tʏ]	    \tab `look.\textsc{imp}'
	\ex	 \textit{ljót}  \tab\relax [ljouːt]	    \tab `ugly.\textsc{f}'	\\
	     \textit{ljótt}	\tab\relax [ljouht]	    \tab `ugly.\textsc{n}'
    \z
\z


\begin{figure}[p]
\caption{Moraic structures showing association of Icelandic preaspiration to the second mora of a stressed syllable}
\label{ex-icelandic-preasp-moraic-structures}
\subfigure[\orthog{hatur} \gloss{hate}]{
\begin{tikzpicture}[moraic-structure]
	\node (c1) at (1,0) {h};	
	\node (v1) at (2,0) {aː};	
	\node (c2) at (3,0) {tʰ};	
	\node (v2) at (4,0) {ʏ};	
	\node (c3) at (5,0) {r̥};	
	%
	\node (m1) at (2,1) {μ};
	\node (m2) at (2.75,1) {μ};
	\node (m3) at (4,1) {μ};
	%
	\node (s1) at (2,2) {σ};
	\node (s2) at (4,2) {σ};
	%
	\draw (v1.north) -- (m1.south);
	\draw (v1.north) -- (m2.south);
	\draw (v2.north) -- (m3.south);
	%
	\draw (m1.north) -- (s1.south);
	\draw (m2.north) -- (s1.south);
	\draw (m3.north) -- (s2.south);
	\draw (c1.north) -- (s1.south);
	\draw (c2.north) -- (s2.south);
	\draw (c3.north) -- (s2.south);
\end{tikzpicture}
}
\hspace{3em}
\subfigure[\orthog{hattur} \gloss{hat}]{
\begin{tikzpicture}[moraic-structure]
	\node (c1) at (1,0) {h};	
	\node (v1) at (2,0) {a};	
	\node (c2) at (3,0) {h};	
	\node (c3) at (4,0) {t};	
	\node (v2) at (5,0) {ʏ};	
	\node (c4) at (6,0) {r̥};	
	%
	\node (m1) at (2,1) {μ};
	\node (m2) at (3,1) {μ};
	\node (m3) at (5,1) {μ};
	%
	\node (s1) at (2,2) {σ};
	\node (s2) at (5,2) {σ};
	%
	\draw (v1.north) -- (m1.south);
	%\draw (v1.north) -- (m2.south);
	\draw (v2.north) -- (m3.south);
	%
	\draw (m1.north) -- (s1.south);
	\draw (m2.north) -- (s1.south);
	\draw (m3.north) -- (s2.south);
	\draw (c1.north) -- (s1.south);
	\draw (c3.north) -- (s2.south);
	\draw (c2.north) -- (m2.south);
	\draw (c4.north) -- (s2.south);
\end{tikzpicture}
}

\subfigure[\orthog{leit} \gloss{search}]{
\begin{tikzpicture}[moraic-structure]
	\node (c1) at (1,0) {l};	
	\node (v1) at (2.5,0) {eiː};	
	\node (c2) at (4,0) {t};	
	%
	\node (m1) at (2,1) {μ};
	\node (m2) at (3,1) {μ};
	%
	\node (s1) at (2.5,2) {σ};
	%
	\draw (v1.north) -- (m1.south);
	\draw (v1.north) -- (m2.south);
	%
	\draw (m1.north) -- (s1.south);
	\draw (m2.north) -- (s1.south);
	\draw [out=90,in=-160] (c1.north) to (s1.south);
	\draw [out=90,in=-20] (c2.north) to (s1.south);
\end{tikzpicture}
}
\hspace{3em}
\subfigure[\orthog{heitt} \gloss{hot.\textsc{n}}]{
\begin{tikzpicture}[moraic-structure]
	\node (c1) at (1,0) {h};	
	\node (v1) at (2,0) {ei};	
	\node (c1r) at (3,0) {h};	
	\node (c2) at (4,0) {t};	
	%
	\node (m1) at (2,1) {μ};
	\node (m2) at (3,1) {μ};
	%
	\node (s1) at (2.5,2) {σ};
	%
	\draw (v1.north) -- (m1.south);
	\draw (c1r.north) -- (m2.south);
	%
	\draw (m1.north) -- (s1.south);
	\draw (m2.north) -- (s1.south);
	\draw [out=90,in=-160] (c1.north) to (s1.south);
	\draw [out=90,in=-20] (c2.north) to (s1.south);
\end{tikzpicture}
}
\end{figure}


This phonological evidence of complementarity between long vowels and preaspiration is supported by phonetic measurements of segmental duration. A production study by \citet{garnes1976} shows that the duration of long vowels is similar to the combined duration of short vowels and preaspiration across a variety of word types. This durational complementarity is the predicted outcome if preaspiration is taking up moraic weight within the syllable that would otherwise be associated to the vowel. 

Additional phonetic evidence that preaspiration is prosodically situated within a stressed syllable comes from {\citet{arnason1986,arnason2011}}. He observes that extra emphasis on a word, such as through contrastive focus, causes lengthening within the stressed syllable, specifically on the second mora of the syllable, as per the structures in \figref{fig:ex-preaspiration-moraic-structures}.%
\footnote{As {\citet{arnason1986,arnason2011}} does not work within a moraic framework, I have rephrased this in terms of moras. {\citeauthor{arnason2011}'s (\citeyear{arnason2011})} model of syllabic structure builds on a proposal by \citet{haugen1958} that the Icelandic syllabic nucleus is either Vː or VC. \citeauthor{arnason2011}'s prosodic representation consists of three constituents: onset, rhyme, and appendix. The last consonant of the syllable is typically the appendix, and considered extrametrical (i.e., non-moraic). The rhyme contains both the vowel nucleus and any following non-appendix consonant (i.e., the moraic segments in the syllable). Thus, \citeauthor{arnason2011}'s rhyme is always Vː or VC, and he observes that lengthening under focus occurs at the end of this constituent, increasing the duration of the consonant if it is present, or the vowel otherwise. Under a moraic model, this describes second mora lengthening.}
In words with a long vowel in the stressed syllable, the vowel lengthens under focus, while in words with a closed stressed syllable, the moraic coda lengthens. This is shown in the examples from {\citet[11--12]{arnason1986}} in (\ref{ex-arnason-hates-fish}). In words with preaspiration, the [h] lengthens under focus, as shown in (\ref{ex-arnason-preaspiration-focus}), which confirms it is in the coda of the stressed syllable. I reproduce \citeauthor{arnason1986}'s transcriptions of the emphasized words at the right,%
\footnote{Apart from (\ref{ex-arnason-hates-fish}a), where \citet{arnason1986} merely says the word has a ``very long [aː]''.}
and have added transcriptions of the expected production in an unemphasized context for comparison.

\ea 
\label{ex-arnason-hates-fish}
\ea \textit{Ég \textsc{hata} fisk!} \jambox[2.5in]{[haːˑ.ta]\hspace{3em} (cf.~unfocused [haː.ta])} 
	\sn \gloss{I \textsc{hate} fish!}
	\ex \textit{Ég \textsc{elska} fisk!} \jambox[2.5in]{[ɛlː.ska]\hspace{3em} (cf.~unfocused [ɛl.ska])} 
	\sn \gloss{I \textsc{love} fish!}
	% \clearpage
	\ex \textit{Ég bað um \textsc{ost}.} \jambox[2.5in]{[ɔsːt]\hspace{4.2em} (cf.~unfocused [ɔst])} 
	\sn \gloss{I asked for \textsc{cheese}.}
	\z
\ex
\label{ex-arnason-preaspiration-focus}
\ea \orthog{Þetta er \textsc{tappi}.} \jambox[2.5in]{[tahː.pɪ]\hspace{2.8em} (cf.~unfocused [tah.pɪ])} 
	\sn \gloss{This is a \textsc{tap}.}
	\ex \orthog{Þú átt að \textsc{vakna}.} \jambox[2.5in]{[vahː.kna]\hspace{1.9em} (cf.~unfocused [vah.kna])} 
	\sn \gloss{You are to \textsc{wake up}.}
	\ex \orthog{Mig vantar \textsc{hatt}.} \jambox[2.5in]{[hahːt]\hspace{3.3em} (cf.~unfocused [haht])} 
	\sn \gloss{I need a \textsc{hat}.}
	% \ex \orthog{\textsc{Epli} eru holl.} \jambox{[ɛhː.plɪ]} 
	% \sn \gloss{\textsc{Apples} are healthy.}
	\z
\z

The uneven increase in the duration of the [h] element rather than the following stop suggests that these are prosodically distinct segments rather than a single affricate-like heterogeneous segment. There is therefore strong evidence supporting the analysis of Icelandic preaspiration as a /h/ segment separately from the following stop, even though the historical development and morphology of the language make the original source of the preaspiration very clear. 

One possible conclusion about Icelandic based on these data is that preaspiration is not a productive process in the phonology of the language; under this view, preaspiration cannot be interpreted as a synchronic instantiation of [spread glottis] on the consonant, or as a gestural misalignment. However, an alternate analysis in which preaspiration is phonologically active is also available, based on [h] alternations within paradigms as well as production of foreign words by Icelandic native speakers. Nonetheless, this analysis does not require preaspiration to be a feature of the following stop. I argue that as preaspiration has a moraic basis, the appearance of [h] in alternations and foreign words is a part of maintaining the desired prosodic structure of Icelandic words. 

In the examples shown earlier in (\ref{ex-icelandic-preaspiration-weight}c--e), preaspiration was observed in morphophonological alternations (e.g., derived geminates: /ljout-t/ [ljouht] \gloss{ugly.\textsc{n}}; C+N sequences: /vak-na/ [vahkna] \gloss{wake up.\textsc{inf}}). With these kinds of data, it is hard to be certain whether the observed patterns are an active part of the phonology or are set of fossilized forms. However, we also see that loans from other languages into Icelandic contain preaspiration \citep{thrainsson1978}, and native Icelandic speakers have been observed for decades to produce preaspiration when speaking L2 languages like English and German \citep{haugen1958,thrainsson1978,sigurjonsson2015}. The data in (\ref{ex-icelandic-L2}) illustrate adaptation of foreign words to Icelandic phonology, with the first three rows showing the appearance of preaspiration. The words in (\ref{ex-icelandic-L2}a) are English loanwords into Icelandic (from \citealp{thrainsson1978}: 15 and \citealp{kvaransvavarsdottir2004}: 12), while (\ref{ex-icelandic-L2}b--c) are productions of English and German words respectively by L1 Icelandic speakers (from \citealp{thrainsson1978}: 13).

\ea Loanwords and L2 productions of foreign words by Icelandic speakers
\label{ex-icelandic-L2}
	\ea \begin{tabular}[t]{@{}OlG@{}}
		 sjoppa	    & [sjɔhpa]	& shop	   \\
		 lotterí	& [lɔhtɛri]	& lottery  \\
		 rokk		& [rɔhk]	& rock	   \\\midrule
		 boddí		& [pɔtti]	& body 	    \\
	\end{tabular}
	\ex \begin{tabular}[t]{@{}Ol@{}}
		  rip	& [rɪhp]	\\
		  met	& [mɛht]	\\
		  block	& [plɔhk]	\\\midrule
		  deep	& [diːp]	\\
		  coke	& [kʰouːk]	\\
	\end{tabular}
	\ex \begin{tabular}[t]{@{}Ol@{}}
		  Mitte & [mɪhtɛ] \\
		  mit 	& [mɪht] \\
		  Acker & [ahkɛr] \\\midrule
		  Miete & [miːtʰɛ] \\
		  Mut   & [muːtʰ] \\
	\end{tabular}
	\z
\z

Although these particular data are relatively old, productions of this kind are still typical of modern Icelandic learners of all ages \citep{sigurjonsson2015}. It is immediately apparent from the data in (\ref{ex-icelandic-L2}b--c) that foreign words with lax vowels followed by voiceless consonants (top three rows) are produced with short vowels and preaspiration, while words with tense vowels (bottom two rows) are produced with long vowels and no preaspiration. This is in line with restrictions on syllable shape in Icelandic: in order to have a short vowel, the syllable must have a moraic coda. Another possible adaptation of foreign words would have been to lengthen the following stop (yielding [rɪpp] for \gloss{rip}, for instance). In fact, this is how other English words with lax vowels are adapted, as seen in the final example in (\ref{ex-icelandic-L2}a) (data from \citealp{kvaransvavarsdottir2004}). This fits the structure of native Icelandic words, e.g., \orthog{odd}
[ɔtː] \gloss{point.\textsc{acc}} \citep[14]{thrainsson1978}. These data show that Icelandic speakers map laryngeal categories
from other Germanic languages onto the Icelandic fortis/lenis contrast, and are able to make use of preaspiration for prosodic purposes to close syllables containing lax vowels.

Ultimately, we can make a few key conclusions about preaspiration in Icelandic: it takes up an entire mora in the syllable before the stop and thus only follows short vowels, and it appears to serve a prosodic role in closing a stressed syllable containing a lax vowel. It is also worth noting that preaspiration is invariant in Icelandic. As mentioned earlier in \S\ref{sec-assumptions}, variable appearance of preaspiration in other languages has often been interpreted as a sign that it is ``merely'' a phonetic effect. While I do not necessarily adopt this as a criterion for its phonological status, preaspiration clearly plays a phonotactic role in Icelandic. 

The discussion around the segmental and prosodic status of preaspiration in Icelandic has illustrated the kind of evidence we can look for in other reported cases of preaspiration in order to determine its moraic status. Specifically, some of these telltale properties might be:

\begin{itemize}
	\item	affinity for stress, such as being limited to positions following a stressed vowel;
	\item 	lengthening under stress or emphasis of the preceding syllable;
	\item 	phonotactic restrictions on preaspiration after long vowels (or lower frequency of preaspiration after long vowels);
	\item	complementary phonetic duration with a preceding vowel ([Vː]$\sim$[V{h}]);
	\item 	complementary phonetic duration with a following geminate ([Cː]$\sim$[{h}C]).
\end{itemize}


Having set some expectations, we will next consider closely related Faroese, where preaspiration patterns are similar but have received a different phonological interpretation. 

\subsection{Faroese}
\label{sec-faroese}

Faroese (\glottolog{faro1244}; North Germanic; Faroe Islands, Denmark) is very closely related to Icelandic, and phonologically quite similar in the consonantal domain, with voiceless aspirated fortis stops contrasting with unaspirated lenis stops that may be realized with voicing. The apparent positional allophony between postaspiration and preaspiration is present in Faroese as well, with initial fortis stops being phonetically postaspirated, and preaspiration found before fortis geminates and fortis stop-nasal sequences. This is illustrated by the Tórshavn Faroese%
\footnote{\citet{Helgason2002} (following \citealp{petersen1995}) describes three main dialect regions for Faroese, which he labels numerically. I refer to these regions geographically: \citeauthor{Helgason2002}'s Area 1 (Mykines, Vágar, Eysturoy, northern Streymoy) is the northwestern dialect, Area 2 (Norðoyar, southern Streymoy) is the central--eastern dialect, and Area 3 (Sandoy, Suðuroy) is the southern dialect. Tórshavn Faroese is one of the central--eastern dialects.}
data in (\ref{ex-far-stop-laryngeal-contrasts}), where fortis stops are given first in each word pair and lenis ones second.%
\footnote{As mentioned in Footnote \ref{fn-icelandic-fortis-lenis} for Icelandic, this stop contrast is now limited to word-initial position in most Faroese dialects. Under an analysis of preaspiration as a distinct element that is no longer a feature of the stop laryngeal contrast, there is no laryngeal contrast in Faroese geminates. Both fortis and lenis stops are unaspirated following long vowels in the central--eastern dialect, and voiced following long vowels in the southern dialect. Only the northwestern dialect still contrasts aspiration after long vowels \citep{Helgason2002}.}
The first key difference between Faroese and Icelandic is already visible in the long medial stops in (\ref{ex-far-stop-laryngeal-contrasts}b): Faroese has retained the historical fortis geminates, while in Icelandic these have all undergone degemination (and only lenis geminates are still present).%


\ea Tórshavn Faroese fortis and lenis stops \citep{arnason2011}%[118, 176, 228, 253, 298]
\label{ex-far-stop-laryngeal-contrasts}
\NumTabs{5}
\ea	 
\ea \textit{peika}	\tab\relax [pʰaiːka]	\tab `to point'\\	
	 \textit{bera}  \tab\relax [peːɹa]	    \tab `to carry'\\
\ex \textit{tosa}  \tab\relax [tʰoːsa]	    \tab `to speak'\\
	 \textit{dalur}	\tab\relax [tɛaːlʊɹ]	\tab `valley'\\
	 \ex \textit{kemur}	\tab\relax [tʃʰeːmʊɹ]   \tab `comes'\\		
	 \textit{geva} 	\tab\relax [tʃeːva] 	\tab `to give'\\
	 \ex \textit{kola} 	\tab\relax [kʰoːla] 	\tab `lamp'\\		
	 \textit{gamal}	\tab\relax [kɛaːmal]	\tab `old'\\
     \z
	
\ex	 
\ea \textit{hoppa} \tab\relax [hɔhpːa]  \tab `to hop'\\ 
	 \textit{pabba} \tab\relax [pʰapːa]  \tab `father'\\
	 \ex \textit{hatt}  \tab\relax [hahtː]   \tab `hat'\\ 
	 \textit{hædd}  \tab\relax [hatː]    \tab `height, level'\\
	 \ex \textit{takka} \tab\relax [tʰahkːa] \tab `to thank'\\ 
	 \textit{nagga} \tab\relax [nakːa]   \tab `to shiver'\\
     \z
	
\ex	
\ea \textit{vatnið} \tab\relax [vahtnɪ] \tab `water.\textsc{def}'\\ 
	 \textit{oynni}  \tab\relax [ɔitnɪ]  \tab `island.\textsc{def.dat}'\\
	 \ex \textit{vakna}  \tab\relax [vahkna] \tab `to wake up'\\ 
	 \textit{eygna}  \tab\relax [ɛkna]   \tab `eyes.\textsc{gen}'\\
     \z
	\z
\z

As in Icelandic, most dialects of Faroese only permit preaspiration following a short vowel.
This is visible in the first word [pʰaiːka] in (\ref{ex-far-stop-laryngeal-contrasts}a), where there is no aspiration between the long vowel and the historically fortis stop, compared to the fortis examples in (\ref{ex-far-stop-laryngeal-contrasts}b), where preaspiration is seen between short vowels and geminates (e.g., [hɔhpːa]). This pattern immediately suggests that preaspiration bears moraic weight in Faroese. An exception to this pattern in the northwest dialect will be further discussed below.

Stress in native Faroese words is overwhelmingly on the first syllable of the word. This means that the examples of preaspiration seen above all occur immediately following a stressed vowel. In stressed syllables, Faroese also exhibits a complementary pattern in consonant and vowel duration, similar to that of Icelandic. The details of this length alternation are somewhat more complicated than in Icelandic due to historical vowel shifts in Faroese, as well as significant but underdescribed dialectal variation \citep{lockwood1955,oneil1964,cathey1997}. Although the details of these vowel changes are outside the scope of this chapter, and I will largely cite data from the Tórshavn dialect of Faroese \citep{arnason2011}, some relevant dialectal variation in preaspiration is discussed later. Some examples of the quantity patterns are given in (\ref{ex-far-vowel-length-alternations}). 

\ea Durational complementarity in heavy stressed syllables \citep{arnason2011}
\label{ex-far-vowel-length-alternations}
\NumTabs{6}
\ea	 
\ea \textit{mytisk} \tab\relax [myː.tɪsk]  \tab `mythical'\\ 
     \textit{mystisk} \tab\relax [mys.tɪsk]  \tab `mysterious'\\
	 \ex \textit{tola} \tab\relax [tʰoː.la]   \tab `to endure'\\	
	 \textit{toldi} \tab\relax [tɔl.tɪ]    \tab `endured'\\
	 \ex \textit{sleta}\tab\relax [stleː.ta]  \tab `sleigh'\\	
	 \textit{hella}\tab\relax [hɛt.la]    \tab `flat rock'\\
	 \ex \textit{gloyma} \tab\relax [klɔiː.ma]	\tab `to forget'\\ 	
	 \textit{gloymdi} \tab\relax [klɔim.tɪ]  \tab `forgot'\\
	 \ex \textit{pápi} \tab\relax [pʰɔɑː.pɪ] 	\tab `father'\\	
	 \textit{hoppa} \tab\relax [hɔhp.pa]	\tab `to hop'
     \z
	
\ex	 
\ea \textit{jól}   \tab\relax [jɔuːl]	\tab `Yule'\\	
	 \textit{mjólk} \tab\relax [mjœl̥k]	\tab `milk'\\
	 \ex \textit{blað}	\tab\relax [plɛaː]	\tab `page'\\  
	 \textit{hædd}  \tab\relax [hatː]   \tab `height, level'\\
	 \ex \textit{ljós}  \tab\relax [ʎɔuːs]	\tab `light'\\
	 \textit{ljótt} \tab\relax [ʎœhtː]  \tab `ugly.\textsc{n}'
     \z
	 \z
\z

The interesting question here is whether preaspiration is limited to stressed positions, however this proves difficult to establish. Due to Faroese stress placement, morpheme size, and phonotactics, I have not found examples of definitively unstressed syllables where preaspiration might also be expected to appear. Native roots are generally monosyllabic or disyllabic, and disyllabic roots seem to have a limited set of possible word-final codas (/n, r, l, rt, st, sk/), none of which allow for assessment of preaspiration.%
\footnote{The final /rt/ codas do surface as [ɹ̥t] or [ɻ̊ʈ], with a voiceless liquid, but as noted for Icelandic, even if this devoicing is analyzed as the same phenomenon as preaspiration synchronically, in a prosodic sense it is unambiguously a moraic coda consonant.}
In compounds, roots tend to have at least secondary stress, making it difficult to find completely unstressed preaspiration \citep{lockwood1955}.%It is unclear whether syllables with secondary stress are also phonologically heavy; 
\footnote{For instance, \citet{arnason2011} does not transcribe preaspiration at the end of the place name \orthog{Klaksvík}, where it is theoretically possible. However, it is unclear to me whether \orthog{vík} [vʊik] \gloss{bay} would be expected to have secondary stress, as this is a place name that may have been lexicalized. Additional data on secondary stress and preaspiration would be valuable here.}

The one situation where I have come across preaspiration outside of a primary stressed syllable is in examples of words with non-initial primary stress: \orthog{aftaná} [aʰtaˈnɔɑː] \gloss{behind, afterwards}; \orthog{afturum} [aʰtəˈɹʊmː] \gloss{behind, rearwards}; \orthog{afturat} [aʰˈtrɛaːt] \gloss{furthermore}; \orthog{upprunaliga} [ʊʰpːˈruːnalija] \gloss{originally} \citep[277]{arnason2011}.%
\footnote{For consistency, and in line with my proposal that preaspiration is not the inverse of postaspiration and bears moraic weight, I consistently transcribe it as [h] throughout this typological survey. However, in this instance I reproduce \citeauthor{arnason2011}'s transcriptions as they are.} 
In most of these words, the preaspiration is actually a lenited form of \hbox{/f/}, unlike ``actual'' preaspiration (found before [Cː] or [C{n}]). I find it interesting that \citet{arnason2011} transcribes this debuccalized coda fricative with a superscript [ʰ], the same way that he transcribes preaspiration in [ʊʰpːˈruːnalija] and elsewhere. This equivalent transcription suggests that he finds these to be phonetically comparable. Establishing whether this fricative debuccalization has the same prosodic and durational properties as preaspiration elsewhere in the language is a promising avenue for future work. As far as stress is concerned, \citet{arnason2011} refers to these as unstressed prefixes, but the morphologically complex nature of these data means that it is difficult to establish whether secondary stress is truly absent. 

% Loanword data provides a suggestion of a connection between stress and preaspiration in Faroese: loans from Greenlandic like  \orthog{kajakk} [kʰa.ˈjahk] \gloss{kayak} and \orthog{anorakkur} [anoˈrahkːʊr] \gloss{anorak} show preaspiration, but also have nonstandard primary stress on the second and third syllable. Additional data on Faroese loanwords would be valuable in establishing  
%This could be interpreted as a weight-sensitive stress assignment, where a structurally heavy second syllable causes a shift in primary word stress; additional work on Faroese loanwords would be quite valuable here. Thus, considering both the absence of environments for preaspiration outside of stressed syllables, and this possible weight-related stress shift, I tentatively conclude that Faroese preaspiration shows a connection to stress in Faroese as well, through moraic weight.

The maintenance of fortis geminates in Faroese, alongside the appearance of preaspiration, indicates that the moraic structure of Faroese is different than that of Icelandic. Like Icelandic, Faroese has non-moraic final consonants, but the moraic association seen in stressed syllables is different, following the pattern of \figref{fig:ex-preaspiration-moraic-structures}b. Proposed moraic structures for Faroese preaspiration are given in \figref{ex-faroese-preasp-moraic-structures}, which show the sharing of the second mora between preaspiration and the following stop.


% Arnason does have preaspiration in unstressed syllables in a few places (e.g., , but these apper to be from lenition of /f/

\begin{figure}
\caption{Proposed moraic structures for Faroese preaspiration}
\label{ex-faroese-preasp-moraic-structures}

%\parbox[t]{0.3\textwidth}{
%\TC\quad\orthog{nagga} \gloss{to shiver} \\
%\begin{tikzpicture}[baseline={([yshift=-1em]current bounding box.north)},moraic-structure]
%	\node (c1) at (1,0) {n};	
%	\node (v1) at (2,0) {aː};	
%	\node (c2) at (3,0) {tʰ};	
%	\node (v2) at (4,0) {ʏ};	
%	\node (c3) at (5,0) {r};	
%	%
%	\node (m1) at (2,1) {μ};
%	\node (m2) at (2.75,1) {μ};
%	\node (m3) at (4,1) {μ};
%	%
%	\node (s1) at (2,2) {σ};
%	\node (s2) at (4,2) {σ};
%	%
%	\draw (v1.north) -- (m1.south);
%	\draw (v1.north) -- (m2.south);
%	\draw (v2.north) -- (m3.south);
%	%
%	\draw (m1.north) -- (s1.south);
%	\draw (m2.north) -- (s1.south);
%	\draw (m3.north) -- (s2.south);
%	\draw (c1.north) -- (s1.south);
%	\draw (c2.north) -- (s2.south);
%	\draw (c3.north) -- (s2.south);
%\end{tikzpicture}
%}
\subfigure[\orthog{takka} \gloss{to thank}]{
\begin{tikzpicture}[moraic-structure]
	\node (c1) at (1,0) {tʰ};	
	\node (v1) at (2,0) {a};	
	\node (c2) at (3,0) {h};	
	\node (c3) at (4,0) {kː};	
	\node (v2) at (5,0) {a};	
	%
	\node (m1) at (2,1) {μ};
	\node (m2) at (3,1) {μ};
	\node (m3) at (5,1) {μ};
	%
	\node (s1) at (2,2) {σ};
	\node (s2) at (5,2) {σ};
	%
	\draw (v1.north) -- (m1.south);
	%\draw (v1.north) -- (m2.south);
	\draw (v2.north) -- (m3.south);
	%
	\draw (m1.north) -- (s1.south);
	\draw (m2.north) -- (s1.south);
	\draw (m3.north) -- (s2.south);
	\draw (c1.north) -- (s1.south);
	\draw (c3.north) -- (m2.south);
	\draw (c3.north) -- (s2.south);
	\draw (c2.north) -- (m2.south);
\end{tikzpicture}
}%
%
%\parbox[t]{0.3\textwidth}{
%\TC\quad\orthog{ljós} \gloss{light} \\
%\hspace*{1em}\begin{tikzpicture}[baseline={([yshift=-1em]current bounding box.north)},moraic-structure]
%	\node (c1) at (1,0) {ʎ};	
%	\node (v1) at (2.5,0) {ɔuː};	
%	\node (c2) at (4,0) {s};	
%	%
%	\node (m1) at (2,1) {μ};
%	\node (m2) at (3,1) {μ};
%	%
%	\node (s1) at (2.5,2) {σ};
%	%
%	\draw (v1.north) -- (m1.south);
%	\draw (v1.north) -- (m2.south);
%	%
%	\draw (m1.north) -- (s1.south);
%	\draw (m2.north) -- (s1.south);
%	\draw [out=90,in=-160] (c1.north) to (s1.south);
%	\draw [out=90,in=-20] (c2.north) to (s1.south);
%\end{tikzpicture}
%}
\hspace{3em}
\subfigure[\orthog{ljótt} `{ugly.\textsc{n}'}]{
\begin{tikzpicture}[moraic-structure]
	\node (c1) at (1,0) {ʎ};	
	\node (v1) at (2,0) {œ};	
	\node (c1r) at (3,0) {h};	
	\node (c2) at (4,0) {tː};	
	%
	\node (m1) at (2,1) {μ};
	\node (m2) at (3,1) {μ};
	%
	\node (s1) at (2,2) {σ};
	%
	\draw (v1.north) -- (m1.south);
	\draw (c1r.north) -- (m2.south);
	%
	\draw (m1.north) -- (s1.south);
	\draw (m2.north) -- (s1.south);
	\draw (m2.south) -- (c2.north);
	\draw (c1.north) to (s1.south);
	% \draw [out=90,in=-160] (c1.north) to (s1.south);
	% \draw [out=90,in=-20] (c2.north) to (s1.south);
\end{tikzpicture}
}
\end{figure}

The absence of degemination in Faroese fortis stops leads {\citet[230]{arnason2011}} to conclude that preaspiration is a ``subsegmental entity'' in the language, unlike its status as a full segment in Icelandic. He supports this position with evidence from the same kind of ``stress test'' described above for Icelandic (\ref{ex-arnason-hates-fish}--\ref{ex-arnason-preaspiration-focus}), in which words are produced with strong emphasis to lengthen the second mora and determine prosodic constituency \citep{arnason2011,schaeferarnason2012}. In Faroese, \citet{arnason2011} reports that the stop following preaspiration lengthens, which he interprets as supporting a subsegmental analysis of preaspiration.

Although this diagnostic yields different results in Faroese and Icelandic, this is unsurprising if the two languages have different moraic structures for preaspiration. In Icelandic, preaspiration occupies the entire second mora, and can lengthen independently of the following non-moraic consonant. In Faroese, following the proposed moraic structures in \figref{ex-faroese-preasp-moraic-structures}, a lengthening of the second mora should result in longer durations for both the preaspiration and the stop closure, as the weight is distributed between the two. This is precisely the pattern seen in the preliminary data reported by \citet{arnason2011}.%
\footnote{The published data in \citet{arnason2011} is quite preliminary, and the subsequent work by \hbox{\citet{schaeferarnason2012}} does not appear to have been published yet. However, I am assuming that the trends reported by \citet{arnason2011} are upheld by the full experimental results of \citet{schaeferarnason2012}.} Although \citeauthor{arnason2011} summarizes the results as stop lengthening under stress, both preaspiration duration and stop duration are seen to increase. 
% in comparison to the preaspiration lengthening separately from the stop under stress in Icelandic,

\citeauthor{arnason2011}'s analysis that preaspiration is ``subsegmental'' in Faroese does not predict this kind of coordinated lengthening. We do not necessarily expect a longer stop closure to imply a longer preaspiration duration. Following \citet{shawetal2021}, multiple gestures associated to a single complex segment are coordinated at their onset, whereas gestures making up a segment sequence have the onset of the second gesture coupled to the offset of the first. If we interpret preaspiration as a [spread glottis] feature on the stop, the onset of the laryngeal spreading gesture should be coupled (with some offset) to the onset of the oral gesture, meaning that the duration of the consonant should not affect the duration of the laryngeal component. Therefore, an analysis in which preaspiration is segmentally distinct and has its own moraic weight is more appealing here.
	
For the varieties of Faroese described thus far, the evidence for the distinct prosodic status of preaspiration appears strong. One small complication in this picture comes from northwestern dialects of Faroese, where preaspiration is reportedly possible after non-high long vowels \citep{Helgason2002}. In this dialect area, preaspiration is not found after long high monophthongs or long diphthongs ending with high vowels, but is found after other long vowels. Two of \citeauthor{Helgason2002}'s four Faroese speakers display this pattern, for which he provides the transcriptions in (\ref{ex-far-northwestern-data}).%
\footnote{One of the two speakers also categorized /eː/ as a high vowel, and thus would not produce preaspiration in \orthog{eta} \gloss{eat}.}

\vbox{
\ea Northwestern Faroese preaspiration \citep[57]{Helgason2002}
\label{ex-far-northwestern-data}
	\ea  \NumTabs{5}
	     \textit{bátin}		\tab\relax [bɔ̟͡ɐːht̪ən]	\tab `boat.\textsc{def}'\\
		 \textit{eta} 		\tab\relax [eːht̪a]	    \tab `eat'	
	\ex	 \textit{vík}		\tab\relax [vʊ͡iːkʰ]	    \tab `bay'\\
		 \textit{lykil}		\tab\relax [liːtʃɘl]	\tab `key' 	
	\z
\z
}

Although this height-based prosodic pattern is interesting, the data does not rule out a moraic account of preaspiration.  It is possible that northwestern Faroese dialects allow the weight of preaspiration to be shared with the coda before geminates, or with the vowel after long vowels, with the constraint that long high vowels must take up a full two moras.%
\footnote{This restriction could be based on vowel tenseness, which is correlated to length. Vowel tenseness shows interactions with preaspiration and laryngeal features in southeastern Welsh as well \citep{iosad2023}.}
Since this preaspiration after long vowels is necessarily followed by a non-moraic consonant, the weight alternation in \figref{ex-faroese-nw-moraic-structures} is possible, although this should be confirmed through phonetic investigation of segment duration ratios. 

 % báti		& [bɔ̟͡ɐːt̪ɛ]	
 % mat 		& [mɛ̠͡ɐːt̪]	

\begin{figure}
\caption{Proposed moraic structures for preaspiration in northwestern Faroese dialects, where it is attested following non-high long vowels}
\label{ex-faroese-nw-moraic-structures}
\subfigure[\orthog{takka} \gloss{to thank}]{
\begin{tikzpicture}[moraic-structure]
	\node (c1) at (1,0) {tʰ};	
	\node (v1) at (2,0) {a};	
	\node (c2) at (3,0) {h};	
	\node (c3) at (4,0) {kː};	
	\node (v2) at (5,0) {a};	
	%
	\node (m1) at (2,1) {μ};
	\node (m2) at (3,1) {μ};
	\node (m3) at (5,1) {μ};
	%
	\node (s1) at (2,2) {σ};
	\node (s2) at (5,2) {σ};
	%
	\draw (v1.north) -- (m1.south);
	%\draw (v1.north) -- (m2.south);
	\draw (v2.north) -- (m3.south);
	%
	\draw (m1.north) -- (s1.south);
	\draw (m2.north) -- (s1.south);
	\draw (m3.north) -- (s2.south);
	\draw (c1.north) -- (s1.south);
	\draw (c3.north) -- (m2.south);
	\draw (c3.north) -- (s2.south);
	\draw (c2.north) -- (m2.south);
\end{tikzpicture}
}
\hspace{3em}
\subfigure[\orthog{bátin} \gloss{boat.\textsc{def}}]{
\begin{tikzpicture}[moraic-structure]
	\node (c1) at (1,0) {b};	
	\node (v1) at (2,0) {ɔ̟͡ɐː};	
	\node (c2) at (3,0) {h};	
	\node (c3) at (4,0) {t̪};	
	\node (v2) at (5,0) {ə};	
	\node (c4) at (6,0) {n};	
	%
	\node (m1) at (2,1) {μ};
	\node (m2) at (3,1) {μ};
	\node (m3) at (5,1) {μ};
	%
	\node (s1) at (2,2) {σ};
	\node (s2) at (5,2) {σ};
	%
	\draw (v1.north) -- (m1.south);
	\draw (v1.north) -- (m2.south);
	%\draw (v1.north) -- (m2.south);
	\draw (v2.north) -- (m3.south);
	%
	\draw (m1.north) -- (s1.south);
	\draw (m2.north) -- (s1.south);
	\draw (m3.north) -- (s2.south);
	\draw (c1.north) -- (s1.south);
	\draw (c2.north) -- (m2.south);
	\draw (c3.north) -- (s2.south);
	\draw (c4.north) -- (s2.south);
\end{tikzpicture}
}
\end{figure}

\citet{Helgason2002} does report phonetic measurements for his Faroese speakers, but more data is needed on some crucial comparisons in order to confirm or reject this proposal. \citeauthor{Helgason2002} reports averaged segment durations for his four speakers, combining across the two dialect groups present, which he admits obscures some asymmetries in the data. On average, preaspiration appears to reduce the length of both the preceding vowel and the following stop, which might be a result of averaging the weight alternations in the structures above. Some trends are clear from his plots divided by speaker: northwestern dialect speakers show approximately similar preaspiration durations in both [Vː{h}C] and [V{h}Cː] words, which is predicted by the consistent 0.5 μ weight that preaspiration has in both word shapes. Vowels are also indeed longer in [Vː{h}C] words than [V{h}Cː] words, when comparing words with similar preaspiration duration. 
However, what is missing as support for these specific moraic structures is a focused comparison of vowel length in words with and without preaspiration. We predict significant shortening of long vowels before preaspiration ([VːC] vs.\ [Vː{h}C] environments), to a much greater degree than short vowels before preaspiration ([VCː] vs.\ [V{h}Cː] environments). \citet{Helgason2002} does report that preaspiration duration in Tórshavn Faroese is correlated with preceding vowel length for all of his speakers, which is to be expected if preaspiration is prosodically within the same syllable as the vowel. %
% \footnote{In words where preaspiration duration is 0, vowel length does not appear significantly different between VːC and VCː words for \citeauthor{helgason2002}'s (\citeyear{helgason2002}) speakers: the points overlap and in fact the longest absolute durations are in VCː words. }

{\citeauthor{Helgason2002}} (\citeyear{Helgason2002}: 161--167) also has a lengthy discussion of phonetic variability and oral coarticulation in Faroese preaspiration.\footnote{Thanks to an anonymous reviewer for pointing this out specifically.} He does not have a similar discussion about Icelandic (as he analyzes phonetic data from Faroese but not from Icelandic), and there is no work explicitly assessing the degree of oral coarticulation in Icelandic preaspiration, but it does appear that Faroese may be more prone to oral fortition of preaspiration than Icelandic is. Oral fortition of preaspiration is commonly reported cross-linguistically, and is sometimes judged to be a representative of a late stage of diachronic development or a step towards the loss of preaspiration \citep[e.g.,][]{Silverman2003,Clayton:2010}. In the context of the moraic account of preaspiration discussed here, I suggest that the articulatory differences between Icelandic and Faroese could be the result of either the shorter duration of preaspiration (making it more prone to coarticulation) or some influence of mora sharing on coarticulation. Although ultimately language-specific phonetic implementation is somewhat idiosyncratic, it is worth examining in the future whether there are cross-linguistic trends between oral coarticulation of preaspiration and its phonetic duration or moraic association.

Finally, I will briefly touch on the issue of variability in Faroese preaspiration. {\citeauthor{arnason2015}} (\citeyear{arnason2015}: 132) notes that preaspiration is variable on medial consonants, describing it as a ``choice between preaspiration \ldots\ or no aspiration''. This description suggests that preaspiration is variable in its incidence, rather than in its phonetic realization, and the associated figure indicates that the appearance of preaspiration is particularly low in the southern dialect region.\footnote{The figure that \citet{arnason2015} provides only shows an average ``grade'' that speakers were given for their use of preaspiration, which I gather reflects average rate of use, but it is not clear to me exactly how the grading was done. It is also unclear whether the variability is within speakers or between speakers or both.} However, more precise details about the variability are difficult to determine from \citeauthor{arnason2015}'s description. For the issues discussed in this chapter, the most important questions concern the durations of the segments when preaspiration is present compared to when it is not. Based on the moraic structures proposed in Figure \ref{fig:ex-preaspiration-moraic-structures}, I would predict that in utterances without preaspiration, either the preceding vowel would lengthen (occupying 1.5 μ) or the following consonant would lengthen (occupying the entire second mora). Crucially, as preaspiration is a prosodic phenomenon and associated with moraic weight, we expect some compensatory segmental length change when it is absent, and not just a shorter word. Variation in the rate of occurrence of preaspiration is not in any way troublesome for this theory, and does not inherently affect the status of preaspiration within the prosodic phonology (\citealp{iosad2017-mfm,hejna2019}; see also the variability in Swedish preaspiration in \S\ref{sec-swedish-norwegian}).

In sum, phonetic evidence from duration ratios in Faroese provides support for the claim that preaspiration bears moraic weight in Faroese, and is not simply a featural property of the following stop. The prosodic structures proposed for Faroese preaspiration provide a phonological proposal about how this phenomenon differs from that seen in Icelandic, in a way that reflects the similarities between these systems and their durational relations with a preceding vowel. Future work on the patterns of Faroese preaspiration will hopefully help confirm or reject the structures proposed above.


\subsection{Swedish \& Norwegian}
\label{sec-swedish-norwegian}

In both Swedish (\glottolog{swed1254}; North Germanic; Sweden) and Norwegian (\glottolog{norw1258}; North Germanic; Norway), stressed syllables are necessarily heavy, and long vowels and moraic codas are in complementary distribution \citep{kristoffersen2000,riad2014}. In both languages, preaspiration is generally reported before voiceless (or fortis) stops and affricates, particularly in medial post-tonic position. The fortis stops are postaspirated in initial position, unless they follow /s/ in an onset cluster \citep{kristoffersen2000,riad2014}. Preaspiration is widespread across Swedish and Norwegian dialects, but is more common and more prominent in some dialects than others, and its frequency of use varies by speaker and lexical item \citep{Helgason2002,tronnier2002,wretlingetal2002,wretlingetal2003,iosad2018}. This is summarized in (\ref{ex-swedia-quantity-pairs}) with data from Central Standard Swedish. Data for Norwegian are extremely similar and are not given here \citep[see][]{kristoffersen2000}.

\vbox{
\ea Central Swedish quantity pairs ({\citealp[59]{schaeffler2005}}, {\citealp[159]{riad2014}})\\[3pt]
\label{ex-swedia-quantity-pairs}
\ea \begin{tabular}[t]{r O  r @{ $\sim$ } l  G}
	i. & tak	& [taːk] & [taːhk]	& roof	\\
	& tack	& [takː] & [tahkː] & thanks \\
	ii. & låt	& [loːt] & [loːht]	& song	\\
	& lott	& [lɔtː] & [lɔhtː] & fate \\
	iii. & dit	& [diːt] & [diːçt]	& there	\\
	& ditt	& [dɪtː] & [dɪçtː] & yours \\
	\end{tabular}
	\ex \begin{tabular}[t]{r O r @{ $\sim$ } l  G}
	i. & såpa	& [ˈsoː.pa] & [ˈsoːh.pa] &soft soap	\\
	& soppa & [ˈsɔp.pa]  & [ˈsɔhp.pa] & soup \\
	ii. & puta	& [ˈpʰʉ̟ː.ta] & [ˈpʰʉ̟ːh.ta] &to pout	\\
	& putta & [ˈpʰɵt.ta] & [ˈpʰɵht.ta] & to push \\
	iii. & läka	& [ˈlɛː.ka] & [ˈlɛːh.ka] & to heal	\\
	& läcka & [ˈlɛ̝k.ka]  & [ˈlɛ̝hk.ka] & to leak \\
	\end{tabular}
	\z
\z
}

The situation in Swedish and Norwegian is largely similar to that discussed for Icelandic and Faroese in \S\ref{sec-icelandic} and \S\ref{sec-faroese}. As in those languages, \citet{Helgason2002} reports that the duration of preaspiration increases as the duration of the preceding vowel increases, suggesting that these are prosodically within the same syllable. The main differences are that preaspiration is much more variable%
\footnote{Of course, it is possible that preaspiration could be just as variable in Faroese, but the details of this are unclear.}
both between and within speakers in Swedish and Norwegian, with some speakers having frequent and significant preaspiration and others less or none at all, and that preaspiration is reported before both short and long vowels. Because of this degree of variability, and relatively short phonetic duration of preaspiration in the standard varieties, preaspiration is often not included in descriptions of these languages. In fact, the book-length phonological descriptions by \citet{riad2014} and \citet{kristoffersen2000} do not discuss preaspiration. However, as noted in \S\ref{sec-assumptions}, variability is not evidence that a phenomenon is not phonological, and as illustrated in \figref{fig:ex-preaspiration-moraic-structures}, under a mora-sharing approach, moraic preaspiration is structurally possible after both short and long vowels. The moraic structures necessary to represent this quantity distribution pattern are shown in \figref{ex-CSS-quantity}; they are essentially the same as those proposed in \figref{ex-faroese-nw-moraic-structures} for non-high vowels in northwestern Faroese. 
In these figures, dotted lines represent associations that are only present when preaspiration is produced.


\begin{figure} 
\caption{Moraic structures for preaspiration in Central Standard Swedish}
\label{ex-CSS-quantity}
\subfigure[\orthog{läka} \gloss{to heal}]{
\begin{tikzpicture}
		\node [seg] (c1) at (0,0) {l};
		\node [seg] (v1) at (1,0) {ɛː};	
		\node [seg] (c2) at (2,0) {(h)};
		\node [seg] (c3) at (3,0) {k};
		\node [seg] (v2) at (4,0) {a};
		%
		\node (m1) at (1,1) {μ};	
		\node (m2) at (2,1) {μ};	
		\node (m3) at (4,1) {μ};	
		%
		\node (s1) at (1,2) {σ};	
		\node (s2) at (4,2) {σ};	
		%
		\draw (c1.north) -- (s1.south);
		\draw (v1.north) -- (m1.south);
		\draw (v1.north) -- (m2.south);
		\draw [thick,dotted] (c2.north) -- (m2.south);
		% \draw (c3.north) -- (m2.south);
		\draw (c3.north) -- (s2.south);
		\draw (v2.north) -- (m3.south);
		\draw (m1.north) -- (s1.south);
		\draw (m2.north) -- (s1.south);
		\draw (m3.north) -- (s2.south);
		%
		%\node [seg] at (1.5,-0.5) {\small kappi \ipa{kʰahpi} \textit{`hero'}};
	\end{tikzpicture}
	}
\hspace{3em}
\subfigure[\orthog{läcka} \gloss{to leak}]{
\begin{tikzpicture}
		\node [seg] (c1) at (0,0) {l};
		\node [seg] (v1) at (1,0) {ɛ};	
		\node [seg] (c2) at (2,0) {(h)};
		\node [seg] (c3) at (3,0) {kː};
		\node [seg] (v2) at (4,0) {a};
		%
		\node (m1) at (1,1) {μ};	
		\node (m2) at (2,1) {μ};	
		\node (m3) at (4,1) {μ};	
		%
		\node (s1) at (1,2) {σ};	
		\node (s2) at (4,2) {σ};	
		%
		\draw (c1.north) -- (s1.south);
		\draw (v1.north) -- (m1.south);
		% \draw (v1.north) -- (m2.south);
		\draw [thick,dotted] (c2.north) -- (m2.south);
		\draw (c3.north) -- (m2.south);
		\draw (c3.north) -- (s2.south);
		\draw (v2.north) -- (m3.south);
		\draw (m1.north) -- (s1.south);
		\draw (m2.north) -- (s1.south);
		\draw (m3.north) -- (s2.south);
		%
		%\node [seg] at (1.5,-0.5) {\small kappi \ipa{kʰahpi} \textit{`hero'}};
	\end{tikzpicture}
	}
\subfigure[\orthog{tak} \gloss{roof}]{
\begin{tikzpicture}
		\node [seg] (c1) at (0,0) {t};
		\node [seg] (v1) at (1,0) {aː};	
		\node [seg] (c2) at (2,0) {(h)};
		\node [seg] (c3) at (3,0) {k};
		%
		\node (m1) at (1,1) {μ};	
		\node (m2) at (2,1) {μ};	
		%
		\node (s1) at (1,2) {σ};	
		%
		\draw (c1.north) -- (s1.south);
		\draw (v1.north) -- (m1.south);
		\draw (v1.north) -- (m2.south);
		\draw [thick,dotted] (c2.north) -- (m2.south);
		% \draw (c3.north) -- (m2.south);
		\draw (m1.north) -- (s1.south);
		\draw (m2.north) -- (s1.south);
		\draw [out=90,in=-15] (c3.north) to (s1.south);
		%
		%\node [seg] at (1.5,-0.5) {\small kappi \ipa{kʰahpi} \textit{`hero'}};
	\end{tikzpicture}
	}
\hspace{3em}
\subfigure[\orthog{tack} \gloss{thanks}]{
\begin{tikzpicture}
		\node [seg] (c1) at (0,0) {t};
		\node [seg] (v1) at (1,0) {a};	
		\node [seg] (c2) at (2,0) {(h)};
		\node [seg] (c3) at (3,0) {kː};
		%
		\node (m1) at (1,1) {μ};	
		\node (m2) at (2,1) {μ};	
		%
		\node (s1) at (1,2) {σ};	
		%
		\draw (c1.north) -- (s1.south);
		\draw (v1.north) -- (m1.south);
		% \draw (v1.north) -- (m2.south);
		\draw [thick,dotted] (c2.north) -- (m2.south);
		\draw (c3.north) -- (m2.south);
		\draw (m1.north) -- (s1.south);
		\draw (m2.north) -- (s1.south);
		%
		%\node [seg] at (1.5,-0.5) {\small kappi \ipa{kʰahpi} \textit{`hero'}};
	\end{tikzpicture}
	}
\end{figure}


Different quantity patterns emerge for some other Swedish dialects, which help flesh out the quantity typology for this language. Two dialects in particular have been subjected to closer investigation because of their notably long preaspiration: Vemdalen (in the central--west region of Härjedalen) and Arjeplog (in an eponymous northern region). In Arjeplog Swedish, there is a contrast between long preaspiration and short preaspiration, while in Vemdalen there is a contrast between long preaspiration and none at all \citep{wretlingetal2002,wretlingetal2003}. The correspondences between forms in these dialects compared to Central Standard Swedish are given in \tabref{tab:ex-swedish-dialectal-forms}. Transcriptions are based on the data in \citep{wretlingetal2002,wretlingetal2003}.

\begin{table} 
\caption{Preaspiration in Central Standard Swedish, Arjeplog, and Vemdalen}
\label{tab:ex-swedish-dialectal-forms}
	\begin{tabular}{l@{~~}O l l l G}
	\lsptoprule
	&  & CSS & Arjeplog & Vemdalen \\\midrule
	a. & tak	& [taːhk] & [tahːk]& [taːk]	& roof	\\
	   & tack	& [tahkː] & [tahkː]& [tahk] & thanks \\ 
	b. & låt	& [loːht] & [lohːt]& [loːt]	& song	 \\
	   & lott	& [lɔhtː] & [lɔhtː]& [lɔtː]	& fate \\
	c. & dit	& [diːht] & [dihːt]& [diːt]	& there	\\
	   & ditt	& [dɪhtː] & [dɪhtː]& [dɪht] & yours \\
   \lspbottomrule
	\end{tabular}
\end{table}

Measurements show that vowel duration does not vary significantly in Arjeplog between forms that are [VːC] vs.\ [VCː] in Central Standard Swedish, but preaspiration does vary in duration significantly. This suggests that preaspiration following earlier long vowels has expanded to take up the whole second mora in Arjeplog. The opposite has taken place in Vemdalen, where preaspiration is no longer found after long vowels, but is significantly longer after short vowels. This change away from mora sharing proceeded in different directions in different words, being lost in \orthog{lott} in favour of a moraic coda \citep{wretlingetal2002}. These durational differences from Central Standard Swedish align well with a moraic account, which is given in \figref{ex-swedish-arjeplog-moraic} for Arjeplog and \figref{ex-swedish-vemdalen-moraic} for Vemdalen.

\begin{figure}
\caption{Moraic structures for Arjeplog preaspiration}
\label{ex-swedish-arjeplog-moraic}
\subfigure[\orthog{tak} \gloss{roof}]{
\begin{tikzpicture}
		\node [seg] (c1) at (0,0) {t};
		\node [seg] (v1) at (1,0) {a};	
		\node [seg] (c2) at (2,0) {h};
		\node [seg] (c3) at (3,0) {k};
		%
		\node (m1) at (1,1) {μ};	
		\node (m2) at (2,1) {μ};	
		%
		\node (s1) at (1,2) {σ};	
		%
		\draw (c1.north) -- (s1.south);
		\draw (v1.north) -- (m1.south);
		% \draw (v1.north) -- (m2.south);
		\draw (c2.north) -- (m2.south);
		% \draw (c3.north) -- (m2.south);
		\draw (m1.north) -- (s1.south);
		\draw (m2.north) -- (s1.south);
		\draw [out=90,in=-15] (c3.north) to (s1.south);
		%
		%\node [seg] at (1.5,-0.5) {\small kappi \ipa{kʰahpi} \textit{`hero'}};
	\end{tikzpicture}
}
\hspace{3em}
\subfigure[\orthog{tack} \gloss{thanks}]{
\begin{tikzpicture}
		\node [seg] (c1) at (0,0) {t};
		\node [seg] (v1) at (1,0) {a};	
		\node [seg] (c2) at (2,0) {h};
		\node [seg] (c3) at (3,0) {kː};
		%
		\node (m1) at (1,1) {μ};	
		\node (m2) at (2,1) {μ};	
		%
		\node (s1) at (1,2) {σ};	
		%
		\draw (c1.north) -- (s1.south);
		\draw (v1.north) -- (m1.south);
		% \draw (v1.north) -- (m2.south);
		\draw (c2.north) -- (m2.south);
		\draw (c3.north) -- (m2.south);
		\draw (m1.north) -- (s1.south);
		\draw (m2.north) -- (s1.south);
		%
		%\node [seg] at (1.5,-0.5) {\small kappi \ipa{kʰahpi} \textit{`hero'}};
	\end{tikzpicture}
}
\end{figure}

\begin{figure}
\caption{Moraic structures for Vemdalen preaspiration}
\label{ex-swedish-vemdalen-moraic}
\subfigure[\orthog{tak} \gloss{roof}]{
\begin{tikzpicture}
		\node [seg] (c1) at (0,0) {t};
		\node [seg] (v1) at (1.5,0) {aː};	
		% \node [seg] (c2) at (2,0) {h};
		\node [seg] (c3) at (3,0) {k};
		%
		\node (m1) at (1,1) {μ};	
		\node (m2) at (2,1) {μ};	
		%
		\node (s1) at (1,2) {σ};	
		%
		\draw (c1.north) -- (s1.south);
		\draw (v1.north) -- (m1.south);
		\draw (v1.north) -- (m2.south);
		% \draw [thick,dotted] (c2.north) -- (m2.south);
		% \draw (c3.north) -- (m2.south);
		\draw (m1.north) -- (s1.south);
		\draw (m2.north) -- (s1.south);
		\draw [out=90,in=-15] (c3.north) to (s1.south);
		%
		%\node [seg] at (1.5,-0.5) {\small kappi \ipa{kʰahpi} \textit{`hero'}};
	\end{tikzpicture}
}
\hspace{2em}
\subfigure[\orthog{tack} \gloss{thanks}]{
\begin{tikzpicture}
		\node [seg] (c1) at (0,0) {t};
		\node [seg] (v1) at (1,0) {a};	
		\node [seg] (c2) at (2,0) {h};
		\node [seg] (c3) at (3,0) {k};
		%
		\node (m1) at (1,1) {μ};	
		\node (m2) at (2,1) {μ};	
		%
		\node (s1) at (1,2) {σ};	
		%
		\draw (c1.north) -- (s1.south);
		\draw (v1.north) -- (m1.south);
		% \draw (v1.north) -- (m2.south);
		\draw (c2.north) -- (m2.south);
		% \draw (c3.north) -- (m2.south);
		\draw (m1.north) -- (s1.south);
		\draw (m2.north) -- (s1.south);
		\draw [out=90,in=-15] (c3.north) to (s1.south);
	\end{tikzpicture}
}
\hspace{2em}
\subfigure[\orthog{lott} \gloss{fate}]{
\begin{tikzpicture}
		\node [seg] (c1) at (0,0) {l};
		\node [seg] (v1) at (1,0) {ɔ};	
		\node [seg] (c2) at (2,0) {tː};
		% \node [seg] (c3) at (3,0) {k};
		%
		\node (m1) at (1,1) {μ};	
		\node (m2) at (2,1) {μ};	
		%
		\node (s1) at (1,2) {σ};	
		%
		\draw (c1.north) -- (s1.south);
		\draw (v1.north) -- (m1.south);
		% \draw (v1.north) -- (m2.south);
		\draw (c2.north) -- (m2.south);
		% \draw (c3.north) -- (m2.south);
		\draw (m1.north) -- (s1.south);
		\draw (m2.north) -- (s1.south);
		% \draw [out=90,in=-15] (c3.north) to (s1.south);
	\end{tikzpicture}
}
\end{figure}

% Gräsö preaspiration is categorical \citep{helgason1999-fonetik}

The pattern seen in Vemdalen also occurs elsewhere in Swedish. On Kökar Island in the southeast of the Åboland archipelago, preaspiration did not develop at all after long vowels (\citealp{karsten1892}, as cited by \citealp{Helgason2002}).

Examination of the durational patterns in Swedish dialects has shown that the different patterns observed for different language varieties all fit into the typology of moraic association to preaspiration in different ways. We can interpret these durational alternations as reflecting an underlying difference in moraic structure, and conclude that preaspiration is weight-bearing in Swedish dialects as well. In Norwegian, preaspiration is just as widespread as in Swedish \citep{Helgason2002,iosad2018}, but detailed phonetic investigation of preaspiration is still needed. Given its close relation to Swedish and its similar quantity system, it would be surprising to find dramatic divergence in this language.

% The durational measurements of Central Swedish quantity presented by \citet{helgasonetal2013} appear to be compatible with this view, but they do not report preaspiration duration as a separate measure, so this is difficult to confirm. Their analytical choice to combine preaspiration duration with vowel duration is compelling, however: clearly, preaspiration is distinct fom the following stop in its prosodic patterning. 


\subsection{Scottish Gaelic}
\label{sec-gaelic}


Scottish Gaelic (\glottolog{scot1245}; often simply Gaelic; autonym \orthog{Gàidhlig} [ˈkaːlɪkʲ]) is a Goidelic Celtic language spoken in northern Scotland. After Icelandic, it is the next most commonly cited example of preaspiration, and it is mentioned prominently in the typological overviews of \citet{Helgason2002}, \citet{Silverman2003}, and \citet{Clayton:2010}. However, as in Icelandic, preaspiration in southern dialects of Scottish Gaelic has been convincingly argued to be a moraic coda \citep{iosadetal2015}. After outlining the evidence for this, I will show that preaspiration in northern dialects also shows a prosodic connection to the preceding syllable.


Stops in Scottish Gaelic contrast in postaspiration word-initially, and are usually considered to contrast in preaspiration elsewhere in the word \citep{ladefogedetal1998,nancestuartsmith2013,nanceomaolalaigh2021}. This echoes the pattern seen in Icelandic, and leads to the general assumption that there is complementary distribution between postaspiration word-initially and preaspiration word-medially and word-finally. Voicing plays no role in the stop inventory, and is largely absent phonetically \citep{nancestuartsmith2013}. 
Examples of the initial aspiration contrasts and reported medial preaspiration contrasts are shown in (\ref{ex-gaelic-aspiration-contrasts}), organized by place of articulation.%
\footnote{As noted earlier in Footnote \ref{fn-icelandic-fortis-lenis} for Icelandic, because of preaspiration there is effectively no laryngeal contrast in Gaelic stops outside of initial position.}
This data is from the Lewis dialect, which is commonly presented as an illustration of the language more generally.

\ea Gaelic stop contrasts \citep[263, 267]{nanceomaolalaigh2021}\label{ex-gaelic-aspiration-contrasts}
\NumTabs{5}
\ea 
\ea \orthog{bò}	 \tab\relax [{p}oː]	\tab  \gloss{cow}\\ 
	 \orthog{poll}	 \tab\relax [{pʰ}ɔul̪ˠ]	\tab  \gloss{mud} \\
     \ex \orthog{obair}	 \tab\relax [o{p}ɪɾʲ]\tab  \gloss{work} \\ 
     \orthog{capall} \tab\relax [kʰa{hp}əl̪ˠ]	\tab  \gloss{mare} 
     \z

\ex 
\ea \orthog{duine}	    \tab\relax [{t̪}in̪ʲə]	\tab  \gloss{anyone}  \\
	 \orthog{tuigsinn}	\tab\relax [{t̪ʰ}ikʃin̪ʲ]	\tab  \gloss{understanding} \\
     \ex \orthog{fada}	    \tab\relax [fa{t̪}ə]	\tab  \gloss{long}	   \\
     \orthog{bata}	    \tab\relax [pa{ht̪}ə]	\tab  \gloss{stick} 	\\
     \ex \orthog{bad}       \tab\relax [pa{t̪}] \tab  \gloss{bunch}    \\
     \orthog{cat}       \tab\relax [kʰa{ht̪}] \tab  \gloss{cat} 
     \z

\ex 
\ea \orthog{deoch}	    \tab\relax [{tʃ}ɔx]	\tab  \gloss{drink} \\
	 \orthog{teòclaid}	\tab\relax [{tʃʰ}ɔːhkl̪ˠɪtʃ]\tab  \gloss{chocolate}  
     \z

\ex 
\ea \orthog{geòla}	\tab\relax [{c}ɔːl̪ˠə]	\tab  \gloss{yawl}     \\
	 \orthog{ceòl}	\tab\relax [{cʰ}ɔːl̪ˠ]	\tab  \gloss{music}  \\
   \ex  \orthog{aige}	\tab\relax [ɛ{c}ə]	\tab  \gloss{at him}      \\
     \orthog{aice}	\tab\relax [e{hc}ə]	\tab  \gloss{at her}
     \z
                    
\ex \ea \orthog{gaol}	\tab\relax [{k}ɯːl̪ˠ]	\tab  \gloss{love}\\
	 \orthog{caol}	\tab\relax [{kʰ}ɯːl̪ˠ]	\tab  \gloss{thin}  \\
   \ex  \orthog{agad}	\tab\relax [a{k}ət̪]	\tab  \gloss{at you}  \\
     \orthog{aca}	\tab\relax [a{hk}ə]	\tab  \gloss{at them}  \\
   \ex  \orthog{bog}	\tab\relax [po{k}]	 	\tab  \gloss{soft}	\\
     \orthog{boc}	\tab\relax [pɔ{hk}]	\tab  \gloss{male goat}
     \z
\z
\z


In the Lewis dialect region, preaspiration is reported before stops at all places of articulation, and is described phonetically as glottal frication. However, there is a great deal of variation across Scottish Gaelic dialects, both broadly and in the domain of preaspiration specifically. As a result of phonological differences between dialects, the moraic status of preaspiration is easier to demonstrate in more southern dialects than in the Lewis dialect. In the more southern dialects, preaspiration is commonly produced with oral fortition as [x], and is sometimes absent before certain places of articulation. These dialect patterns are summarized in \tabref{tab:ex-gaelic-isoglosses}.

\begin{table}
\caption{Gaelic preaspiration across dialect regions \citep{sgds}. Abbreviations refer to cardinal directions}
\label{tab:ex-gaelic-isoglosses}
\begin{tabular}{l @{\hspace{1em}[\quad} lll @{\quad]\quad} l}
\lsptoprule
Region 1 &	xp &	xt &	xk & 	SE Inverness, NW Perth, N Argyll	\\
Region 2 &	 p &	xt &	xk &	small area around Aberfoyle 		\\
Region 3 &	hp &	ht &	xk &	Hebrides except Lewis, W Inverness    		\\
Region 4 &	hp &	ht &	hk &	Lewis, most of Ross-shire 		\\
Region 5 &	 p &	 t &	xk &	South Argyll 		\\
Region 6 &	 p &	 t &	 k &	Kintyre, Isle of Arran		\\
\lspbottomrule
\end{tabular}
\end{table}

Overall, preaspiration is produced with glottal frication more often in northern dialects (R3, R4), while southern dialects have only a dorsal fricative articulation (R1, R2, R5). Preaspiration is present before labials in the fewest dialect regions, and is present before velars in all preaspirating dialects; dorsal fricative realizations of preaspiration also appear to be least common before labials and most common before velars (\citealp{Clayton:2010,iosad2020}, \textit{inter alia}).

In the North Argyll dialect (Region 1), preaspiration is exclusively realized with dorsal frication ([x] or [χ]), rather than the glottal frication of Lewis (Region 4); furthermore, preaspiration is often absent after a long vowel \citep{iosadetal2015}. This pattern is illustrated in (\ref{ex-north-argyll-preaspiration}).

\ea North Argyll Gaelic preaspiration \citep{sgds,iosadetal2015}\label{ex-north-argyll-preaspiration}\\
\NumTabs{6}
\ea
 \orthog{cupan} \tab\relax [kʰu{xp}an]  \tab \gloss{cup}\\ 	
 \orthog{pàpa}	\tab\relax [pʰaː{p}ə]	\tab \gloss{Pope} \\
\ex \orthog{putan}	\tab\relax  [pʰu{xt}an] \tab \gloss{button}\\	
 \orthog{bàta}	\tab\relax  [paː{t}ə]	\tab \gloss{boat} \\
\ex \orthog{poca}	\tab\relax [pʰo{xk}ə]   \tab \gloss{pocket}	
\z
\z

Based on an acoustic study of North Argyll Gaelic, \citet{iosadetal2015} argue that preaspiration in this dialect region is best analyzed as a distinct moraic segment. Operating under the usual assumption that preaspiration is an instantiation of the [spread glottis] feature of the following stop, \citet{iosadetal2015} claim that the aspiration contrast is neutralized in the position after a long vowel, with preaspiration being inhibited when it is not prosodically within the head foot (e.g., \orthog{bàta} would be footed [(paː)tə], leaving no room for preaspiration within the foot). Under their analysis, feet are bimoraic, and a syllable with a long vowel takes up both moras and the entire foot. Since the foot licenses the [spread glottis] feature, preaspiration can only be realized after a short vowel, when the second mora is available to host the preaspiration. 

This analysis crucially requires preaspiration to be moraic, and prosodically linked to the syllable preceding the stop; thus, it supports my claims about the inherently moraic nature of preaspiration more generally. The most significant point of difference between their analysis of North Argyll Gaelic and my own is in the relevance of foot structure. In my view, the crucial element in the realization of preaspiration is its linkage to moraic weight rather than to foot structure. Furthermore, apart from its distribution preceding fortis stops, there is no clear evidence of a synchronic phonological link between [h] and a following stop.  
My proposed moraic structures, shown in \figref{ex-north-argyll-prosodic-structures}, are largely the same as those of \citet{iosadetal2015}, apart from their introduction of the foot level and the indication of preaspiration as sharing [spread glottis] with the stop.

\begin{figure} 
\caption{North Argyll prosodic structures (adapted from \citealp{iosadetal2015})}
\label{ex-north-argyll-prosodic-structures}
\subfigure[\orthog{bàta} {[paːtə]}  `{boat}']{
\begin{tikzpicture}[moraic-structure]
	\node (c1) at (1,0) {p};	
	\node (v1) at (2,0) {a};	
	\node (c3) at (3,0) {t};	
	\node (v2) at (4,0) {ə};	
	%
	\node (m1) at (2,1) {μ};
	\node (m2) at (2.75,1) {μ};
	\node (m3) at (4,1) {μ};
	%
	\node (s1) at (2,2) {σ};
	\node (s2) at (4,2) {σ};
	%
	\draw (c1.north) -- (s1.south);
	\draw (v1.north) -- (m1.south);
	\draw (v1.north) -- (m2.south);
	%
	\draw (c3.north) -- (s2.south);
	\draw (v2.north) -- (m3.south);
	%
	\draw (m1.north) -- (s1.south);
	\draw (m2.north) -- (s1.south);
	\draw (m3.north) -- (s2.south);
\end{tikzpicture}
\hspace*{.3cm}
}
\hspace{3em}
\subfigure[\orthog{cupan} {[kʰuxpan]} `{cup}']{
\begin{tikzpicture}[moraic-structure]
	\node (c1) at (1,0) {kʰ};	
	\node (v1) at (2,0) {u};	
	\node (c2) at (3,0) {x};	
	\node (c3) at (4,0) {p};	
	\node (v2) at (5,0) {a};	
	\node (c4) at (6,0) {n};	
	%
	\node (m1) at (2,1) {μ};
	\node (m2) at (3,1) {μ};
	\node (m3) at (5,1) {μ};
	%
	\node (s1) at (2,2) {σ};
	\node (s2) at (5,2) {σ};
	%
	\draw (c1.north) -- (s1.south);
	\draw (v1.north) -- (m1.south);
	\draw (c2.north) -- (m2.south);
	%
	\draw (c3.north) -- (s2.south);
	\draw (v2.north) -- (m3.south);
	\draw (c4.north) -- (m3.south);
	%
	\draw (m1.north) -- (s1.south);
	\draw (m2.north) -- (s1.south);
	\draw (m3.north) -- (s2.south);
\end{tikzpicture}
}
\end{figure}

\begin{sloppypar}
This analysis of preaspiration extends to other southern dialects of Scottish Gaelic, where there is additional data supporting its moraic affiliation. In the Gaelic dialects of Argyllshire, including those of the islands of Jura, Islay, Gigha, and Colonsay, there is a process of glottal stop insertion that is complementary with preaspiration \citep{sgds,smith1999,jones2006,jones2009,scouller2017,morrison2019,iosad2021}. In these dialects, it appears that non-final stressed syllables must be heavy, with a glottal stop inserted in the coda to ensure this \citep{iosadetal2015}. The alternation involving the glottal stop is illustrated in (\ref{ex-gaelic-stress-to-weight}), where stress placement conditions the presence or absence of [ʔ]. In (\ref{ex-gaelic-stress-to-weight}a), monosyllabic \orthog{sruth} /sru/ requires glottal epenthesis to form a heavy syllable. In the compounded form in (\ref{ex-gaelic-stress-to-weight}b), stress is on the initial syllable of the second root, the initial syllable remains light, and glottal epenthesis does not occur.
\end{sloppypar}

\ea South Argyll Gaelic \citep{jones2006}\label{ex-gaelic-stress-to-weight}
	\NumTabs{5}
	\ea  sruth 		   \tab\relax [ˈsruʔ]		\tab \gloss{stream}
	\ex  sruth-lìonadh \tab\relax [sru.ˈtʃiː.nəɣ]	\tab \gloss{flood} (\textit{lit.} \gloss{stream-filling}) 
	\z
\z

This epenthetic glottal stop appears only following a short vowel in stressed syllables, but its insertion is blocked by the presence of preaspiration \citep{jones2006}.%
\footnote{I appreciate a reviewer pointing out that some caution is necessary here: subsequent work by \citet{morrison2019} and \citet{iosad2021} observes that glottal coda epenthesis is not blocked only in cases where there is a medial preaspirated stop, but also in words with a medial voiceless fricative: we get \orthog{drochaid} [trɔxətʲ] \gloss{bridge}, \orthog{seasamh} [ʃesəv] \gloss{standing}, and \orthog{toiseachd} [tʰoʃəx], where we otherwise would have expected either *[trɔʔxətʲ] or *[trɔhxətʲ] \citep{morrison2019}. This is interpreted by \citet{morrison2019} and \citet{iosad2021} to be a result of the [+spread glottis] character of the voiceless fricative, which blocks the glottalization. This requires some additional research and ideally phonetic investigation in order to confirm that there is no preaspiration here after all.}
As in North Argyll Gaelic, preaspiration is only found after short vowels, meaning that in South Argyll Gaelic, syllables can have either a glottal coda or a long vowel, but not a combination of these. Examples of this distribution in Colonsay Gaelic, a variety of South Argyll Gaelic, are shown in (\ref{ex-gaelic-colonsay-glottals}).%; data primarily from the SGDS \citep{sgds} with interpretation and discussion by \citet{scouller2017}.%


\ea Colonsay heavy stressed syllables \citep{sgds,scouller2017}\label{ex-gaelic-colonsay-glottals}
\NumTabs{4}
	\ea	 \orthog{crùban} \tab\relax [ˈkruːpan̥]   \tab \gloss{brown crab}\\
		 \orthog{tobar}  \tab\relax [ˈtʰoʔpər̥]   \tab \gloss{well}\\
		 \orthog{cupan}  \tab\relax [ˈkohpan̥]    \tab \gloss{cup}
	\ex	 \orthog{bàta} 	 \tab\relax [ˈpaːtə]	 \tab  \gloss{boat}\\
		 \orthog{bradan} \tab\relax [ˈpraʔtan̥]   \tab \gloss{salmon}\\
		 \orthog{Cille Chatan} \tab\relax [kʲʰɪʎɪ ˈxahtan̥] \tab \gloss{Kilchattan}
	\ex	 \orthog{dh'fhaoidte}  \tab\relax [ˈɣøːtʃɪ]	 \tab  \gloss{perhaps}\\   
		 \orthog{oide}         \tab\relax [ˈøʔtʃɪ]	 \tab  \gloss{tutor}\\
		 \orthog{ite}          \tab\relax [ˈihtʃɪ] \tab \gloss{feather} 
	\ex	 \orthog{fàgail} 	\tab\relax [ˈfaːkal̥ʲ]	 \tab  \gloss{leaving}\\
		 \orthog{lagan}     \tab\relax [ˈlˠaʔkan̥] 	 \tab  \gloss{little hollow, dell}\\          
		 \orthog{paca}      \tab\relax [ˈpʰaxkə] \tab \gloss{pack} 
	\z
\z

In these examples, each set of examples shows the same medial stop. The first word in each set has a preceding long vowel, the second a preceding [ʔ], and the third a preceding [h]. Historically, the forms with [hC] had a fortis stop, which is still reflected in the orthography. However, this contrast has been neutralized, which has made an analysis of preaspiration as a distinct coda [h] very plausible in this dialect. The complementary relationship between the long vowels and the laryngeal codas strongly supports a moraic analysis of preaspiration.

One caveat to the pattern in (\ref{ex-gaelic-colonsay-glottals}) is that short vowels also appear to be attested in stressed initial syllables in Colonsay Gaelic, which casts some doubt on the account of glottal stop insertion as a synchronic phonological process. Examples of this are numerous, including \orthog{gainmheach} [kɛɲax] \gloss{sandy}, \orthog{soirbheas} [sɔras] \gloss{prosperity}, and \orthog{coileach} [kʰʏlʲəx] \gloss{rooster}.%
\footnote{These exceptions still exist even after setting aside words where the presence of a short vowel in an initial syllable might result from a different phonological process. For instance, in words with ``hiatus'', a medial glottal stop separates a short vowel and a phonologically predictable ``echo'' vowel (e.g., \orthog{cridhe} [kriʔi] \gloss{heart}, \orthog{gobha} [koʔo] \gloss{blacksmith}). Similarly, words with \textit{svarabhakti} vowel epenthesis (e.g., \orthog{dealg} [tʃalˠək] \gloss{pin}, \orthog{garbh} [karɪv̥] \gloss{rough}) appear to be disyllabic with a short first vowel, but are often analyzed as underlyingly monosyllabic \citep{scouller2017}.}
\citet{scouller2017} notes some of these exceptions, and explains that glottal stops are not found in words where epenthesis took place historically, at a great enough time depth that the epenthesis is reflected in the (fairly deep) orthography. This indicates that glottal stop epenthesis is likely not a synchronically productive process, and that it may have taken place historically in the Argyllshire varieties of Gaelic. However, this does not change the moraic analysis of the modern forms. 
Crucially, the glottal codas [h,~ʔ] do not co-occur with each other or with a preceding long vowel in Colonsay Gaelic, which supports the claim that preaspiration behaves as a moraic coda.
In fact, the historical nature of glottalization aligns with the historical nature of preaspiration. Although it is undeniable that the origin of preaspiration is related to the original laryngeal category of the following stop, there is no clear evidence that it is synchronically derived from or dependent on the stop, or clear evidence that a laryngeal contrast still exists on medial stops themselves.%

The fact that this analysis extends across both North and South Argyll Gaelic also demonstrates that it is not limited to a particular type of preaspiration pattern. The North Argyll Gaelic studied by \citet{iosadetal2015} is a [xp, xt, xk] variety (Region 1), while the South Argyll dialect of Colonsay Island is a [hp, ht, xk] variety (Region 3). This shows that the moraic status of preaspiration is not tied to its realization as [x], or to oral fortition of the fricative more generally.%

Turning to Lewis Gaelic, and the [hp, ht, hk] varieties of Gaelic more generally, the phonological status of preaspiration appears to be slightly more difficult to establish. Preaspiration does not show the same categorical alternation with long vowels discussed in Argyllshire varieties, and glottalization is not a feature of this dialect, so the moraic structures presented earlier may not be applicable here. However, as this is one of the most studied dialects of Scottish Gaelic, there is substantially more phonetic research on preaspiration \citep{nichasaideodochartaigh1984,nichasaide1985,ladefogedetal1998,Clayton:2010,nancestuartsmith2013,nanceomaolalaigh2021}. The trends seen in aspiration duration display a prosodic relationship between preaspiration and the preceding vowel.

Across a variety of studies spanning several decades, preaspiration in Lewis Gaelic has been shown to be significantly shorter after long vowels than after short vowels \citep{nichasaide1985,Clayton:2010,nancestuartsmith2013}. This trend is explicitly discussed as evidence of a prosodic complementarity between the vowel and the preaspiration within the syllable. \citet{nichasaide1985} suggests that vowel shortening before preaspiration could be a compensatory durational effect, serving to mitigate duration differences in syllables with and without preaspiration. \citet{nancestuartsmith2013} say it is possible that preaspiration is phonologically part of the vowel length system: rather than being cued only by the duration of the vowel itself, phonemic vowel length might be cued by the length of the following preaspiration as well. These theoretical proposals approach the same fundamental insight I am pursuing from slightly different angles. The interdependence of vowel and preaspiration duration supports a prosodic model in which preaspiration is affiliated with the same syllable as the vowel, with the moraic weight of that syllable distributed across both elements. Rather than a categorical ``all or nothing'' preaspiration pattern dependent on vowel length, of the type in \figref{fig:ex-preaspiration-moraic-structures}a (seen in North Argyll Gaelic or Icelandic), a mora-sharing structure as in \figref{fig:ex-preaspiration-moraic-structures}c can capture a pattern in which vowel duration and preaspiration duration are complementary, as illustrated in \figref{ex-lewis-prosodic-structures}.

\begin{figure} 
\caption{Proposed moraic structures for Lewis Gaelic preaspiration}
\label{ex-lewis-prosodic-structures}
\subfigure[\orthog{bàta} {[paːhtə]} `boat']{
\begin{tikzpicture}[moraic-structure]
	\node (c1) at (1,0) {p};	
	\node (v1) at (2,0) {a};	
	\node (c2) at (3,0) {h};	
	\node (c3) at (4,0) {t};	
	\node (v2) at (5,0) {ə};	
	%
	\node (m1) at (2,1) {μ};
	\node (m2) at (3,1) {μ};
	\node (m3) at (5,1) {μ};
	%
	\node (s1) at (2,2) {σ};
	\node (s2) at (5,2) {σ};
	%
	\draw (c1.north) -- (s1.south);
	\draw (v1.north) -- (m1.south);
	\draw (v1.north) -- (m2.south);
	\draw (c2.north) -- (m2.south);
	%
	\draw (c3.north) -- (s2.south);
	\draw (v2.north) -- (m3.south);
	%
	\draw (m1.north) -- (s1.south);
	\draw (m2.north) -- (s1.south);
	\draw (m3.north) -- (s2.south);
\end{tikzpicture}
}
\hspace{3em}
\subfigure[\orthog{cupan} {[kʰuhpan]} `cup']{
\begin{tikzpicture}[moraic-structure]
	\node (c1) at (1,0) {kʰ};	
	\node (v1) at (2,0) {u};	
	\node (c2) at (3,0) {h};	
	\node (c3) at (4,0) {p};	
	\node (v2) at (5,0) {a};	
	\node (c4) at (6,0) {n};	
	%
	\node (m1) at (2,1) {μ};
	\node (m2) at (3,1) {μ};
	\node (m3) at (5,1) {μ};
	%
	\node (s1) at (2,2) {σ};
	\node (s2) at (5,2) {σ};
	%
	\draw (c1.north) -- (s1.south);
	\draw (v1.north) -- (m1.south);
	\draw (c2.north) -- (m2.south);
	%
	\draw (c3.north) -- (s2.south);
	\draw (v2.north) -- (m3.south);
	\draw (c4.north) -- (m3.south);
	%
	\draw (m1.north) -- (s1.south);
	\draw (m2.north) -- (s1.south);
	\draw (m3.north) -- (s2.south);
\end{tikzpicture}
}
\end{figure}

These structures capture the duration difference in preaspiration after short and long vowels, as well as the overall shorter phonetic duration of preaspiration in Lewis Gaelic as compared to more southern Gaelic dialects \citep{nichasaide1985}. Although the moraic representations above are categorical, the phonetic realization of preaspiration after a long vowel suggests that a more gradient distribution of weight across [V{h}] may be possible, either phonologically or in the phonetic implementation. \citet{nichasaide1985} finds a strong negative correlation in Lewis Gaelic between the duration of preaspiration and the durational difference between long and short preceding vowels. As the preaspiration gets longer (takes up closer to one full mora), phonologically long and short vowels become more phonetically similar in duration. 

Looking across a variety of Scottish Gaelic dialects, we have seen that different preaspiration patterns all show a connection between preaspiration and moraic weight, and that the proposed moraic structures in \figref{fig:ex-preaspiration-moraic-structures} are able to capture the cross-dialectal variation in a satisfying way.

\subsection{North Sámi}
\label{sec-saami}

Preaspiration is commonly found in the Sámi languages, where it alternates within the morphological grade system. \citet{balsbaal2012} present a compelling moraic account of preaspiration in North Sámi (\glottolog{nort2671}; Uralic; Norway, Sweden, Finland) in their analysis of the language's phonology. Although the prosodic system is quite complicated, here I will simply show that preaspiration bears moraic weight in North Sámi, as the duration of preaspiration changes with the addition of a mora to the word.

The North Sámi quantity system is generally described as having three degrees, often referred to as Q1, Q2, and Q3 based on the length of the medial consonant, as illustrated in (\ref{ex-saami-ternary-length}). These consonants are also described as short (C), long (CC), and overlong (CːC). Within individual pairs that alternate in quantity, one form is usually referred to as the ``strong'' grade and the other the ``weak'' grade. The difference between these grades can be represented by the association of moraic weight within the word: grade alternations within a root are the result of the addition of a single mora in the strong grade. 
In some cases, the difference between these quantity degrees is expressed through segmental duration, as shown in (\ref{ex-saami-ternary-length}), but this quantity distinction can be expressed in other ways as well, including through preaspiration.

\ea\label{ex-saami-ternary-length}
\begin{tabular}[t]{llG}
	Q1 & [ruosa] & Sweden.\textsc{acc} \\
	Q2 & [ruossa] & cross.\textsc{acc} \\
	Q3 & [rŭosːsa] & cross.\textsc{nom} \\
\end{tabular}
\z

In the examples in \tabref{ex-saami-preasp-grade}, words in the nominative case are in the strong grade (marked by addition of a mora), and words in the accusative case are in the weak grade. There are, among others, visible alternations in both the presence of preaspiration and its phonetic length, resulting in a [$\varnothing$$\sim$h$\sim$hː] alternation from Q1 to Q3.

\begin{table}
\caption{Preaspiration quantity in strong and weak grade \citep{balsbaal2012}}
\label{ex-saami-preasp-grade}
\begin{tabular}{l@{~~}ll G}
\lsptoprule
& \textsc{nom} & \textsc{acc} \\\midrule
a.
&[neahpi] & [neabi] & nephew, niece \\
&[goahti] & [goaði] & big tent \\
&[tʃiehka] & [tʃiega] & corner \\
&[aːhtsi] & [aːdzi] & hay-rack \\
&[geahtʃi] & [geadʒi] & end \\
b.
&[ruhːtaː] & [ruðaː] & money \\
&[jahːki] & [jagi] & year \\
&[ohːtsu] & [odzu] & search \\
c. 
&[ŏahːpa] & [oahpa] & teaching \\
&[lahːti] & [laːhti] & floor \\
&[mĭehːki] & [miehki] & sword \\
&[bihːtsi] & [biːhtsi] & frost \\
&[gĕahːtʃu] & [geahtʃu] & surveillance \\
\lspbottomrule
	\end{tabular}
\end{table}

Under {\citeauthor{balsbaal2012}'s (\citeyear{balsbaal2012})} analysis, preaspiration requires a moraic association in order to surface phonetically. In Q1 forms, seen in the accusative column in \tabref{ex-saami-preasp-grade}a--b, this does not happen. In contrast, in the strong grade forms in the nominative column, the preaspiration receives either half a mora or a full mora of weight, surfacing as phonetically short or long. This is reflected in \citeauthor{balsbaal2012}'s moraic structures, which are shown in \figref{ex-saami-q-alternations-preasp}. In these structures, \fbox{μ} represents the affixal mora that marks the nominative singular case in their analysis.

\begin{figure} 
\caption{Alternations in preaspiration quantity \citep{balsbaal2012}}
\label{ex-saami-q-alternations-preasp}
\subfigure[Q1--Q2 alternation]{
\parbox{0.4\textwidth}{\centering%
\begin{tikzpicture}[moraic-structure]
	\node [seg] (c1) at (0,0) {k};
	\node [seg] (v1) at (1,0) {o};	
	\node [seg] (v2) at (2,0) {a};	
	\node [seg] (c2) at (3,0) {h};
	\node [seg] (c3) at (4,0) {t};
	\node [seg] (v3) at (5,0) {i};
	%
	\node (m1) at (1,1) {μ};	
	\node (m2) at (2,1) {μ};	
	\node (m3) at (5,1) {μ};	
	%
	\node (s1) at (1,2) {σ};	
	\node (s2) at (5,2) {σ};	
	%
	\draw (c1.north) -- (s1.south);
	\draw (v1.north) -- (m1.south);
	\draw (v2.north) -- (m2.south);
	\draw (c3.north) -- (s2.south);
	\draw (v3.north) -- (m3.south);
	\draw (m1.north) -- (s1.south);
	\draw (m2.north) -- (s1.south);
	\draw (m3.north) -- (s2.south);
	%
	\node at (2,-0.5) {\small [goaði] \gloss{big tent.\textsc{acc}}};
\end{tikzpicture}}
$\rightarrow$
\parbox{0.4\textwidth}{\centering%
\begin{tikzpicture}[moraic-structure]
	\node [seg] (c1) at (0,0) {k};
	\node [seg] (v1) at (1,0) {o};	
	\node [seg] (v2) at (2,0) {a};	
	\node [seg] (c2) at (3,0) {h};
	\node [seg] (c3) at (4,0) {t};
	\node [seg] (v3) at (5,0) {i};
	%
	\node (m1) at (1,1) {μ};	
	\node (m2) at (2,1) {\fbox{μ}};	
	\node (m3) at (5,1) {μ};	
	%
	\node (s1) at (1,2) {σ};	
	\node (s2) at (5,2) {σ};	
	%
	\draw (c1.north) -- (s1.south);
	\draw (v1.north) -- (m1.south);
	\draw (v2.north) -- (m2.south);
	\draw (c2.north) -- (m2.south);
	\draw (c3.north) -- (s2.south);
	\draw (v3.north) -- (m3.south);
	\draw (m1.north) -- (s1.south);
	\draw (m2.north) -- (s1.south);
	\draw (m3.north) -- (s2.south);
	%
	\node  at (2,-0.5) {\small [goahti] \gloss{big tent.\textsc{nom}}};
\end{tikzpicture}
}
}\medskip\\
\subfigure[Q1--Q3 alternation]{
\parbox{0.4\textwidth}{\centering%
\begin{tikzpicture}[moraic-structure]
	%\foreach \x in {0,1,...,6}{\node at (\x,2.5) {\tiny \x};}
	\node [seg] (c1) at (0,0) {j};
	\node [seg] (v1) at (1,0) {a};	
	\node [seg] (c2) at (2,0) {h};
	\node [seg] (c3) at (3,0) {k};
	\node [seg] (v3) at (4,0) {i};
	%
	\node (m1) at (1,1) {μ};	
	\node (m3) at (4,1) {μ};	
	%
	\node (s1) at (1,2) {σ};	
	\node (s2) at (4,2) {σ};	
	%
	\draw (c1.north) -- (s1.south);
	\draw (v1.north) -- (m1.south);
	\draw (c3.north) -- (s2.south);
	\draw (v3.north) -- (m3.south);
	\draw (m1.north) -- (s1.south);
	\draw (m3.north) -- (s2.south);
	%
	\node  at (2,-0.5) {\small [jagi] \gloss{year.\textsc{acc}}};
\end{tikzpicture}}
$\rightarrow$
\parbox{0.4\textwidth}{\centering%
\begin{tikzpicture}[moraic-structure]
	\node [seg] (c1) at (0,0) {j};
	\node [seg] (v1) at (1,0) {a};	
	\node [seg] (c2) at (2,0) {h};
	\node [seg] (c3) at (3,0) {k};
	\node [seg] (v3) at (4,0) {i};
	%
	\node (m1) at (1,1) {μ};	
	\node (m2) at (2,1) {\fbox{μ}};	
	\node (m3) at (4,1) {μ};	
	%
	\node (s1) at (1,2) {σ};	
	\node (s2) at (4,2) {σ};	
	%
	\draw (c1.north) -- (s1.south);
	\draw (v1.north) -- (m1.south);
	\draw (c2.north) -- (m2.south);
	\draw (c3.north) -- (s2.south);
	\draw (v3.north) -- (m3.south);
	\draw (m1.north) -- (s1.south);
	\draw (m2.north) -- (s1.south);
	\draw (m3.north) -- (s2.south);
	%
	\node  at (1.5,-0.5) {\small [jahːki] \gloss{year.\textsc{nom}}};
\end{tikzpicture}}
}\medskip\\
\subfigure[Q2--Q3 alternation]{
\parbox{0.4\textwidth}{\centering%
\begin{tikzpicture}[moraic-structure]
	\node [seg] (c1) at (0,0) {m};
	\node [seg] (v1) at (1,0) {i};	
	\node [seg] (v2) at (2,0) {e};	
	\node [seg] (c2) at (3,0) {h};
	\node [seg] (c3) at (4,0) {k};
	\node [seg] (v3) at (5,0) {i};
	%
	\node (m1) at (1,1) {μ};	
	\node (m2) at (2,1) {μ};	
	\node (m3) at (5,1) {μ};	
	%
	\node (s1) at (1,2) {σ};	
	\node (s2) at (5,2) {σ};	
	%
	\draw (c1.north) -- (s1.south);
	\draw (v1.north) -- (m1.south);
	\draw (v2.north) -- (m2.south);
	\draw (c2.north) -- (m2.south);
	\draw (c3.north) -- (s2.south);
	\draw (v3.north) -- (m3.south);
	\draw (m1.north) -- (s1.south);
	\draw (m2.north) -- (s1.south);
	\draw (m3.north) -- (s2.south);
	%
	\node  at (2,-0.5) {\small [miehki] \gloss{sword.\textsc{acc}}};
\end{tikzpicture}}
$\rightarrow$
\parbox{0.4\textwidth}{\centering%
\begin{tikzpicture}[moraic-structure]
	\node [seg] (c1) at (0,0) {m};
	\node [seg] (v1) at (1,0) {i};	
	\node [seg] (v2) at (2,0) {e};	
	\node [seg] (c2) at (3,0) {h};
	\node [seg] (c3) at (4,0) {k};
	\node [seg] (v3) at (5,0) {i};
	%
	\node (m1) at (1,1) {μ};	
	\node (m2) at (2,1) {\fbox{μ}};	
	\node (m3) at (5,1) {μ};	
	%
	\node (s1) at (1,2) {σ};	
	\node (s2) at (5,2) {σ};	
	%
	\draw (c1.north) -- (s1.south);
	\draw (v1.north) -- (m1.south);
	\draw (v2.north) -- (m1.south);
	\draw (c2.north) -- (m2.south);
	\draw (c3.north) -- (s2.south);
	\draw (v3.north) -- (m3.south);
	\draw (m1.north) -- (s1.south);
	\draw (m2.north) -- (s1.south);
	\draw (m3.north) -- (s2.south);
	%
	\node  at (2,-0.5) {\small [mĭehːki] \gloss{sword.\textsc{nom}}};
\end{tikzpicture}}
}
\end{figure}


As these patterns and structures show, preaspiration is intrinsically linked to moraic weight in North Sámi, with two possible surface lengths depending on the moraic association. This is reminiscent of the pattern seen earlier in Arjeplog Swedish (\S\ref{sec-swedish-norwegian}), which (coincidentally or not) is spoken in a Pite Sámi area.

As noted earlier, \citet{balsbaal2012} nonetheless still consider preaspiration to be a feature of the following stop. This unary analysis relies on a claim that preaspiration shows different grade alternations than those seen in underlying consonant clusters. Words with a sequence of a glide, liquid, or nasal before another medial consonant have length alternations in both consonants, unlike words with preaspiration, where only the [h] changes in length. However, this claim overlooks words with medial voiceless fricative codas. These do not follow the pattern seen in sonorant codas, but instead pattern like words with preaspiration. These patterns are shown very briefly in \tabref{ex-saami-cluster-grade}.


\begin{table}
\caption{Grade alternations in medial clusters in North Sámi \citep{balsbaal2012}}
\label{ex-saami-cluster-grade}
\begin{tabular}{l@{~~} ll l G}
	\lsptoprule
	& \textsc{nom} & \textsc{acc} \\\midrule
	a.
	&[lahːti] & [laːhti] && floor \\
	&[bihːtsi] & [biːhtsi] && frost \\
	%
	b.
	&[doajːvu] & [doajvːu] && belief \\
	&[gŭowːlu] & [gŭowlːu] && area \\
	&[gumːpe] & [gumpːe] && wolf \\
	c.
	&[basːte] & [baːste] & *[bastːe] & spoon \\
	&[lŭosːku] & [luosku] & *[luoskːu] & loose snow \\
	&[ʃuʃːmi] & [ʃuːʃmi] & *[ʃuʃmːi] & heel \\
	\lspbottomrule
	\end{tabular}
\end{table}

The words with a Q2--Q3 preaspiration pattern in \tabref{ex-saami-cluster-grade}a have long [hː] in the nominative and short [h] in the accusative, while the following consonant remains a non-moraic onset. This parallels the length changes seen with other coda fricatives in \tabref{ex-saami-cluster-grade}c. Crucially, fricative codas do not follow the same pattern as non-fricative codas, seen in \tabref{ex-saami-cluster-grade}b. If they did follow the non-fricative length pattern, we would expect the unattested accusative forms in the third column of \tabref{ex-saami-cluster-grade}c. This means that there is no basis to consider preaspiration phonologically distinct from other medial clusters in which the first consonant is a fricative.
Although a more detailed discussion of North Sámi is beyond the scope of this brief overview, the data shown here demonstrate that North Sámi preaspiration both has a moraic association and can reasonably be analyzed as a coda [h].


\subsection{Spanish}
\label{sec-spanish}


Spanish is well known for a process commonly referred to as \textit{s-aspiration}, which involves lenition or deletion of voiceless fricatives (most frequently /s/) in coda position. This lenition often results in [h] or phonetic breathiness, which has given the phenomenon its name \citep{torreira2012}. It is often summarized as a process in which /s/ has one of three allophones [s], [h], or $\varnothing$ \citep[often with compensatory lengthening of the preceding vowel; e.g.,][]{lipski1984,lipski1994,obrien2012,walkeretal2014} when it occurs in coda position. Thus, a word like \textit{pasta} /pasta/ might be phonetically realized as [pasta], [pahta], or [paːta], depending on the dialect and speaker. Although productions like [pahta] are not usually termed ``preaspiration'', the main reason for this comes down to analytical tradition rather than a principled structural argument. In the typological literature on preaspiration, it is stereotypically viewed as a property of a stop or the result of degemination, whereas here it clearly originates as a discrete consonant. However, it is phonologically indistinguishable from preaspiration in other languages from a synchronic perspective, and serves a prosodic role in maintaining the weight of the coda consonant.

The simplified description of s-aspiration as /s/ becoming [h] is the basis for theoretical analyses that depict it as a debuccalization process \citep[e.g.,][]{goldsmith1981,vaux1998,obrien2012}. The oral features or supralaryngeal articulation of the fricative are lost in coda position, but its laryngeal features remain. %This debuccalization is schematized in (\ref{ex-spanish-debuccalization}) below.
However, the facts of s-aspiration are much more complex. The phonetic outcome of the coda /s/ lenition is quite variable, both within and across dialects, and it is often only a part of a larger phonological pattern of coda lenition, which may involve other segments as well, complicating a theoretical analysis. Most importantly for the arguments made here, some of these phonetic outcomes involve gemination of a following consonant (i.e., [hC] $\rightarrow$ [Cː]).%, a process that resembles the inverse of the diachronic degemination process that results in preaspiration in other languages, and highlights the role of [h] as a mechanism for preserving prosodic weight.

Andalusian Spanish (\glottolog{anda1279}) has been an area of particular focus for research on coda lenition and {s-aspiration}, because of the diversity of phonetic realizations that have been reported. Lenition of syllable codas may result here in lengthening of the preceding vowel or the following consonant, pre- or postaspiration of that consonant, or even coalescence of a new post-consonantal fricative on the other side \citep[e.g.,][]{gerfen2002,torreira2007proc,torreira2007ch,torreira2012,herrerodeharohajek2021}. % As the Western and Eastern dialects of Andalusian Spanish differ slightly, these will be described separately below.
In Western Andalusian Spanish, the realization of an underlying coda /s/ depends primarily on the phonological environment. If there is no following consonant, /s/ is deleted or produced as [h], but if there is a consonant then gemination often occurs. The patterns are illustrated below, where (\ref{ex-andalusian-codas}a) shows /s/ utterance-finally, (\ref{ex-andalusian-codas}b) shows /s/ in word-final prevocalic position, and (\ref{ex-andalusian-codas}c) shows /s/ in pre-consonantal position.


\vbox{
	\ea Realization of Western Andalusian Spanish coda /s/ \citep[49, 50]{torreira2012}\\[3pt]
\label{ex-andalusian-codas}
\tablecounter
\begin{tabular}[t]{ll OlG}
\TC & $\varnothing$ & gatos	& [gato] & cats \\
\TC & [h] & los otros & [lohotro] & the others \\
\TC & gemination & los martes & [lomːarte] & on Tuesdays \\
	&& los lunes & [lolːune] & on Mondays \\
	&& las flores & [lafːloɾe] & the flowers \\
 &  & los gatos & [loɣːato] & the cats \\
\end{tabular}
\z
}


This gemination is essentially the reverse of the process that yielded preaspiration in Icelandic and other languages, and highlights the behaviour of [h] as a moraic coda, where it participates in a variety of diverse quantity-related prosodic changes. However, degemination is often connected to a debuccalization analysis, in which [h] is derived phonologically from the laryngeal features of a fortis stop. Conversely, the gemination process in Western Andalusian Spanish does not involve the features of [h] at all, and can result in sonorant geminates that would not share laryngeal features with [h]. This shows that preaspiration can fulfill a role as a length compensation mechanism in the prosodic phonology, irrespective of its featural content.

% , and the presence of [h] is structurally equivalent to other 

An anonymous reviewer notes that postaspiration being one of the outcomes of s-aspiration (e.g., /pasta/ $\rightarrow$ [pahtʰa]$\sim$[patʰa]; \citealp{parrell2012,ruchpeters2016}) may be unexpected for the moraic account described here. Although the appearance of postaspiration is not necessarily predicted by this moraic model, it is not problematic for the model either, so long as we do not actually see variation between [ht] and non-moraic onset [tʰ]. Phonetic data from \citet{torreira2007proc} shows that stop closure duration is longer on average following underlying /s/ in Andalusian Spanish, while preaspiration is quite short. In other words, postaspirated productions of words like /pasta/ might more accurately be [patːʰa] or [pahtːʰa]. The appearance of postaspiration in this context could be interpreted as spreading of the [spread glottis] feature to the stop, but this is tangential to the phonological account of preaspiration. Gemination of the following segment (as an alternative to preaspiration) supports the idea that either gemination or preaspiration can compensate for the loss of the moraic coda /s/ (cf.\ \S\ref{sec-discussion}). 

In Dominican Spanish (\glottolog{domi1242}), s-aspiration is part of a larger pattern of coda lenition, in which other fricatives and also voiced segments like /r/ are reduced to [h] in coda position. This is illustrated in (\ref{ex-r-debuccalization}).

\vbox{
\ea ``Debuccalization'' of [ɾ] in Dominican Spanish \citep[58]{nunezcedeno2014} \\[3pt]
\label{ex-r-debuccalization}
\begin{tabular}[t]{*{2}{>{\itshape}r @{} >{[}c<{]} @{} >{\itshape}l} G }
di&f&teria & di&h&teria		& diphtheria \\
a&f&gano & a&h&gano		& Afghan \\
ca&ɾ&ne & ca&h&ne		& meat \\
O&ɾ&lando & O&h&lando		& Orlando \\
\end{tabular}
\z
}

The fact that /r/ reduces to [h] in this context shows that this is not straightforwardly a lenition process in which voiceless coda fricatives debuccalize, as there is no [spread glottis] feature to be left behind by debuccalization. Rather, it appears that [h] serves as a general lenited consonantal coda. In this case, use of coda [h] in this way can be interpreted as a structure-preserving alternative to full deletion and compensatory vowel lengthening. Although this would not typically be called ``preaspiration'', the process is consistent with preaspiration patterns seen elsewhere, and thus the term seems equally applicable here. We are dealing with a period of glottal frication appearing before another consonant, which serves to maintain the prosodic structure of the word in the same way as in Icelandic degemination.


\subsection{Algonquian languages}
\label{sec-algonquian}
\largerpage[2]

The term ``preaspiration'' is widely used in research on Algonquian languages, in spite of a widespread recognition that what is being described is a heterosyllabic [hC] cluster (Will Oxford, p.c.)\footnote{I am deeply grateful to Will Oxford for sharing many observations and data about these patterns in Algonquian languages.}. For instance, the syllable templates given by \citet{wolfart1996} for Cree show that [h] and [s] are the only possible syllable codas. Most Algonquian languages have [{h}C] sequences, with the following consonant being most commonly a stop or affricate, but with following fricatives or sonorants attested in some languages as well. The attested ``preaspirated'' consonants are summarized in \tabref{tab:ex-algonquian-preaspiration} 
\citep[based on][]{bloomfield1925,bloomfield1946,davis1962,pentland1979,ellis1983,greensmith1985,hayes1995,wolfart1996,starksballard2003,schmirler2016,flynnetal2019}.
% \citep{bloomfield1925,bloomfield1946,schmirler2016,wolfart1996,hayes1995,starksballard2003,greensmith1985,ellis1983,pentland1979,davis1962,flynnetal2019}.

\begin{table}[h]
\caption{[{h}C] in Algonquian}
\label{tab:ex-algonquian-preaspiration}
	\begin{tabular}[t]{llllllllll}
		\lsptoprule
		Proto-Algonquian	& hp & ht & htʃ & hk & hs & hʃ & hθ & hl \\
		Plains Cree		& hp & ht & htʃ & hk &    &    &    & hj\\
		Woods Cree		& hp & ht & hts & hk &    &    &    & hð\\
		Moose Cree		& hp & ht & htʃ & hk &    &    &    & hl\\
		Swampy Cree		& hp & ht & htʃ & hk &    &    &    & hn\\
		Meskwaki		& hp & ht & htʃ & hk \\
		Menominee		& hp & ht & htʃ & hk & hs &  &  &  & hn \\
		Ojibwe			& hp & ht & htʃ & hk & hs & hʃ \\
		% Miami-Illinois		& hp & ht & htʃ & hk & hs & hʃ \\
		% Shawnee			& hp & ht & htʃ & hk & hs &  & hθ \\
		Massachusett		& hp & ht & htʃ & hk & hs & hʃ \\
		Delaware		& hp & ht & htʃ &  &  &  &  & hl & hm \\
		Cheyenne		& hp & ht & htʃ  & hk &   &   &   & hn\\
		Blackfoot		& hp & ht &  & hk & hs \\
		\lspbottomrule
	\end{tabular}
\end{table}

\begin{table}[h]
\caption{Emergence of preaspiration in Cree and Menominee. Reconstructions are from \citet{hewson1993}, Cree data from \citet{wolvengrey2011}, and Menominee data from \citet{bloomfield1962}}
\label{tab:ex-algonquian-diachronic-debuccalization}
\begin{tabular}[t]{lllG}
	\lsptoprule
	Proto-Algonquian  & Cree     & Menominee \\\midrule
	*teːnteːwa        & teːhteːw & teːhtew &  bullfrog  \\
	*aloːθkana        & ayoːskan & anoːhkan & raspberry  \\
	\lspbottomrule
\end{tabular}
\end{table}

Although many [{h}C] sequences are reconstructed back to Proto-Algonquian \citep{hewson1993}, some have been innovated. Diachronic loss of coda consonants has created additional instances of preaspiration before sonorants in Cree and Menominee (\glottolog{cree1272}, \glottolog{meno1252}; \citealp{bloomfield1946,pentland1979}), as shown in \tabref{tab:ex-algonquian-diachronic-debuccalization}. 
These examples show [h] appearing through loss of both fricatives and nasals. This indicates that a debuccalization account is not appropriate here: [h] is being used as a moraic filler for the lost coda.\footnote{The same process is shown for Hopi in the next section, and is also seen historically in North Germanic through the derivation of Old Norse (ON) geminates (later degeminated) from nasal--stop sequences: e.g., Latin \orthog{campus} \gloss{(battle)field} > ON \orthog{kapp} \gloss{contest} \citep{page1997}. I am grateful to an anonymous reviewer for pointing out this parallel.}

Loss of coda consonants is also the historical basis for preaspiration in Meskwaki (\glottolog{mesk1242}). In the example below, the final /t/ of the prefix \gloss{with} is maintained prevocalically (\ref{ex-meskwaki}a) but has become [h] before the root-initial [p] (\ref{ex-meskwaki}b).

\ea
\label{ex-meskwaki}
	\ea \orthog{wi:tapike:wa} \\
	\gll wi:t- api -ke: -wa \\
	     with- sit -\textsc{indef.obj} -\textsc{3sg} \\
    \glt \gloss{s/he sits with people} \citep{hewson1993} 
	\ex \orthog{wi:hpe:wa} \\
    \gll wi:t  -pe:   -wa \\
		 with  -sleep   -\textsc{3sg} \\
    \glt \gloss{s/he sleeps with someone} \citep[91]{bloomfield1946}
	\z
\z


In other cases, preaspiration may be lost diachronically. In Sheshatshiu Innu (\glottolog{mont1268}; also Innu-Aimun or Montagnais), loss of preaspiration caused compensatory lengthening of preceding vowels. This shows that prosodic readjustment was necessary after the loss of [h]. In \tabref{tab:ex-sheshatshiu-innu}, pre-Cree reconstructions are compared with modern Plains Cree (\glottolog{plai1258}) forms where preaspiration is retained, and with Sheshatshiu Innu forms where preaspiration has been replaced by vowel length. 

\begin{table} 
\caption{Loss of preaspiration in Sheshatshiu Innu. Pre-Cree and Sheshatshiu Innu forms are from \citeauthor{mackenzie1980} (\citeyear{mackenzie1980}: 68); Cree forms are from \citet{wolvengrey2011}}
\label{tab:ex-sheshatshiu-innu}
\begin{tabular}{lll G}
	\lsptoprule
	{Pre-Cree} & {Cree} & {Sheshatshiu Innu} \\\midrule
	*akuhp		& akohp   	& akuːp   & dress, blanket \\
	*miht		& mihti   	& miːt    & firewood \\
	*ispimihk	& ispimihk	& ispimiːt& above \\
	*atihkw		& atihk   	& atiːkw  & caribou \\
	\lspbottomrule
\end{tabular}
\end{table}


In Munsee (\glottolog{muns1251}), word stress follows a Weight-by-Position pattern, with coda [h] creating a heavy syllable (\citealp[212]{hayes1995}, with data from \citealp{goddard1982}). Feet in Munsee are iambic and built from the left edge of the word, with word-final feet being extrametrical. Stress appears on the final parsed foot. Degenerate feet appear to be permitted so long as they are heavy syllables. The example that Hayes provides is the word \orthog{awasáhkame:w} \gloss{in heaven}, where stress falls on the heavy syllable [sah]. Following \citet{hayes1995}, if this were not a heavy syllable, we would expect it not to be footed, and word stress would appear on [wa] instead. This is illustrated in (\ref{ex-munsee}), where the first line shows Hayes's metrical parse, and the second an unattested hypothetical parse in which [sah] does not form a degenerate foot. Parentheses delimit feet and angled brackets delimit final extrametrical syllables.

\ea Weight-by-Position in Munsee \citep[212]{hayes1989}\smallskip\\\label{ex-munsee}
\begin{tabular}[t]{@{}c@{}c@{\,}c@{\,}cc}
	& (a.wà) & (sáh) & <ka.mè:w> & \gloss{in heaven} \\
	{*} & (a.wá) & sah & <ka.mè:w> \\
	\end{tabular}
\z
Finally, in Ojibwe (\glottolog{ojib1241}), we find the reverse of the diachronic degemination seen in Icelandic. Most Ojibwe dialects underwent a shift in which the obstruents /p t k tʃ s ʃ/ split into a lenis and fortis series: some segments remained voiceless unaspirated singletons, but obstruents after /h/ became geminate /pː tː kː tʃː sː ʃː/ \citep{rhodestodd1981}. 


From this very brief overview of a few Algonquian languages, it is clear that preaspiration in the family is also in all respects a coda [h] segment. Its moraic affiliation is seen in weight-sensitive stress assignment and diachronic development into long vowels or long consonants.



\subsection{Hopi}
\label{sec-uto-aztecan}

In Toreva Hopi, preaspiration resulted from loss of historical coda consonants, which were usually nasals \citep{manasterramer1986}. As in the Algonquian languages, this particular pathway of development poses difficulties for a conception of preaspiration as a [spread glottis] feature, and more strongly suggests preaspiration as a mechanism for maintaining weight within a syllable. Preaspiration was already in the process of being lost in Toreva Hopi around the time of early fieldwork by \citeauthor{whorf1936} (\citeyear{whorf1936,whorf1946}), as it is absent before fricatives in his transcriptions. Later work on the language shows long vowels (as well as tonogenesis) resulting from loss of earlier preaspiration. Some illustrative data is given in \tabref{tab:ex-hopi-preaspiration-loss}.

\begin{table}
\caption{Loss of preaspiration in Toreva Hopi \citep{whorf1936,malotki1979}}
\label{tab:ex-hopi-preaspiration-loss}

\begin{tabular}[t]{llG}
\lsptoprule
\citet{whorf1936}& \citet{malotki1979} \\\midrule
{[wɨʰti]}	& [wɨ́ɨ̀ti]	& woman \\
{[leʰpe]}	& [léèpe]	& to fall \\
{[kɨʰkɨ]}	& [kɨ́ɨ̀kɨ]	& foot \\
\lspbottomrule
\end{tabular}
\end{table}


\subsection{Other languages}
\label{sec-other}

There is not enough space here for a full detailed discussion of every language where preaspiration is attested: these cases are far more numerous than is generally thought. Preaspiration is also attested in English, Irish, Welsh, Italian, Mongolian, Bora, Achumawi, and Purépecha, as well as in multiple languages in the Tibetan, Pomoan, Panoan, Tucanoan, Oto-manguean, Iroquoian, Cariban, Arawakan, Uralic, and Uto-Aztecan families. However, similar prosodic patterns are seen in many of these instances of preaspiration, and similar arguments for the phonological distinctness of preaspiration from the following stop and for its moraic weight can be made in all cases. The patterns of preaspiration in Ecuadorian Siona \parencitetv{chapters/18_vantveer} are also largely compatible with a view of [h] as a prosodic element. More detailed discussion of other languages and supporting evidence 
can be found in \citet{craioveanu-thesis}. See also \textcitetv{chapters/16_Hejná} for a typological database of attested cases of preaspiration.


One issue that must be abbreviated here, but which recurs in some of the omitted languages and is worth mentioning, is that of initial preaspiration. As noted in \S\ref{sec-introduction}, traditional accounts of positional allophony in aspiration describe aspirated phonemes, which are realized with postaspiration word-initially, but with preaspiration in word-medial or word-final positions \citep[e.g.,][]{LadefogedMaddieson1996,Silverman2003,Clayton:2010}. However, preaspiration is also found in word-initial position (although not in utterance-initial position) in languages like Mongolian \citep{svantesson2005,svantessonkarlsson2012} and Bora \citep{thiesen1998,thiesenweber2012}, among others. In these cases, preaspiration is only found word-initially if there is another syllable before it to ``host'' the [h]. This is the case for Mongolian, where preaspiration is phonetically realized through partial devoicing of a preceding segment: preaspiration appears before fortis stops in all positions except the very beginning of an utterance, where there is no preceding segment \citep{svantessonkarlsson2012}. Preaspiration is phonetically shorter in this word-initial position, but co-occurs there with postaspiration; thus, it is not quite accurate to say that preaspiration alternates with postaspiration. 

In Bora, preaspiration appears before some aspirated stops. This is lexically determined, and not all aspirated stops have a preceding [h]. This preaspiration is also found root-initially, appearing before some root-initial postaspirated stops. However, preaspiration only surfaces when there is a preceding CV syllable within the same phonological word, such as in genitive constructions or with proclitic pronouns \citep{thiesenweber2012}. This is illustrated in (\ref{ex-bora}), with initial preaspiration marked in bold when it surfaces. The root has no initial preaspiration when in isolation in (\ref{ex-bora}a), but [h] appears when immediately preceded by a short vowel from a pronoun (\ref{ex-bora}b) or in a genitive construction (\ref{ex-bora}c). The appearance of initial preaspiration is clearly dependent on the (morpho)phonological structure of the word, as the same demonstrative does not permit preaspiration to appear when it is structurally more distant from the root (\ref{ex-bora}d).

\ea Root-initial preaspiration in Bora \citep[41, 51]{thiesenweber2012}\label{ex-bora}
	\NumTabs{4}
	\ea\relax [tsʰɨ̀ːmɛ̀nɛ̀]\tab `child' \\
	\ex\relax [tʰá-\textbf{h}tsʰɨ̀ːmɛ́nɛ̀]\tab `my child' \\
	\ex\relax [áːnɯ́ \textbf{h}tsʰɨ́ːmɛ́nɛ̀]\tab `this one's child' \\
	\ex\relax [áːnɯ́ tsʰɨ́ːmɛ́nɛ̀]\tab `this one is a child' \\
	\z
\z

There is complementary distribution in Bora between long vowels and codas, which can only be glottal [h] or [ʔ]. If initial preaspiration is preceded by a long vowel or a coda, only one of these can surface. In some cases the appearance of the root-initial [h] is blocked, and in other cases it surfaces but replaces a [ʔ] coda \citep{thiesenweber2012}. These patterns are not illustrated here as they are complex and poorly understood, but it is clear from the complementary distribution of codas, long vowels, and initial preaspiration that these all fully take up the second mora of a syllable. Therefore, Bora illustrates a case where initial preaspiration exists underlyingly but cannot be realized phonetically unless associated to a mora. Furthermore, postaspiration and preaspiration appear to be completely independent and do not show positional allophony: postaspiration appears in all cases, and preaspiration is entirely lexical in its distribution.


\section{Discussion}
\label{sec-discussion}

The widespread conception of preaspiration is that it is a property of stops, and specifically an unusual phonetic implementation of the contrast between fortis and lenis stops. This is a result of a focus within previous research on preaspiration, which has centered around languages where preaspiration appears only before stops and is frequently the result of historical degemination. Based on a small typological survey and in line with \citet{Helgason2002}, {\citeauthor{Clayton:2010}} (\citeyear{Clayton:2010}: 32) stipulates that ``true'' preaspiration ``should be phonotactically associated with stops to the (near) exclusion of other types of consonants''. The intention behind this narrow definition of preaspiration is to exclude cases of coda [h], which are assumed to be inherently different. However, I argue that this viewpoint has excluded other examples of preaspiration from phonological consideration.
Preaspiration occurs before a wider range of segments than just stops, and does not show different phonological properties in those contexts. In some cases, preaspiration has only been described before stops, even when it occurs before other segments as well. This is the case for southwestern Welsh, where only fortis stops are reported to be preaspirated \citep{iosad2017}, but examination of the corresponding open-source dataset shows that voiceless fricatives are preaspirated as well \citep{iosad-welshdata}.

If preaspiration is not in fact a phonetic or featural property of a following stop, the dichotomy between preaspiration and coda [h] falls apart. The alternate view, that preaspiration is ``merely'' a coda [h] or other voiceless segment, and not the mirror image of postaspiration at all, should not make preaspiration seem mundane or phonologically uninteresting. Such a view allows us to investigate phonological patterns from a different angle and see broader patterns. It also means that the restricted distribution of preaspiration in languages like Icelandic is simply the result of a historical sound change: if degemination of fortis geminates conditioned the appearance of preaspiration, then finding it only before fortis stops in the modern language is unsurprising. Restrictions on coda inventories and distribution are not unusual within languages, so having preaspiration be part of the shape of the syllable rather than part of the stop contrast is no less desirable. 

In the typological survey in \S\ref{sec-typology}, preaspiration is shown to be complementary with vowel or consonant length, and is seen across numerous languages to fulfil a role in making a syllable heavy or compensating for the loss of an oral consonant. In some of these instances, it shows parallels with vowel lengthening and glottal stop insertion. Given this prosodic patterning, I propose that (at least in some languages) preaspiration may serve as a phonological lengthening strategy, used wherever vowel lengthening is not desirable, or where multiple systems of prosodic weight are necessary to preserve quantity contrasts. In South Argyll Gaelic, preaspiration is complementary with glottal stop insertion and vowel lengthening as a (historical) strategy to ensure heavy stressed syllables. In Spanish, coda debuccalization is a form of preaspiration for weight preservation, where compensatory vowel lengthening and gemination also exist as an alternative weight preservation strategy. In this case, it is also clear that preaspiration is not the realization of an underlying [spread glottis] feature, as it may appear in place of voiced sonorants as well.  

\begin{sloppypar}
The consistent moraic association observed across cases of preaspiration aligns well with previous proposals by \citet{kehreingolston2004} and \citet{golstonkehrein2013}, in which the feature [spread glottis] is dependent directly on prosodic structure rather than being a segmental feature. This interpretation of aspiration allows Golston and Kehrein to account for the absence of laryngeal contrasts between two segments in the same syllabic constituent, and the alignment of laryngealization at word edges. It also easily accounts for segment devoicing as a phonetic outcome of (pre)aspiration. Although in the present work I have suggested that preaspiration can be understood as a distinct coda consonant, the facts I have discussed here are also compatible with a model where [spread glottis] shifts to, or is inserted to fill, an empty moraic position, and is thus directly dependent on prosodic structure in a way that echoes {\citeauthor{kehreingolston2004}'s (\citeyear{kehreingolston2004})} syllable-structure-dependent aspiration. This may allow a more streamlined and unified analysis of the wide range of different types of phenomena associated with preaspiration (see \citetv{chapters/16_Hejná}). In particular, patterns of vowel devoicing (e.g., Udihe: \citealp{kuznetsova2022}) and sonorant devoicing (e.g., Mongolian: \citealp{svantessonkarlsson2012}) have been set aside here but are areas where association of [h] to levels of prosodic structure may prove insightful.
Ultimately, the key insight of this research is that preaspiration should be approached phonologically as a distinct element that has moraic association, and not a feature of a weightless onset consonant. In some cases, this will mean preaspiration is for all intents and purposes a distinct segment, and this will be the simplest way of viewing it. However, full segmental status is not mandatory for moraic association.\footnote{Regarding the segmental status of aspiration, a reviewer notes that in southeastern Welsh dialects, [h] has been lost in all contexts other than preaspiration \citep{iosad2023}, so declaring it to nonetheless be a distinct consonant may be undesirable. This is similar to the case of Italian, which is traditionally described as lacking aspiration or phonemic /h/, but where preaspiration nonetheless appears before voiceless geminates in several dialects \citep{stevenshajek2004,stevenshajek2007,kramer2009,stevens2011}. In instances like this, I am not claiming that phonemic /h/ is necessary within the language's inventory: in fact, patterns like these support the idea that preaspiration can be used as a strategy for occupying prosodic space, regardless of whether [h] is otherwise a contrastive element in the inventory. These may be instances in which the preaspiration is part of the phonetic module of grammar, since otherwise we expect the phonology to only make reference to contrastive features \citep{dresher2009}.}
\end{sloppypar}

One issue relating to the origins of preaspiration that remains to be investigated more fully is the phonologization of this phenomenon. The boundary between phonetic and phonological phenomena is blurred somewhat in this chapter by the use of phonetic evidence to establish formal phonological structures. Under a modular view of phonetics and phonology, we expect that preaspiration should have a phonetic origin and become more regularized and categorical over time. Therefore, both phonetic and phonological preaspiration should be attested. It is possible that languages with a variable realization of preaspiration (e.g., Central Standard Swedish, Faroese) reflect a preaspiration pattern within the phonetic module of grammar, while languages with categorical preaspiration (e.g., North Sámi) reflect phonological preaspiration. Critically, however, a conception of phonetic preaspiration must still capture the affinity of preaspiration for moraic positions, as its emergence in these positions is not coincidental.

Given this reinterpretation of preaspiration as a compensatory prosodic mechanism, and the broader scope of the typological surveys in \citet{craioveanu-thesis} and \textcitetv{chapters/16_Hejná}, it is worth considering whether preaspiration is actually a rare phenomenon at all. \textcitetv{chapters/02_Iosad} approaches this question at length as well. As noted in \S\ref{sec-assumptions}, speculation that preaspiration is under-reported is not new \citep{nichasaide1985,hejna2015,hejna2019,iosad2017-mfm,iosad2018}. I follow this previous work in suggesting that preaspiration is much less rare than previously thought. Thorough recent phonological descriptions of languages like Norwegian, Swedish, and Welsh do not discuss preaspiration or include it in transcriptions, even though it is widespread in each of these languages \citep{kristoffersen2000,riad2014,hannahs2013}, so it is quite plausible that it has been overlooked in other languages as well. Why might this be? 

One possibility is that fieldworkers simply did not notice it, were not expecting to find it, or did not consider it important to mention; \citet{iosad2018} discusses these possibilities for Norwegian. Regarding expectations, the general impression of what preaspiration is and of what counts as ``real'' preaspiration are important here: since preaspiration is thought to be rare, it might not be considered as a possibility in a given language. Alternately, it is quite possible that restrictive definitions of preaspiration in previous literature have limited the degree to which preaspiration is identified. There are also linguistic factors that could contribute to this rarity: competition from alternative prosodic weight-marking strategies like gemination and compensatory lengthening could reduce the frequency of preaspiration, or replace it in a given language. There may also be a general dispreference for glottal [h] in coda position, which would be balanced against (or at odds with) the moraic affinity shown by preaspiration. This coda [h] dispreference underlies the tendency for preaspiration to undergo fortition to an oral fricative or to be deleted, as well as a tendency in some languages to avoid glottal /h/ geminates intervocalically (e.g., Swedish: {\citealp[45]{riad2014}}; Maltese: \citealp{mitterer2018}).

In this chapter, I have presented a critical reassessment of the phonological status of preaspiration, revisiting well known examples of preaspiration and outlining some new ones in an abridged typology. The patterns described here show that preaspiration has intrinsic phonological weight, and is prosodically distinct from the following segment rather than a feature of it.



\section*{Abbreviations}
\begin{multicols}{3}
\begin{tabbing}
\textsc{mmmm} \= accusative \kill
\textsc{acc} \> accusative \\
\textsc{dat} \> dative \\
\textsc{def} \> definite \\
\textsc{f} \> feminine \\
\textsc{gen} \> genitive \\
\textsc{imp} \> imperative \\
\textsc{indef} \> indefinite \\
\textsc{inf} \> infinitive \\
\textsc{m} \> masculine \\
\textsc{n} \> neuter \\
\textsc{nom} \> nominative \\
\textsc{obj} \> object \\
\textsc{pl} \> plural \\
\textsc{refl} \> reflexive \\
\textsc{sg} \> singular \\
\textsc{subj} \> subjunctive
\end{tabbing}
\end{multicols}

\section*{Acknowledgements}

I am grateful for the detailed comments and suggestions on this chapter from Natalia Kuznetsova, Pavel Iosad, and two anonymous reviewers. This feedback has both improved the present work and been thought-provoking about issues in preaspiration and phonology more generally.  I am also grateful for questions and comments on earlier iterations of this work from audiences at the CLA annual meeting at the University of British Columbia in 2019, the SPIPS workshop at the University of Tromsø in 2019, and workshops at the University of Toronto (MOT 2019 \& SPF 2022). Finally, I would like to thank Keren Rice for her steadfast support and valuable discussions throughout the progression of this research. 


\sloppy
\printbibliography[heading=subbibliography,notkeyword=this]
\end{document}
