\documentclass[output=paper]{langscibook}
\ChapterDOI{10.5281/zenodo.15148166}
\author{Robert Blust\orcid{}\affiliation{University of Hawaiʻi, Mānoa}}
\title{*b > \textit{-k-}: A Berawan sound change for the ages}
\abstract{Berawan, an Austronesian language spoken in northern Sarawak, Malaysian Borneo, is one of several languages in central and western Borneo that have unusually innovative phonologies. Not only are these phonologies rich in number of changes, and the effect they sometimes have on concealing cognation (e.g., Malay \textit{bǝruaŋ} and Long Terawan Berawan \textit{kǝbiŋ} ‘the Malayan sun bear: \textit{Ursus malayanus}’ are cognate), but the search for theoretically-supported motivations for some changes leads nowhere. One of these changes in all dialects of Berawan is *b > \textit{k} in intervocalic position, a change that is abundantly attested, and therefore not in question as a valid transition from an earlier to a later state of the language. A basic question is whether this was a one-step change, or a telescoping of several phonetically more ‘natural’ changes, and while it can be resolved into a two-step change, this hardly relieves our sense of theoretical angst, since so far as the evidence allows us to infer, these changes were *b > \textit{g} in intervocalic position, followed by intervocalic devoicing of \textit{g} from two historically distinct sources. Attempts so far to show that this change involved other intermediate steps have yet to be successful.

\keywords{Berawan; Austronesian languages, rare sound change, unnatural sound change, social factors}
}
\IfFileExists{../localcommands.tex}{
  \addbibresource{../localbibliography.bib}
  \usepackage{tabularx, multicol, multirow, longtable}
\usepackage{url}
\urlstyle{same}

\usepackage{orcidlink}
\definecolor{orcidlogocol}{cmyk}{0,0,0,1}
\RenewDocumentCommand{\LinkToORCIDinAffiliations}{ +m }
  {%
    \orcidlink{#1}\,%
  }
\SetupAffiliations{orcid placement=before}

\usepackage{siunitx}
\sisetup{detect-weight=true, detect-family=true, group-digits=none}

\usepackage{mathtools}
\usepackage{langsci-optional}
\usepackage{langsci-lgr}
\usepackage{langsci-gb4e}

\usepackage{stmaryrd}
\usepackage[capitalize]{cleveref}
\babelfont[macedonian]{rm}[Language=Macedonian,ItalicFont=LibertinusSerif-Italic.otf]{LibertinusSerif-Regular.otf}
\usepackage{eqparbox}
\usepackage[autostyle]{csquotes}
\usepackage[linguistics]{forest}

\usetikzlibrary{positioning, matrix}
\usepackage[glosses,inline]{leipzig}
\PassOptionsToPackage{xindy,toc,nopostdot}{glossaries}
\usepackage{glossary-inline}
\setglossarystyle{inline}
\makeglossaries

\usepackage{phonrule}
\usepackage{threeparttable}


\usepackage{textcomp,gensymb}


\usepackage[preservefont]{tipauni}

\usepackage[normalem]{ulem}

\usepackage{enumitem} %so lists aren't ugly
	
\usepackage{threeparttable}	%allows tables with tablenotes. note marks: †‡
	\makeatletter 
	\g@addto@macro\TPT@defaults{\footnotesize} 
	\makeatother

\usepackage{colortbl}
	\definecolor{Pink}{rgb}{0.96, 0.76, 0.76} 
	\definecolor{PaleBlue}{rgb}{0.67, 0.9, 0.93}
	\definecolor{carolinablue}{rgb}{0.6, 0.73, 0.89}
	\definecolor{goldenyellow}{rgb}{1.0, 0.87, 0.0}
	\definecolor{Orange}{rgb}{1.0, 0.66, 0.07}
	\definecolor{puce}{rgb}{0.8, 0.53, 0.6}
	\definecolor{turquoisegreen}{rgb}{0.63, 0.84, 0.71}


% add all extra packages you need to load to this file
\usepackage{langsci-textipa}
\usepackage{vowel}
\usepackage{textgreek}

% \usepackage{langsci-branding}
% \usepackage{subcaption}
\usepackage{subfigure}

\usepackage{tabto}


\usetikzlibrary{tikzmark}
\usepackage{pgfplots}


\newfontfamily\tibetan{NotoSerifTibetan-Regular.ttf}
\usepackage{langsci-branding}
\usepackage{hyphenat}

\usepackage{accents}

  \renewcommand{\lsChapterFooterSize}{\footnotesize}

\makeatletter
\let\thetitle\@title
\let\theauthor\@author
\makeatother

\newcommand{\togglepaper}[1][0]{
   \bibliography{../localbibliography}
   \papernote{\scriptsize\normalfont
     \theauthor.
     \titleTemp.
     To appear in:
     Natalia Kuznetsova, Cormac Anderson \& Shelece Easterday (ed.).
     Rarities in phonetics and phonology.tex.
     Berlin: Language Science Press. [preliminary page numbering]
   }
   \pagenumbering{roman}
   \setcounter{chapter}{#1}
   \addtocounter{chapter}{-1}
}

\newbool{bookcompile}
\booltrue{bookcompile}
\newcommand{\bookorchapter}[2]{\ifbool{bookcompile}{#1}{#2}}

\newcommand{\textarab}[1]{\RL{\arabicfont #1}}

\newcommand\mb[1]{\eqparbox[t]{examples}{#1}\hspace{1em}}
\newcommand\mbi[1]{\mb{#1}}
\newcommand{\twe}[3]{\mbi{#1}\eqparbox[t]{orths}{\emph{#2}}\hspace{1em}`#3'\hspace{1em}} % three-way example
\providecommand\glottocode[1]{[\href{https://glottolog.org/resource/languoid/id/#1}{#1}]}
\newcommand{\phonreal}[1]{\ensuremath{\llbracket}#1\ensuremath{\rrbracket}}

\DeclareRobustCommand\dash{\unskip\nobreak\thinspace\textendash\allowbreak\thinspace\ignorespaces}

\forestset{minus/.style={edge label={node[midway, left] {\ensuremath{-}\hspace*{2mm}}}},
plus/.style={edge label={node[midway, right] {\hspace*{2mm}\ensuremath{+}}}}}
\providecommand\ipa[1]{#1}


\newcommand{\tone}[1]{\textsuperscript{#1}}

\newcommand{\orthog}[1]{\textit{#1}}
\newcommand{\gloss}[1]{`#1'}

\newcommand{\glottolog}[1]{\texttt{\href{https://glottolog.org/resource/languoid/id/#1}{#1}}}

\newcolumntype{O}{>{\itshape }l<{}}
\newcolumntype{G}{>{`}l<{'}}

\newcounter{tabsubcounter}
\newcommand{\tablecounter}{\setcounter{tabsubcounter}{0}}
\newcommand{\TC}{\stepcounter{tabsubcounter}\alph{tabsubcounter}.}

\usetikzlibrary{chains,positioning,calc,decorations.markings}
\tikzset{
	seg/.style={text height=0.6em, text depth=0.3em},
	moraic-structure/.style={xscale=0.6,yscale=1.1, text height=0.65em,text depth=0.25em},
 }

%05_Culhane_Edwards
%%%%%%%%%%%%%%%%%%%%%%%%%%%%%%%%
%%	Symbols and Characters  	%%
%%%%%%%%%%%%%%%%%%%%%%%%%%%%%%%% αβσµ

\newcommand{\tl}{\char`~}						%middle tilde ~
\renewcommand{\Q}{\textquotesingle}		%straight apostrophe444
\newcommand{\ra}{→} 								%right arrow ->
\newcommand{\0}{∅} 									%zero symbol
\newcommand{\gap}{\textunderscore} 	%underscore
%\renewcommand{\j}{ʤ}								%dezh digraph
\newcommand{\syll}{σ}								%lowercase sigma medial form
\newcommand{\wrd}{ω}								%lowercase omega
\newcommand{\ft}{φ}									%lowercase phi
\newcommand{\gw}{gʷ}								%g with superscript w
\newcommand{\B}{β}									%voiced bilabial fricative
\newcommand{\hp}{\hphantom}					%space equal to width of argument
\newcommand{\it}{\textit}	%italics

%%%%%%%%%%%%%%%%%%%%%%%%%%%%%%%%
%%	Font Styles & Formatting	%%
%%%%%%%%%%%%%%%%%%%%%%%%%%%%%%%%

\definecolor{DarkBlue}{RGB}{0,0,130}										%dark blue colour
% \newcommand{\ve}[1]{\textcolor{DarkBlue}{\textit{#1}}}	%vernacular text
\newcommand{\ve}[1]{{\textit{#1}}}	%vernacular text
\definecolor{DarkRed}{RGB}{150,0,0}											%dark red colour
% \newcommand{\tbr}[1]{\textcolor{DarkRed}{\textbf{#1}}}	%Bold red text
\newcommand{\tbr}[1]{{\textbf{#1}}}	%Bold red text
%\renewcommand{\it}{\textit}																%italics
\newcommand{\tsc}{\textsc}															%small caps
\newcommand{\sub}{\textsubscript}												%subscript
\newcommand{\su}{\textsuperscript}											%superscript

%%%%%%%%%%%%%%%%%%%%%%%%%%%%%%%%%%%%%%%%%%%%%%%%%%%%
%% Tables %% Tables %% Tables %% Tables %% Tables %%
%%%%%%%%%%%%%%%%%%%%%%%%%%%%%%%%%%%%%%%%%%%%%%%%%%%%

% \newcommand{\mc}{\multicolumn}									%multicolumn
% \newcommand{\st}[1]{\setlength{\tabcolsep}{#1}}	%reduce column width in tables
%
%%%%%%%%%%%%%%%%%%%%%%%%%%%%%%%%
%%    Cross   References      %%
%%%%%%%%%%%%%%%%%%%%%%%%%%%%%%%%

% \def\Plus{\texttt{+}}
% \def\Minus{\texttt{-}}
% \newcommand{\GS}{ʔ}
% \def\SH{ʃ}
% \newcommand{\TSH}{ʧ}
% \def\ZH{ʒ}
% \def\DZH{ʤ}
% \def\:{ː}
% \def\UP{\textsuperscript}
% \def\rs{ʂ}
% \newcommand{\rn}{ɳ}
% \def\rt{ʈ}
% \def\tllr{ɺ}
% \newcommand{\Bb}{β}
% \def\Eps{ɛ}
% \def\Oo{ɔ}
% \def\Gm{ɣ}
% \def\NG{ŋ}
% \def\barU{ʉ}
\newcommand{\CM}{\ding{51}}
\newcommand{\XM}{\ding{53}}
% \newcommand{\tap}{ɾ}
% \def\darkL{ɫ}
% \def\schwa{ə}
%
% \def\BUL{\textbullet}


%%%%%%%%%%%%%%
%					%
%	Secondaries		%
%					%
%%%%%%%%%%%%%%
%	Post
\newcommand{\Post}[2]{#1\textsuperscript{#2}}
%	Pre
\newcommand{\Pre} [2] {\textsuperscript{#1}#2}
%	Undertilde
\newcommand{\utilde}[1]{\ensuremath{\smash{\underset{\mathclap{\sim}}{\text{#1}}}}}
%	Devoiced
% \newcommand{\VCLS}[1]{\textsubring{#1}}
%%%%%%%%%%%
%				%
%	Definitions		%
%	Markup		%
%				%
%%%%%%%%%%%
% \def\->{$\rightarrow$}
% \def\__{\underline{\hspace{1em}}}
\def\NoPoss{\cellcolor{gray!30}}

\newcommand{\VOICELESS}{\textsc{voiceless}}
\newcommand{\VOICED}{\textsc{voiced}}
\newcommand{\tablenote}[2][1]{\parbox{#1\textwidth}{\footnotesize\raggedright #2}}

\newcommand{\appref}[1]{Appendix~\ref{#1}}
\renewcommand{\sectref}[1]{Section~\ref{#1}}


\newcommand{\dobuibox}[5]{#1\\[-1.1em]
\hspace*{-.8cm}
 \begin{tabularx}{.9\textwidth}{@{}lQQ@{}}
       &  {oral} &  {nasal} \\
       \midrule
     {controlled} &\parbox[t]{4cm}{\raggedright  #2} & \parbox[t]{4cm}{\raggedright #3} \\
     \tablevspace
     {ballistic} &\parbox[t]{4cm}{\raggedright  #4} & \parbox[t]{4cm}{\raggedright  #5} \\
 \end{tabularx}
}

\newfontfamily\VdottildeFont{LibertinusVdottilde.otf}

\newcommand{\Vdottilde}{{\VdottildeFont V̰̣}}

% \renewcommand{\keywords}[1]{\textbf{#1}}
 
  %% hyphenation points for line breaks
%% Normally, automatic hyphenation in LaTeX is very good
%% If a word is mis-hyphenated, add it to this file
%%
%% add information to TeX file before \begin{document} with:
%% %% hyphenation points for line breaks
%% Normally, automatic hyphenation in LaTeX is very good
%% If a word is mis-hyphenated, add it to this file
%%
%% add information to TeX file before \begin{document} with:
%% %% hyphenation points for line breaks
%% Normally, automatic hyphenation in LaTeX is very good
%% If a word is mis-hyphenated, add it to this file
%%
%% add information to TeX file before \begin{document} with:
%% \include{localhyphenation}
\hyphenation{
    af-fri-cates
    al-ve-o-pal-a-tal
    Ama-nu-ban
    Ara-wak-an
    Árna-son
    Ber-ber
    can-di-dates
    Cam-er-oon
    Chi-nan-tec
    Chir-ko-va
    Crai-o-ve-a-nu
    di-chot-o-my
    Ec-ua-do-rian
    Ec-ua-dor
    elec-tro-glot-to-gra-phy
    Faro-ese
    Ike-ma
    Kuznet-sova
    Mes-kwa-ki
    Mio-ma-fo
    mono-mor-aic
    Ne-ca-xa
    Oto-man-gue-an
    par-a-digm
    post-as-pi-rat-ed
    post-as-pi-ra-tion
    pre-as-pi-rat-ed
    pre-as-pi-ra-tion
    pros-o-dic
    pros-o-dies
    re-con-struc-table
    Sheh-ret
    Svan-tes-son
    Ta-ras-can
    Tórs-havn
    Ural-ic
    epen-the-sis
    Anin-dil-yak-wa
    Mi-nyag
    Na-ka-ma
}

\hyphenation{
    af-fri-cates
    al-ve-o-pal-a-tal
    Ama-nu-ban
    Ara-wak-an
    Árna-son
    Ber-ber
    can-di-dates
    Cam-er-oon
    Chi-nan-tec
    Chir-ko-va
    Crai-o-ve-a-nu
    di-chot-o-my
    Ec-ua-do-rian
    Ec-ua-dor
    elec-tro-glot-to-gra-phy
    Faro-ese
    Ike-ma
    Kuznet-sova
    Mes-kwa-ki
    Mio-ma-fo
    mono-mor-aic
    Ne-ca-xa
    Oto-man-gue-an
    par-a-digm
    post-as-pi-rat-ed
    post-as-pi-ra-tion
    pre-as-pi-rat-ed
    pre-as-pi-ra-tion
    pros-o-dic
    pros-o-dies
    re-con-struc-table
    Sheh-ret
    Svan-tes-son
    Ta-ras-can
    Tórs-havn
    Ural-ic
    epen-the-sis
    Anin-dil-yak-wa
    Mi-nyag
    Na-ka-ma
}

\hyphenation{
    af-fri-cates
    al-ve-o-pal-a-tal
    Ama-nu-ban
    Ara-wak-an
    Árna-son
    Ber-ber
    can-di-dates
    Cam-er-oon
    Chi-nan-tec
    Chir-ko-va
    Crai-o-ve-a-nu
    di-chot-o-my
    Ec-ua-do-rian
    Ec-ua-dor
    elec-tro-glot-to-gra-phy
    Faro-ese
    Ike-ma
    Kuznet-sova
    Mes-kwa-ki
    Mio-ma-fo
    mono-mor-aic
    Ne-ca-xa
    Oto-man-gue-an
    par-a-digm
    post-as-pi-rat-ed
    post-as-pi-ra-tion
    pre-as-pi-rat-ed
    pre-as-pi-ra-tion
    pros-o-dic
    pros-o-dies
    re-con-struc-table
    Sheh-ret
    Svan-tes-son
    Ta-ras-can
    Tórs-havn
    Ural-ic
    epen-the-sis
    Anin-dil-yak-wa
    Mi-nyag
    Na-ka-ma
}
 
  \togglepaper[3]%%chapternumber
}{}

\begin{document}
\maketitle 
%\shorttitlerunninghead{}%%use this for an abridged title in the page headers
% ATTENTION: Diacritics on the following phonetic characters might have been lost during conversion: {'ə'}



\section{Introduction}\label{sec:Blust-Introduction}
Bizarre sound change is not a topic that is well-suited to congenial dinner chat. Before they know it, the conversationalists are apt to snatch up fork and knife, and face off across the table in a confrontational mood, ready for battle. The defender of theoretical orthodoxy demands: “How do you \emph{know} that this unexpected transition was a single change, and not the telescoping of multiple smaller and (of course) more expected sound changes?”, to which the other, ignoring his eggplant, responds with equal passion: “How do \emph{you} know that there were 16 baby steps between *x and /y/ if there is no direct evidence of them?” In the end, those who are unwary enough to be dragged into this kind of conversation are more likely to experience indigestion than enlightenment.

That is the risk I take here in introducing to the world of linguists a piece of the phonological history of Berawan, a cluster of four closely-related Austronesian languages or divergent dialects of a single language spoken in the basin of the Baram river in northern Sarawak, Malaysian Borneo. The Berawan languages (Long Teru, Long Jegan, Batu Belah and Long Terawan) belong to the North Sarawak subgroup, which comprises four primary branches: 

\begin{enumerate}
\item Dayic (Lun Dayeh/Lun Bawang, Kelabit, Sa’ban, Tring, etc.).
\item Kenyah (many communities in central Borneo).
\item Berawan-Lower Baram (the Berawan languages and others in the lower course of the Baram river, such as Kiput, Narum, and Miri).\footnote{The Berawan-speaking communities are Long Terawan, on the Tutoh branch of the Baram river, Long Teru and Long Jegan, on the Tinjar branch of the Baram, and Batu Belah, on the Apoh branch of the Baram. Place names in Sarawak often are based on a reflex of *əluŋ ‘confluence, place where two rivers meet’ (hence ‘Long Teru’ is at the confluence of the small Teru river with the larger Tinjar). Batu Belah, by contrast, means ‘split rock’, and shows up in more than one part of the Austronesian world as a place name based on a topographical feature.}
\item Bintulu.
\end{enumerate}
The Berawan group itself breaks down into two primary branches:

\begin{enumerate}
\item Northern Berawan (Long Terawan).
\item Southern Berawan (Long Teru, Batu Belah, Long Jegan).
\end{enumerate}

Gluttons for punishment can find further details in \citet{Blust2010}.

\section{Sound change and reflex}\label{sec:Sound-change-reflex}
Any historical linguist worth his salt knows the difference between a sound change and a reflex.  Nonetheless, when he attempts to express the mapping between the phonological shape of a language at earlier and later states in the most neutral terms possible, well-meaning critics feel little restraint in peppering him with questions of the form “Have you thought about intermediate steps x, y and z?” To some extent, this is because the term ‘sound change’ is often used loosely either for a single phonetic transition from one state to another, as in the title of this chapter, or for a reflex, which may encode an accumulated history of changes (but would you want to read a chapter titled ‘*b > \textit{-k-}: A Berawan reflex for the ages’?).

However, more generally, this insistence on breaking down phonetically puzz\-ling reflexes into
smaller, phonetically motivated steps is due to an ideological stance that adheres consciously or unconsciously to the proposition “Most sound changes are phonetically motivated; \textit{therefore} all sound changes are phonetically motivated.” We need not comment on the groans coming out of Aristotle’s grave when we dare to state something so outrageous to basic principles of logic, but it is hard not to infer this presupposition when theoretical purists insist that a change such as Proto-Polynesian *l > Rennellese /g/ ([ŋg]), or *w > Sundanese \textit{c-, -nc}- \textit{must} have involved intermediate steps \citep{Blust2005}.

Science in all its manifestations is obviously more than a collection of observations about the real world. What organizes these observations into a network of mutually-supportive data is the framework of theoretical constructs that show \textit{why} these observations take the form they do as observable consequences of an unobservable reality. One can think of this as triangulation: at the base of the triangle are two or more (possibly many more) observations about the world that may or may not be causally connected, and at the top is a theoretical construct that cannot be observed, but is justified by its ability to show that various superficially dissimilar observations are the expected consequences of a single underlying reality. Although we daily place our trust in its existence, no one has ever seen gravity, yet we accept it because of a wide range of sensory impressions that tell us it must exist. A proto-language is a theoretical construct that occupies the top of a theoretical triangle much like gravity does, and it is justified only to the extent that it serves to explain diverse observations about languages that follow as expected consequences of its existence. In the rest of this chapter, I will adhere to that basic principle of good science called “Occam’s razor”, meaning that I will assume only what is necessary to explain the primary observations that make up a scientific corpus as the expected consequences of an underlying reality that has been independently justified through prior reference to a wider range of other observations.

\section{Before Berawan}\label{sec:Before-Berewan}
  As already noted, the Berawan languages form part of a Berawan-Lower Baram subgroup that itself is one of four primary branches of the North Sarawak subgroup of Austronesian languages. Suffice it to say that there are various proto-languages to which the sound changes in Berawan could refer. In the interest of citing forms that are well-known to a relatively large number of people (at least those familiar with Austronesian historical linguistics), one might refer to Proto-Malayo-Polynesian (PMP), the hypothetical ancestor of the non-Formosan Austronesian languages. However, because Proto-North Sarawak (PNS) had already undergone several important sound changes that are relevant to discussing the phonological history of Berawan, I will generally use PNS reconstructions, and only resort to higher-level (and hence more widely-known) reconstructions if this sheds additional light on the problem at hand.

Proto-North Sarawak had a simple four-vowel system (*i, *u, *a and the schwa *ə), and the consonant system shown in \cref{tab:PNS-Cs}. %, where 1 = bilabial, 2 = dental/alveolar, 3 = palatal, 4 = velar, 5 = uvular, and 6 = glottal.
Canonical shape of base morphemes was CVCVC, or less commonly CVNCVC, where N was a nasal homorganic with the following obstruent.

\begin{table}
\caption{\label{tab:PNS-Cs}The Proto-North Sarawak consonant system}
\fittable{%
\begin{tabular}{lllllll}
\lsptoprule
                        &  bilabial & dental/alveolar & palatal & velar & uvular&  glottal\\
\midrule
voiceless plosive       &  p & t &   & k &   & Ɂ \\
voiced plosive          &  b & d & j & g &   &   \\
voiced aspirated plosive&  bʰ& dʰ& jʰ& gʰ&   &   \\
voiceless sibilant      &  s &   &   &   &   &   \\
nasal                   &  m & n & ñ & ŋ &   &   \\
lateral liquid          &  l &   &   &   &   &   \\
flap                    &  r &   &   &   &   &   \\
trill                   &    &   &   &   & ʀ &  \\
glides                  &  w &   &y  &   &   &   \\
\lspbottomrule
\end{tabular}}
%   \todo[inline]{check and possibly rearrange this table}
\end{table}

The voiceless obstruents require little discussion.  So far as their reflexes permit us to infer, they were unaspirated. Plosives *p, *t, and *k could occur in any position, but *Ɂ was contrastive only medially and finally, not morpheme-initially.

The voiced obstruents *b, *d, *g were the voiced equivalents of *p, *t, *k, except that *t was postdental, and other consonants in column 2 were alveolar.  In PNS, the palatal affricate *j lacked a voiceless counterpart, which either merged with *s before PNS came into being, or merged with *s recurrently after the split-up of PNS, leaving no trace of its former presence.

As noted in a number of previous publications (\citealt{Blust1969,Blust1974,Blust1993,Blust2005,Blust2006,Blust2016}, the segments written \textit{bʰ}, \textit{dʰ}, \textit{jʰ}, \textit{gʰ} meet \citegen[9]{Ladefoged1971} definition of true voiced aspirates (phonologically unitary segments that begin voiced and end voiceless, with optional delayed VOT on the following vowel), although throughout his career he continued to deny this for reasons that were never clear to me personally (e.g. \citealt[80]{LadefogedMaddieson1996}).

The only other segment that requires special comment is *R, which is reflected as /r/ or /l/ in many Austronesian languages, but as /g/ or /h/ in others (for a full discussion of the variety of reflexes of this rhotic see \citealt[595--596]{Blust2013}).

Appendix A provides minimal evidence supporting this reconstructed system in intervocalic position, which is the position that most concerns us in this chapter. Insufficient evidence is available for reconstructing PNS *-ñ- and *-r-, although at least *ñ is reconstructable as a word onset. *r is more problematic throughout Austronesian, but is supported as distinct from *R in a handful of forms in Kenyah languages. These gaps have no effect on the argument to follow.

\section{What happened to Berawan?}\label{sec:What-happened-Berewan}
  The four Berawan dialects/languages naturally share some phonological innovations apart from other North Sarawak languages, but they also each have individual peculiarities that set them apart from their subgroup-mates. The sound change I address here applies to all four Berawan speech communities, but in the interest of coherence, I will consider mainly the Batu Belah dialect (hereafter BBB), for which I recorded the largest number of relevant forms, with passing remarks on the others where I feel this might be helpful.

Since my concern is with the development of intervocalic *b in the Berawan languages, it will be well to start by observing the reflex of PNS *abu ‘ash’ in Appendix 1. In Kelabit this is \textit{abuh}, the only change being the historically secondary -\textit{h} that was added here and in a number of other forms. The Long Anap dialect of Kenyah lacks a cognate, although other Kenyah dialects have one (e.g. Long Atip \textit{avoɁ} ‘ashes, hearth’), and the Bintulu form is \textit{avəw}. In each of these languages, we see a readily recognizable sound change in which a voiced bilabial stop has been either retained or lenited to a labiodental fricative. But what happened to Berawan? BBB \textit{akkuh} sticks out like the proverbial sore thumb. Is it related at all? One’s first guess is “probably not”, but the only way to test decisions of cognation is by \textit{recurrence}, which is, and always has been, the key to determining cognation. This is a point that is often misunderstood by scholars in sister disciplines, such as cultural anthropology, and even by some linguists who have had little experience in dealing with historical questions. So, the next question must be: “What happened to intervocalic *b in other reconstructed forms?”

\cref{tab:Reflexes-PNS-b} lists all other BBB reflexes of PNS forms with medial *b for which I have data.

\begin{table}
\caption{\label{tab:Reflexes-PNS-b}Reflexes of PNS *-b- in Batu Belah Berawan}
\begin{tabular}{lll}
\lsptoprule
PNS     &   BBB    & \\
\midrule
*abu      & akkuh       & `ash'     \\
*babuy    &   bikuy     & `pig; wild boar'  \\
*bəlabaw  &   bəlilkiw  & `rat'  \\
*bubu     &  bukkuh     & `conical bamboo fish/eel trap'     \\
*bubuŋ    &   bukuŋ     & `ridgepole of house'   \\
*kabiŋ    &   kakiŋ     & `left side'   \\
*lubaŋ    &   lukiŋ     & `hole in the ground'   \\
*mabuk    & makuk       & `drunk'     \\
*nibuŋ    &   nikuŋ     & `nibong palm, \textit{Oncosperma} spp.'   \\
*Rabun    & gikuŋ       & `cloud'     \\
*Ribu     &  gikkuh     & `thousand'   \\
*tuba     &  tukkih     & `fish poison, \textit{Derris elliptica}'   \\
*ubi      & ukkih       & `yam'     \\
\lspbottomrule
\end{tabular}
\end{table}

All thirteen of these etymologies are completely straightforward: the reconstructions are well-established not only in PNS, but in higher-level proto-lan\-guages, as cognates that contain a medial voiced bilabial stop or some phonetically transparent lenition of it are found in scores or even hundreds of other languages, depending on the form. They are completely straightforward, and they are completely crazy -- how does a language change *b to /k/, and undergo this change only in intervocalic position?

The next step, then, is to show that PNS *b > Berawan \textit{-k}- was conditioned. This is already clear from the four *b-initial words in \cref{tab:Reflexes-PNS-b}, but to show that PNS developed along fundamentally different lines in word-initial, medial and final positions, a fuller set of data is given in \cref{tab:Reflexes-BBB-PNS-b}, leaving aside the forms already mentioned.

\begin{table}[b]
\caption{\label{tab:Reflexes-BBB-PNS-b}Reflexes of PNS *b- and *-b in Batu Belah Berawan}
\begin{tabular}{lll}
  \lsptoprule
    PNS  &     BBB     & \\
\midrule
*b- > \textit{b}      \\
\midrule
    *bahu   &   biɁoh   &    `stench, odor' \\
    *balu   &   billoh  &    `widow(er)'\\
    *baRa   &   bikkeh  &    `shoulder'\\
    *baRiw  &  bikiw    &    `wind'  \\
    *batu   &   bittoh  &    `stone'\\
    *batuk  &    bitok  &    `nape; neck'\\
    *bawaŋ  &  biwaŋ    &    `expanse of water; lake' \\
    *bədʰuk &   bəcuk   &    `monkey sp.' \\
    *bəkən  &    bəkən  &    `other, different' \\
    *bəRas  &    bəkiɁ  &    `husked rice' \\
    *buaya  &    bijjih &    `crocodile' \\
    *buku   &   bukkuh  &    `node, joint' \\
    *bulan  &    bulin  &    `moon' \\
    *bulu   &   bulluh  &    `body hair; feather' \\
\midrule
*-b > \textit{m} \\
\midrule
    *eleb   &   lu-ləm\footnote{Cf. Long Jegan, Long Teru \textit{ləm} ‘knee’. For the likely explanation of the first syllable in the BBB form, cf. Malay \textit{lutut} and similar forms for ‘knee’ in other languages < PMP *qulu tuhud ‘head of the knee’ (= knee cap).}    &    `knee'\\
            &   ŋ-uam   &     `to yawn\\
\lspbottomrule
\end{tabular}
\end{table}

Although only two examples of *b > -\textit{m} could be found in my fieldnotes, the nasalization of word-final voiced stops in Berawan is supported by more numerous examples of *d > -\textit{n/ŋ}:

\ea
PNS *alud > \textit{aloŋ} ‘boat’, *kuyad > \textit{kuyan} ‘gray langur’, *likud > \textit{likoŋ} ‘back (anat.)’, *pusəd > \textit{pusən} ‘navel’, *tumid > \textit{tumin} ‘heel’, and *uləd > \textit{ulən} ‘maggot, worm’, where word-final *d normally became \textit{ŋ} after rounded vowels, and \textit{n} elsewhere.
\z

One other thing to show is that PNS *p did not undergo labial backing in intervocalic position, which enables us to infer that this change in the data of \cref{tab:Reflexes-PNS-b} must have preceded intervocalic devoicing (hereafter IVD), since otherwise PNS *p and *b would have merged as /k/ intervocalically. This should be clear from \cref{tab:Reflexes-BBB-PNS-p}.

\begin{table}[H]
\caption{\label{tab:Reflexes-BBB-PNS-p}Reflexes of PNS *-p- in Batu Belah Berawan}
\begin{tabular}{lll}
\lsptoprule
PNS      &  BBB       \\
\midrule
*anipa    &    lippah&    \enquote*{snake sp.}    \\
*apuR     &   apon   &    \enquote*{lime (for betel)}    \\
*apuy     &   apoy   &    \enquote*{fire}    \\
*kapal    &    kapan &    \enquote*{thick (as a plank)}    \\
*lipen    &    dipan &    \enquote*{tooth}    \\
*lupi     &   luppeh &    \enquote*{dream}    \\
*sepaq    &    supa  &    \enquote*{betel quid}    \\
*tapan    &    tapan &    \enquote*{winnowing basket}    \\
\lspbottomrule
\end{tabular}
\end{table}

Several of these Batu Belah forms show irregularities in the development of a single vowel or single consonant, but all appear to be native, and together with data from other Berawan dialects they leave no question that PNS *p remained unchanged, in stark contrast to the development of intervocalic *b.

The last thing to mention in this section is that *b is not the only PNS phoneme that has undergone IVD. As noted following \cref{tab:PNS-Cs}, PNS *R (and its predecessor in earlier proto-languages back to Proto-Austronesian (\textsc{PAn})) apparently was either an alveolar trill that became uvular in many daughter languages before undergoing further change, or a uvular trill that became alveolar. Some of the best-known languages in the Austronesian family reflect it as /r/ (e.g., Malay).  Others reflect it as /g/ (e.g., Tagalog).  Still others reflect it as /h/ (Ngaju Dayak in southeast Borneo), zero (Javanese), /d/ (Inati/Inete), /l/ (Bunun), a voiceless lateral distinct from /l/ and /r/ (Thao), a retroflex flap (Saisiyat), /n/ (Mekeo), /s/, /x/ or /y/. Given the direction of front-back movement for trills in better-known languages, it seems likely that *R was an alveolar trill that was backed to uvular position in many daughter languages. Berawan evidently is one of the latter languages. Although it is the only language in the North Sarawak group to do so, it reflects *R as \textit{g} in initial position. Intervocalically it is usually reflected as \textit{k}, which is what interests us here, and word-finally it disappeared. Examples of these changes are shown in \cref{tab:Reflexes-BBB-PNS-R}.

\begin{table}
\caption{\label{tab:Reflexes-BBB-PNS-R}Reflexes of PNS *R in Batu Belah Berawan}
\begin{tabular}{lll}
\lsptoprule
      PNS &      BBB\\
\midrule
*R- > \textit{g}    \\
\midrule
*Rabun    &gikuŋ   &   \enquote*{cloud}                   \\
*Ramut    &gimok   &   \enquote*{root}                   \\
*Ratas    &  gitaɁ &   \enquote*{milk}                 \\
*Ratus    &  gitoh &   \enquote*{hundred}                 \\
*Ribu     & gikkuh &   \enquote*{thousand}                 \\
*Rusuk    &gusok   &   \enquote*{chest}                   \\
\midrule
*-R- > \textit{k}  &        &                              \\
\midrule
*aRəm     & akəm   &   \enquote*{pangolin}     \\
*baRa     & bikkeh &   \enquote*{shoulder}   \\
*baRiw    &bikiw   &   \enquote*{wind}     \\
*bəRas    &  bəkiɁ &   \enquote*{husked rice}   \\
*bəRat    &  pəkit &   \enquote*{heavy}   \\
*duRi     & dukkih &   \enquote*{thorn}   \\
*kaRaw    &kikiw   &   \enquote*{to scratch (an itch)}     \\
*paRa     & pakkeh &   \enquote*{storage rack}   \\
*suRat    &  sukit &   \enquote*{wound}   \\
*təgəRaŋ  &  takiŋ &   \enquote*{ribs}   \\
*təRəp    &  təkəp &   \enquote*{k.o. breadfruit}   \\
*uRat     & ukit   &   \enquote*{vein; tendon}     \\
\midrule
*-R > zero& &                              \\
\midrule
*alaR     & aka    &  \enquote*{vine, creeper}    \\
*ikuR     & iko    &  \enquote*{tail}    \\
*ipaR     & l-ipa  &  \enquote*{opposite bank or side}  \\
*tuduR    &  turo  &  \enquote*{to sleep}  \\
\lspbottomrule
\end{tabular}
\end{table}

Comparing the reflexes of PNS *b and *R in BBB then, we see wide divergence in initial and final position, but identity (and hence merger) word-medially, as shown in \cref{tab:BBB-b-R}.\pagebreak

\begin{table}
\caption{\label{tab:BBB-b-R}Reflexes of PNS *b and *R in Batu Belah Berawan}
\begin{tabular}{llll}
\lsptoprule
  PNS  &   &  BBB\\\cmidrule{2-4}
     &initial &   medial  &  final   \\\midrule
  *b &    b   & k         &     m    \\
  *R &    g   & k         &     ∅    \\
  \lspbottomrule
\end{tabular}
\end{table}

Since the simplest way to account for the difference between word-initial and medial reflexes of *R is to assume *R > \textit{g} as syllable onset, followed by intervocalic devoicing, it seems clear that intervocalic devoicing also accounts for *b > -\textit{k}- as a two-step change that began as *b > -\textit{g}-.

\section{Ockham and me}\label{sec:Ockham-Me}
The stage has now been set: PNS (and earlier) *b did not change word-initially, became a voiceless velar stop intervocalically, and became the homorganic nasal word-finally, the latter as part of a more general process in which voiced stop codas were nasalized as an alternative to final devoicing \citep{Blust2018}. Since PNS *p shows no change in intervocalic position, we can rule out the possibility that IVD preceded labial backing, and since PNS *R also shows IVD, the simplest way to account for this range of observations is that (1) *b backed to *g in intervocalic position, and (2) *g from both *b and *R devoiced intervocalically.

Where does this leave us as practitioners of the kind of science that is governed by Occam’s razor?\footnote{William of Ockham is usually cited as such, but his famous “razor” is more often called “Occam’s razor” (although “Ockham’s razor” also appears). I let the inconsistency stand here, as it makes me consistent with (at least) tens of thousands of references in the scientific literature.} Although I \textit{could} begin to speculate about possible intermediate steps that would allow these puzzling observations to be seen as outcomes of natural sound change, it is unnecessary, since the two assumptions made in the previous paragraph are sufficient to account for the observations. The only thing that might prevent us from stopping here is that the explanation is inconvenient for the theoretical assumption (and it is no more than that) that because \textit{most} sound change is phonetically motivated, \textit{all} sound change must be phonetically motivated.

This is where we return to the dinner table, fork and knife in hand.

\section{Inside the purist’s lab}\label{sec:Purists-Lab}
 There are two major publications that have dealt with some of the oddities of Berawan historical phonology since my data was collected in 1971. The first is \citet{Burkhardt2014}, a doctoral dissertation done at Goethe University in Frankfurt, Germany, in 2014. As its title suggests, it aims at a comprehensive account of the phonology of Proto-Berawan, and its development in the modern languages through a bottom-up reconstruction. Of the two publications that I cite here, it is the more data-oriented, less theory-focused of the two.  The other is \citet{Beguš2018}, a dissertation defended at Harvard University. It focuses on a wide crosslinguistic sample of problematic phonological phenomena, in each case seeking to find a
solution that is phonetically ``natural''. One of the changes that it addresses is *b > -\textit{k}- in Berawan. I will return to \citet{Burkhardt2014} shortly, but for the moment let me try to summarize the approach that \citet{Beguš2018} takes to the problem at hand.

\citet[122--130]{Beguš2018} proposes something he calls the “Blurring chain hypothesis” (BCH), which involves the following steps in order to get from *b to \textit{k} only in intervocalic position, entirely through phonetically-natural changes:

\ea %\todo{should we use arrows -b- $\to$  -β-?}
Step 1: The voiced stops *b/d/g developed voiced fricative allophones intervocalically, hence:\\
-\textit{b}-  $\to$  -\textit{β}-\\
-\textit{d}-  $\to$  -\textit{ð}-\\
-\textit{g}-  $\to$  -\textit{γ}-\\
\z

This is considered a natural change, since intervocalic lenition of voiced stops is common in the world’s languages.

\ea
Step 2: The non-coronal voiced fricatives devoiced.\\
-\textit{β}-  $\to$  -\textit{ϕ}-\\
-\textit{ð}-  $\to$  -\textit{r}-\\
-\textit{γ}-  $\to$  -\textit{x}-\\
\z

This step is justified by an abundant phonetics literature which states or implies that voiced fricatives are unstable, and hence show a strong tendency to devoice.

\ea
Step 3: Labial fricatives were backed to velars\\
-\textit{ɸ}-  $\to$  -\textit{x}-\\
-\textit{r}-  $\to$  -\textit{r}-\\
-\textit{x}-  $\to$  -\textit{x}-\\
\z

This step is based on the claim that the change from labial to velar position is more likely with fricatives than with stops.

\ea
Step 4: Fricatives return to stops.\\
-\textit{x}-  $\to$  -\textit{k}-\\
-\textit{r}-  $\to$  -\textit{r}-\\
-\textit{x}-  $\to$  -\textit{k}-\\
\z

Beguš sums up the BCH in the following formula, which is to be interpreted as: 
\begin{enumerate} 
\item[(1)] stops become fricatives intervocalically; 
\item[(2)] voiced fricatives devoice;
\item[(3)] (after backing) fricatives return to stops.
\end{enumerate}

\ea
D $\to$ Z / V\_V \\
Z $\to$ S / V\_V \\
S $\to$ T / V\_V \\
\z

He illustrates this with the following example (correcting errors in his reconstruction, and the phonemic representation of Berawan):

\ea
*babuy > *biβuy > *biϕuy > *bixuy > \textit{bikuy} ‘pig’
\z

By all accounts we should be happy -- we now have an explanation for a truly puzzling sound change that shows it to be the outcome of a series of intermediate steps, each of which purportedly can be motivated by reference to general phonological processes in human languages as a whole. Book closed?

\section{Reality strikes back}\label{sec:Reality-Strikes-Back}
 The first thing likely to trouble anyone who thinks seriously about the BCH is its violation of Occam’s razor.  We start with voiced stops that are visible from their reflexes in numerous languages outside the Berawan group (reflexes of \textsc{PAn} *qabu ‘ashes’, *babuy ‘pig’, etc.).  Then, in the history of Berawan, these segments go out of sight and become fricatives, only to re-emerge as voiceless stops with different place features in the daughter languages when they are visible again (\textit{akkuh, bikuy}, etc.). When a stop is reflected as a stop, with no direct evidence of any intermediate stage in which it was a fricative, standard scientific method would not support the claim that it became something different, and then reverted to its original state once it became possible to see it again. To scholars in many scientific disciplines, this would hardly be considered a sound scientific procedure. So, how is this claim justified?

First, to account for IVD, \citet[127]{Beguš2018} draws attention to the phonetics literature where it is commonly accepted that “Voicing in fricatives is highly dispreferred and articulatorily difficult to maintain … Because voiced fricatives at this stage surface only intervocalically, the result is an apparent intervocalic devoicing.” This provides a potential explanation for IVD, provided that an independent line of evidence supports the claim that stops became fricatives before becoming stops again. To date, no such independent line of evidence has been forthcoming.

Second, to account for the backing of labials to velars, Beguš refers to the study of consonant changes by \citet{Kümmel2007}, which focuses on Indo-European, Semitic and Uralic languages. In his sample of 294 languages, Kümmel found no cases of labial backing for stops, but he reportedly found two cases for fricatives. For reasons that many statisticians will surely find puzzling, \citet[128]{Beguš2018} uses this observation to state that “The sound change [ϕ] > [x] or [β] > [γ] (if it happened prior to devoicing) is \textit{much more common} than [p] > [k] or [b] > [g]” (italics added). In fact, two cases to none makes the backing of labial fricatives to their velar counterparts \textit{infinitely} more common than the similar change for stops, but what can this mean in such a tiny sample?

I maintain a close watch on a language family with over 1,200 members, and I have seen \textit{no} examples of labial fricatives backing to their velar counterparts anywhere in this family. If it happens in any language on the planet it must be very rare, and to claim that even two occurrences makes it “much more common” than the backing of labial stops to their velar counterparts is essentially meaningless. Of course, *f > \textit{h} is a common sound change, but that is part of the universal lenition sequence *p > \textit{f} > \textit{h} > zero in many of the world’s languages, and is irrelevant to this discussion.

To summarize, the first objection to the BCH is that it violates Occam’s razor by positing hypothetical intermediate stages in a sound change that are not needed to account for the facts.  Some might see this objection as more esthetic than substantive, although that is certainly debatable. More seriously, however, a central prediction of the BCH is contraindicated by the data, a fact that Beguš never mentions, and, I assume, was unaware of.  

As seen above, his steps 1 and 2 introduce a voiced bilabial fricative that then devoices before backing and returning to its original state as a stop. However, as \citet[166]{Burkhardt2014} makes clear, Proto-Berawan (PB) had a voiced bilabial fricative derived by glide fortition from automatic transitional glides, and this occurs in some very common words, as shown in \cref{tab:PB-Voiced-Fric} (LTB is Long Terawan Berawan; LJB is Long Jegan Berawan). 

\begin{table}[t]
\caption{\label{tab:PB-Voiced-Fric}Proto-Berawan voiced fricatives and their reflexes in the modern languages}
\begin{tabular}{llllll}
\lsptoprule
PNS  &  PB &   LTB  &  BBB &   LJB\\
\midrule
*bəRuaŋ  & *bəguβiŋ& kəbiŋ  &  kuβiŋ &   kuβiŋ   & `Malayan sun bear'  \\
*dua     & *duβa   &ləbih   & duβeh  &  duβyəy   & `two'               \\
*bituɁən  & *təku$^ə$n\footnote{For reasons that are unclear, \citet[166]{Burkhardt2014} has PB *kǝtuβǝn, based on a form of this shape only in BBB, as against \textit{təkuβən} in all other Berawan languages.  My own fieldnotes for BBB have \textit{təkuβən}, showing agreement in all four dialects, hence the reconstruction given here.} & təkəbin &   təkuβən & təkuβən & `star'\\
*kuay    & *kuβe   & kəbe  &  guβi  &  guβiæ  &  `Argus pheasant'\\
*puɁan    & *puβan & pəban &  puβan &  poβan  &  `squirrel'\\
\lspbottomrule
\end{tabular}
\end{table}

This is not a large number of forms, but it is sufficient to test the adequacy of steps 1 and 2 in the BCH. Moreover, the historical reality of automatic transitional glide fortition is further illustrated by a similar change for the palatal glide in forms such as PNS *ia, PB *jəh, LTB \textit{jəh}, BBB \textit{jah}, LJB \textit{jiæ} ‘3sg., s/he’, PNS *lia, PB *ləjəh, LTB \textit{ləjəh}, BBB \textit{ləjeh}, LJB \textit{ləjiæ} ‘ginger’, PNS *duRian, PB *dugəjin, LTB \textit{kəjin}, BBB \textit{kəjin}, LJB \textit{kəjin} ‘durian’, etc.\footnote{To avoid confusion, the reader should keep in mind that while the phonetic symbol [j] refers to a palatal glide in accordance with IPA conventions, the phonemic symbol /j/ refers to a voiced palatal affricate in accordance with conventions common to the spelling of languages throughout Indonesia and Malaysia.}

What matters here is that PB shows a voiced bilabial fricative identical to what Beguš posits in Step 1 of the BCH, as in PNS *lubaŋ > *lubiŋ (> hypothetical *luβiŋ > *luϕiŋ) > \textit{lukiŋ} ‘hole in the ground’. However, unlike the hypothetical fricative in the BCH, the empirically-grounded (i.e., “real”) fricative did not
1) devoice, 2) back to a velar, or 3) revert to a stop. Since all science is ultimately observation-based, there is only one scientifically-responsible way to explain this difference, namely that the hypothetical voiced bilabial fricative in the BCH \textit{did not exist}, since if it did, it should have remained unchanged like the examples in \cref{tab:PB-Voiced-Fric}. It is obvious that his conclusion is fatal to the BCH -- without an unobservable intermediate stage in which *-b- became a fricative before devoicing and reverting to a stop, the entire structure of Beguš’s theory collapses, and we are back to a theory that is more responsive to Occam’s razor.

This is the most serious problem with Beguš’s treatment of Berawan historical phonology.  However, it is not the only problem with his treatment of the history of this language.

\section{Post mortem}\label{sec:Post-Mortem}
 Before I say anything else, let me make it clear that I believe Gašper Beguš is a fine scholar.  His ‘Blurring chain hypothesis’ is an ingenious theoretical construct that required considerable skill and knowledge to propose. It has failed for one simple reason: its claims do not match the data of the real world. In many ways I feel that people like Beguš are victims of their foundational assumptions, in particular the commonly-held but rarely-expressed assumption that I have expressed earlier as: “Most sound changes are phonetically motivated; therefore \textit{all} sound changes are phonetically motivated.” What this belief commonly triggers is an argument chain that I would characterize as follows:

\ea
mindset   $>>$    freewheeling treatment of data    $>>$    careless treatment of data    $>>$ erroneous conclusions
\z

What I mean by this is that if one begins with an unshakeable mindset that all sound changes, no matter how phonetically challenging, \textit{must} be products of the telescoping of a larger number of smaller changes that are themselves natural, the temptation becomes irresistible to force the data by any means possible to conform to theoretical expectation. This leads to a freewheeling treatment of the primary data, as with Beguš’s willingness to assume that *b became a voiced bilabial fricative before undergoing further changes to emerge as /k/, hence going from \textsc{stop} to \textsc{fricative} to \textsc{stop}, with no direct evidence, or even indirect supporting evidence, that the intermediate stage ever existed as part of this sound change. The need for intervocalic fricatives rather than stops to provide a reason for labial backing is also based on the flimsiest of evidence (two cases to none). Yet, consistent with the freewheeling treatment of data, the reader is told that labial backing to velars is \textit{much more common} with fricatives than with stops, when simple statistical tests provide no support for such hyperbole. What I mean by ‘freewheeling’, then, is that the argument may still be anchored in an accurate factual basis, but the leap from observation to inference begins to take on the appearance of an elaborate contrivance -- there is pressure to find a way to show how an odd phonetic transition in the history of a language \textit{must} have been the product of smaller, phonetically more natural steps, so a full arsenal of speculative proposals is brought to bear on the question.

Once this habit of “reaching” for a way to explain away theoretically non-conforming data begins to gain momentum, it is hard to stop, and may easily lead to careless treatment of the data itself -- something we might call ‘data-boggling’. In the case at hand, although the BCH was Beguš’s own creation, and the responsbility for its failure therefore falls squarely on his shoulders, some other serious factual errors in his treatment of Berawan historical phonology are products of an over-reliance on \citet{Burkhardt2014}, when other sources going back to at least \citet{Blust1992} were available. In short, it appears that Beguš’s analysis fell victim to both of his foundational assumptions and his dependence on a primary source that itself contains serious flaws.

Let me begin with errors that are related to *b > -\textit{k-} as a historical change, but are not fatal to the BCH, and then mention others that are separate from this issue, but which involve serious misrepresentations of the data.

Keeping in mind that he cites Burkhardt’s PB *b and *g, rather than my PNS *b, *g and *R, \citet[123]{Beguš2018} says the following with reference to the development of voiced stops in word-initial position:

\begin{quote}
In contrast to intervocalic position, *b and *g remain unchanged in initial position. There are 46 reconstructed words with initial *b in Pre-Berawan. In all but one word the initial *b remains unchanged.  A similar distribution holds for the velar voiced stop in initial position … In the one exception, devoicing occurs initially in all four dialects: *bəlippiəŋ > \textit{pəlipiŋ} ([‘butterfly’, RAB]). According to \citet[144]{Burkhardt2014}, this development is sporadic in a word that already exhibits another sporadic development: degemination of -\textit{pp}-. There is only one other example in which devoicing initially occurs only in Long Terawan: *buraq > [purǎh] \citep{Burkhardt2014}.
\end{quote}

\largerpage
Unfortunately, it is demonstrably not true that *b and *g “remain unchanged in initial position” with only a single exception each. Although \textcite[150ff]{Burkhardt2014} notes that *b- is sometimes reflected as \textit{p}- if it was intervocalic as a result of prefixation, he does not point out that it is also reflected as \textit{k}- in the same languages under the same condition  (or consistently reflected as \textit{k-} in LTB), and Beguš simply failed to see examples such as those in \tabref{tab:Anomalous-PNS-Reflexes}.

\begin{table}[t]
\caption{\label{tab:Anomalous-PNS-Reflexes}Anomalous reflexes of PNS *b-, *g-, and *R- in three Berawan dialects. (My field data for Long Teru is too limited to permit useful generalizations, and so is omitted from this chapter.)}
\subfigure[LTB]{
\begin{tabular}{p{\widthof{*R > k- (2 instances)}}l@{~}lll}
\lsptoprule
*b- > \textit{k} (3 instances) & 1. &*bəsuR  &  kəco    &  \enquote*{full, satiated}  \\
                      & 2. &*buat   & kəbəiɁ   &  \enquote*{long} \\
                      & 3. &*buRuk  &  kuroɁ   &  \enquote*{rotten} \\
*g- > \textit{k} (1 instance)  & 4. &*gatəl  &  kitən   &  \enquote*{itch(y)} \\
*R- > \textit{k} (2 instances) & 5. &*Raqən  &  kiɁən   &  \enquote*{light (weight)} \\
                      & 6. &*Raya   & kijih    &  \enquote*{big}  \\
\lspbottomrule
\end{tabular}
}
\subfigure[BBB]{
\begin{tabular}[t]{p{\widthof{*R > k- (2 instances)}}l@{~}llll}
\lsptoprule
*b- > \textit{k} (2 instances)& 1.& *baRəq  &  kiki    &  \enquote*{swollen}     \\
                     & 2.& *buat  &  kuvit   &  \enquote*{heavy}    \\
*b- > \textit{p} (2 instances) & 3.& *beRat   & pəkit    &  \enquote*{long}     \\
                     & 4.& *buRuk  &  purok   &  \enquote*{rotten}    \\
*g- > \textit{k} (1 instance) & 5.& *gatəl  &  kitan   &  \enquote*{itch(y)}    \\
*R- > \textit{k} (2 instances)& 6.& *Raqən  &  kiɁan   &  \enquote*{light (weight)}    \\
                     & 7.& *Raya   & kijih    &  \enquote*{big}     \\
\lspbottomrule
\end{tabular}
}
\subfigure[LJB]{
\begin{tabular}[t]{p{\widthof{*R > k- (2 instances)}}l@{~}llll}
\lsptoprule
*b- > \textit{k} (2 instances)& 1.& *baRəq  &  kikeæ  &   `swollen'  \\
                     & 2.& *buat  &  kuvit  &   `heavy'  \\
*b- > \textit{p} (3 instances) & 3.& *bəsuR  &  pəco   &   `full, satiated'   \\
                     & 4.& *bəRat   & pəkit   &   `long'   \\
                     & 5.& *buRuk  &  puriuɁ &   `rotten' \\
*g- > \textit{k} (1 instance) & 6.& *gatəl  &  kætən  &   `itch(y)'  \\
*R- > \textit{k} (1 instance) & 7.& *Raqən  &  keɁan  &   `light (weight)'  \\
\lspbottomrule
\end{tabular}
}
\end{table}

This is a minority pattern, but it is sufficiently well-attested that it should not have been overlooked, as it includes data from the basic vocabulary (‘swollen’, ‘heavy', ‘long’, ‘rotten’).  What we would expect for the six LTB forms is \textit{b}- for the first three, and \textit{g}- for the last three, but instead we find \textit{k}- for all six. As seen already, this is the normal reflex of PNS *b and *R in intervocalic position, and so is a clue that each of these bases was intervocalic when labial backing and IVD occurred.

The next thing to notice is that all of these words in every dialect are stative or adjectival, while this is true of none of the words in \cref{tab:Reflexes-PNS-b}. Since PNS had an adjectival or stative verb prefix *mə- that is still common in Lun Bawang/Lun Dayeh (cf. \textit{mə-baraɁ ‘}swollen’, \textit{mə-bərat ‘}heavy’, \textit{mə-buruk} ‘rotten’, \textit{mə-gatəl} ‘itchy’\textit{, mə-raan ‘}light in weight’, etc.), we may assume that this prefix was still in place in the Berawan languages at the time of labial backing and intervocalic devoicing, and that after these changes took place, it was lost, leaving the bare stems with the normal reflexes of intervocalic *b, *g and *R in word-initial position. The same conclusion follows from the reflexes of trisyllabic nouns that regularly lost the first CV- after IVD had already taken place, as with the word for the ‘Malayan sun bear’  (PNS *bəRuaŋ, PB *bəguβiŋ) in \cref{tab:PB-Voiced-Fric}, and the word for ‘durian’ (PNS *duRian, PB *dugəjin) in the paragraph immediately after it, which show parallel examples of glide fortition at bilabial and palatal places of articulation and IVD before loss of the first syllable.\footnote{In both cases, the high vowel that triggered glide formation in the first place was centralized to schwa in LTB -- a complex sound change that is also found in other languages of coastal Sarawak. The fronting and raising of *a after a voiced obstruent is also a widespread change in northern Sarawak (\citealt{Blust2000,Blust2020}).}

The situation for Batu Belah and Long Jegan is somewhat more complicated. In both of these communities, some instances of *b- are reflected as \textit{k} and others as \textit{p}. It is important to keep the subgrouping of the Berawan languages in mind: the first split probably separated LTB from Southern Berawan, and we can see a clear difference in the pattern of anomalous reflexes of initial *b- in \cref{tab:Anomalous-PNS-Reflexes}, where LTB is consistent in reflecting what is now a word-initial reflex of *b- as \textit{k,} while Batu Belah and Long Jegan show variation between \textit{k} and \textit{p}. The most straightforward explanation for this data appears to be that labial backing preceded IVD in LTB, but that these two changes overlapped in Southern Berawan. The ordering of relevant changes, then, evidently was as follows:

\ea
Long Terawan Berawan:
\begin{enumerate}
\item[(1)] labial backing/V\_\_V;
\item[(2)] devoicing;
\item[(3)] loss of CV-. 
\end{enumerate}
\z

\ea
Batu Belah Berawan:
\begin{enumerate}
\item[(1)] labial backing/V\_\_V for items 1 and 3;
\item[(2)] devoicing;
\item[(3)] loss of CV-, but devoicing before labial backing for items 2 and 4 (which then could not back).
\end{enumerate}
\z

\ea
Long Jegan Berawan:
\begin{enumerate}
\item[(1)] labial backing/V\_\_V for items 1 and 4, but  devoicing before labial backing for items 2, 3, and 5 (which then could not back).
\end{enumerate} 
\z

What can we learn from this bit of neglected data? The firmest inference appears to be that IVD and labial backing overlapped in time. This may be of general interest, but has little relevance to the BCH, since the only ordering relation that this hypothesis requires is for *b to develop a fricative allophone intervocalically before IVD and labial backing, with the relative order of the latter two changes making no difference to the outcome of the historical derivation. However, this observation raises other questions, in particular, why does the change *-b- > \textit{p} occur across a morpheme boundary, but never within a base morpheme? It seems that labial backing had to precede IVD within a morpheme, but could occur before or after the other change across a morpheme boundary.

The second example of data-boggling in \citet{Beguš2018} that I will discuss concerns the history of the PNS voiced aspirates, a matter of some importance, since these were also voiced stops, although voiced stops with terminal devoicing. As noted in \citet{Blust2006}, PNS developed true voiced aspirates *bʰ, *dʰ, *jʰ, *gʰ, almost certainly from earlier geminates *bb, *dd, *jj, *gg, most of which arose after a non-moraic schwa. As early as \citet{Blust1969}, this was taken as the defining innovation for the North Sarawak subgroup. To show that this change was complete by PB, one need only refer to nearly a dozen papers that have addressed this issue over the past half century, or to Appendix A.  However, to provide a fuller account for the reader, the relevant data is summarized in \cref{tab:Reflexes-PNS-Voiced-Asp}.\footnote{Batu Belah Berawan represents all Berawan dialects since the developments do not differ by dialect. Note also that data for *gʰ is limited, and the reader is referred to Appendix A for the lone example thereof.}

\begin{table}[t]
\caption{\label{tab:Reflexes-PNS-Voiced-Asp}Reflexes of the PNS voiced aspirates in North Sarawak Languages}
  \subfigure[*bʰ]{
    \begin{tabularx}{.8\textwidth}{lXX}
    \lsptoprule
                      &  ‘head hair’ &   ‘sugarcane’\\
    \midrule
    PNS                & *əbʰuk  &    *təbʰuh  \\
    Bario Kelabit      & ~əbʰuk  &    ~təbʰuh  \\
    Long Anap Kenyah   & ~puk    &    ~təpu      \\
    Batu Belah Berawan & ~puk    &    ~təpuh   \\
    Bintulu            & ~ɓuk    &    ~təɓəw    \\
    \lspbottomrule
    \end{tabularx}
    }
  \subfigure[*dʰ]{
    \begin{tabularx}{.8\textwidth}{lXX}
    \lsptoprule
                       & ‘day’          &      ‘woman’  \\
    \midrule
    PNS                &  *ədʰaw       & *dədʰuR   \\
    Bario Kelabit      & ~ədʰo         &  ~dədʰur   \\
    Long Anap Kenyah   & ~taw          &  ~ləto     \\
    Batu Belah Berawan & ~iciw         &  ~dicu  \\
    Bintulu            & ~ɗaw          &  ~rəɗu     \\
    \lspbottomrule
    \end{tabularx}
    }
  \subfigure[*jʰ]{
    \begin{tabularx}{.8\textwidth}{lXl}
    \lsptoprule
                      & ‘one’       &      ‘notched log ladder’   \\
\midrule
    PNS                &  *əjʰa      &  *Rəjʰan    \\
    Bario Kelabit      &   ~ədʰəh    &  ~ədʰan  \\
    Long Anap Kenyah   &   ~ca       &  ~can       \\
    Batu Belah Berawan &   ~acih     &  ~acin      \\
    Bintulu            &  (jiɁəŋ)    &  ~k-əjan  \\
    \lspbottomrule
    \end{tabularx}
    }
\end{table}

Against this backdrop of information, which shows that pre-PNS voiced geminates had \textit{already} produced a distinctive series of true voiced aspirates in PNS, \citet[126, fn. 66]{Beguš2018} says:

\begin{quote}
Labial geminate stops arising after schwa and from consonant clusters do not undergo a change in place of articulation (unlike simple stops), e.g. *təbu > *təbbu > [təppu], *mə-bənnən > *mə-ppənnən > *ppənnən > [pənnən] (after the loss of *mə- and initial degemination) or *əbbis > *əppiq > [piɁ] (after the loss of initial schwa and initial degemination). Geminates arising via “h-accretion”, however, do undergo a change in place of articulation: they develop to voiceless velar geminate stops.
\end{quote}

One might respond: “Of course pre-PNS [bb] didn’t undergo labial backing, as it had already become /bʰ/ before the Berawan languages existed, and so was no longer a simple voiced stop, either singleton or geminate, when they began to differentiate.” This was hardly a secret since, as already noted, it has been addressed repeatedly in a literature now covering over half a century, beginning with \citet{Blust1969}.

\largerpage
On the other hand, \citet[188]{Burkhardt2014} misrepresents the history of the PNS voiced aspirates in Berawan in a different way, maintaining that “PBn did not inherit the geminates *bb, *dd or *gg as they had devoiced to *pp, *tt and *kk respectively at a Pre-PBn stage.” That this is false should be immediately obvious from the data in \cref{tab:Reflexes-PNS-Voiced-Asp}, which shows that PNS *dʰ and *jʰ merged as a new phoneme /c/ (voiceless palatal affricate) in the modern languages. In addition, the same data in \cref{tab:Reflexes-PNS-Voiced-Asp}, and other examples such as PNS *əbʰaɁ > LTB, BBB \textit{pi}, LJB \textit{piæ} ‘fresh water’ show Low Vowel Raising, which only happened after voiced obstruents, confirming that at least a three-way distinction in the voiced aspirates still existed in Proto-Berawan, namely *bʰ, *jʰ, *gʰ.

These details may seem arcane, but they are critical to understanding the phonological history of the entire North Sarawak group of languages, and why both Burkhardt and Beguš would totally ignore a fairly rich literature concerned with them in favor of personal speculations that fail to account for actual observations is a mystery.

The last example of data-boggling in \citet{Beguš2018} that I will discuss concerns the gemination of the onsets of open final syllables. As in many other AN languages of insular Southeast Asia, the schwa, which was extra-short (non-moraic) could not hold stress unless it geminated a following consonant, so allophonic gemination arose in Berawan following the reflex of an original penultimate schwa. This is of minor importance, as it remains allophonic unless one adheres to the ‘once a phoneme, always a phoneme’ principle, which I do not. More surprisingly, as noted in \citet{Blust1992,Blust1995}, the onsets of final syllables appear to have geminated if and only if they were open. At some point after this happened \textit{-h} was added after all final vowels, possibly an areal change, as it is also found in Lower Baram languages such as Kiput, Narum, and Miri, in several dialects of Penan (Lowland Kenyah), in Sebop, Long Wat Kenyah, various Melanau dialects, several Land Dayak languages, and throughout the Dayic languages. \citet[260]{Burkhardt2014} attempts to account for this peculiar condition as follows:

\begin{quote}
In PWMP [Proto-Western-Malayo-Polynesian] items with an open syllable, simple consonants became PBn geminate consonants after *-h had been added at a Pre-PBn stage. The study assumes that this caused the following chain reaction: The accretion of *-h caused the vowel nucleus of the ultima, which had been phonetically long by default at the end of the word, to become phonetically short. This then caused the normal phonetic length of the penult nucleus (which was short to medium-short) to become phonetically extra-short, which then led to the gemination of the consonant that followed it (that is the onset of the ultima) to make up for the extrashortness of the penult vowel.
\end{quote}

\citet[124]{Beguš2018} simply accepts this hypothesis without question, but it can confidently be dismissed without elaborate argumentation. First, it is based entirely on speculation, without actual measurements of vowel length. Second, as already noted, -\textit{h} addition is common to a number of languages of this area, and it triggered gemination nowhere else. Third, and most crucially, Burkhardt assumes that these geminates were found in PB and were added only after final \mbox{/h/} accretion, yet the latter change did not affect the Long Jegan dialect, which implies that /h/ accretion \textit{followed} the gemination of open final syllable onsets, as explained in \citet{Blust1992}, rather than triggering it, as shown in \cref{tab:Accretion-Berewan}.\footnote{The LJB reflex of PB *-u was recorded as -\textit{əw {\textasciitilde} -aw} in free variation.}

\begin{table}
\caption{\label{tab:Accretion-Berewan}-/h/ Accretion in Berawan dialects}
\begin{tabular}{lllll}
\lsptoprule
PB     &  LTB   &   BBB   & LJB     &                        \\
\midrule
*accu  &  accoh &   accoh &   accəw  & `dog'                   \\
*bullu &  bulluh&   bulluh&   bulləw & `body hair, feather'  \\
*dukki &  dukkih&   dukkih&   dukkəy & `thorn'               \\
*kuttu &  kuttoh&   kuttoh&   kottaw & `head louse'          \\
*matta &  mattəh&   mattah&   matta  & `eye'                  \\
*təllu &  təlloh&   təlloh&   təllaw & `three'               \\
*təppu &  təppuh&   təppuh&   təppəw & `sugarcane'           \\
*ullu  &  ulloh &   ulloh &   ollaw  & `head'                    \\
\lspbottomrule
\end{tabular}
\end{table}

\section{Alternatives to ``sound change''}\label{sec:Alt-to-sound-change}
Undoubtedly, the question foremost in the minds of phonologists who have stayed with me this long is: “O.K., if *b > -\textit{k-} really was *b > \textit{-g}- followed by intervocalic devoicing in Berawan, what phonological process could have motivated either of these changes?”~My answer is: “None.”


This leads to battle, as any phonologist worth his salt believes that sound change must involve some type of phonological process as typically understood in feature-based phonology.~However, there have long been indications that this is not true.~While the great majority of sound changes clearly \textit{are} phonetically or phonologically motivated, there is no logical principle that says: “\textit{Most} X are Y, therefore \textit{all} X are Y”.~Conscious manipulation of language for social reasons is well-known in phonology, lexical semantics and morphosyntax (\citealt{Conklin1956,Li1980,Li1982,Li2004,Hale1971,Blust1980,Thomason2007}). A particularly striking example is seen in the use of what Tagalog speakers call \textit{baliktád}, or ‘backward speech’.\footnote{It should be noted however that Tagalog \textit{baliktad} means ‘backwards’ in general, not just of speech.}


In a brief but dense description of this secret language, used mainly by teen\-ag\-ers at the time to disguise the content of messages from their elders, \citet{Conklin1956} lays out the workings of a system of deliberate language manipulation that makes English ‘Pig Latin’ look jejune by comparison.~He identifies eight types of structural rearrangement or affixation used to form \textit{baliktád}\textbf{ }words, including:

\begin{enumerate}
\item Complete reversal of the phonemic shape of the base (\textit{salá:mat > tamá:las} ‘thanks’).
\item Partial reversal of the phonemic shape of the base (\textit{dí:to > dó:ti} ‘here’).
\item Complete reversal of the syllable shape of the base (\textit{pá:ŋit > ŋitpá} ‘ugly’).
\item Partial reversal of the syllable shape of the base (\textit{ma-gandá > damagán} ‘beautiful’).
\item Insertion of the Actor Voice infix -\textit{um-} (bolded) according to the usual pattern in productive verb morphology (\textit{tiná:pay > t-}\textbf{\textit{um}}\textit{-iná:pay} ‘bread’, \textit{na > n-}\textbf{\textit{um}}\textit{-a} ‘already’).
\item Infixation of a separate -VC- infix after all syllable-initial consonants (\textit{sí:loɁ} > \textit{s}\textbf{\textit{-ig}}\textit{-í:-l-}\textbf{\textit{o:g}}\textit{-óɁ} ‘snare trap’ \textit{salá:mat póɁ} > \textit{s-}\textbf{\textit{ag}}\textit{-a:l-}\textbf{\textit{ag}}\textit{-á:m-}\textbf{\textit{ag}}\textit{-át p-}\textbf{\textit{og}}\textit{-óɁ} ‘Thank you, sir’).
\item Double infixation with -VC- followed by -VCVC-, a shape that does not occur in ordinary language (\textit{hindíɁ} > \textit{h-}\textbf{\textit{um-}}\textit{ind-}\textbf{\textit{imí:p}}\textit{-iɁ} ‘no, not’, \textit{puntá} > \textit{p-}\textbf{\textit{um}}\textit{-ú:nt-}\textbf{\textit{amá:p}}\textit{-á} ‘goes’).    
\item Complete reversal of the base and double infixation (\textit{hindíɁ} > \textit{d-}\textbf{\textit{im}}\textit{-í:h-}\textbf{\textit{in}}\textit{-ín} ‘no, not’, \textit{saɁán} > \textit{Ɂ}\textbf{\textit{um}}\textit{-a:ns-}\textbf{\textit{am}}\textit{-á ‘where?’}).
\end{enumerate}

A similar system, called \textit{cakap balek} (which, like \textit{baliktád} means ‘backward speech’) was recorded very briefly for Malay by \citet{Evans1923}, suggesting that such systems of speech disguise may be more general in insular Southeast Asia than is commonly appreciated.

~

Given the purpose of \textit{baliktád}, it is clear that it cannot remain constant over time, or its function would be lost (at least for those adults who might recall their own earlier use of the system).~One can legitimately object that this type of word-play could never give rise to a permanent sound change, given both its specialized function, and its inherent lability.~However, what is \textit{does} show is that speakers are capable of creating deliberate changes in their language for special purposes, and the suggestion that conscious manipulation might lie behind a sound change like *b > -g- or intervocalic devoicing is not inherently unlikely.

\section{Conclusion}\label{sec:Conclusion}
It should be obvious that the issue of the obligatory naturalness of sound change is one that is going to divide the community of linguists into opposing camps, probably as much as any other issue in the field. Until studies of language change in progress are able to capture an example of a sound change that is both abrupt and phonetically “unnatural”, we will have no way to determine by direct observation whether a change like *b > -\textit{g}- in Berawan has actually occurred in any language. My own guess is that there have been many such changes, and my recommendation to all scholars, regardless of which side they take in this debate, is that they respect the integrity of the scientific process by adhering to Occam’s razor, and limiting questionable inferences by demanding converging lines of independent evidence, rather than resorting to freewheeling speculation because they “know” in advance that their position is correct, when in fact its correctness is exactly what is at issue.


\begin{paperappendix}
\section{Sample evidence supporting the PNS consonants in intervocalic position in Bario Kelabit, Long Anap Kenyah (LA Kenyah), Batu Belah Berawan (BB Berawan), and Bintulu}

\begin{tabularx}{\textwidth}{lXXXl}
\lsptoprule
PNS            & *apuy   & *batu   & *ikuR  &  *daɁun       \\
               & ‘fire’    & ‘stone’    & ‘tail’   & ‘leaf’          \\
\midrule
Bario Kelabit  &  apuy   &  batuh  &  iur   & daɁun         \\
LA Kenyah      & (lutən) &   batu  &  iko   & (tuŋ kayu)    \\
BB Berawan     & apoy    & bitoh   &  iko   & dioŋ          \\
Bintulu        & (jarəɁ) &   batəw &   ikoy &   raɁun       \\
\lspbottomrule
\end{tabularx}

\begin{tabularx}{\textwidth}{lXXXl}
\lsptoprule
PNS           & *abu  &  *ŋadan &   *ujan &  *təgəRaŋ \\
              & ‘ash’   &   ‘name’  &  ‘rain’   &‘ribcage’    \\
\midrule
Bario Kelabit &  abuh &   ŋadan &   udan  & ---     \\
LA Kenyah     &(lisəŋ)&    ŋadan&    ujan &  təgaaŋ   \\
BB Berawan    &  akuh &   ŋaran &   usin  & takiŋ     \\
Bintulu       &avəw   & ñaran   & ujan    & ---     \\
\lspbottomrule
\end{tabularx}

\begin{tabularx}{\textwidth}{lXXXl}
\lsptoprule
PNS            &  *əbʰuk   & *ədʰaw&    *əjʰa  &  *məgʰəl                                                         \\
               &  ‘head hair’&  ‘day’  &      ‘one’  &  ‘sleep’     \\
\midrule
Bario Kelabit  &  əbʰuk    &ədʰo   & ədʰəh     & məgʰəl\footnote{‘stay with a small child to make him sleep’.}    \\
LA Kenyah      &   puk     &taw     & ca        & məkən\footnote{‘to rest, to lie down’.}                          \\
BB Berawan     &   puk     &iciw    & acih      &(turo)                                                            \\
Bintulu        &   ɓuk     &ɗaw     &(jiɁəŋ)    & məgən\\
\lspbottomrule
\end{tabularx}

\begin{tabularx}{\textwidth}{lXXXl}
\lsptoprule
PNS              & *kuman  &*tanəɁ        &   *-ñ-\footnote{There are no good candidates for PNS *ñ in medial position.} &   *taŋih  \\
                 &  ‘eat’    &‘earth, soil’   &&    ‘to weep, cry’                                                                         \\
\midrule
Bario Kelabit    &  kuman  &  tanaɁ       &&  taŋe                                                                                   \\
LA Kenyah        &  uman   & tanaɁ        && taŋe                                                                                    \\
BB Berawan       &  kuman  &  tana        && taŋeɁ                                                                                   \\
Bintulu          &  kuman  &  tanəɁ       &&  (məŋit)\\
\lspbottomrule
\end{tabularx}

\begin{tabularx}{\textwidth}{lXXXl}
\lsptoprule
PNS              &   *asu  &  *təlu   & *-r-\footnote{There are no good candidates for PNS *r in medial position.}  &  *bəRat    \\
                 &   ‘dog’   &   ‘three’  &&      ‘heavy’                                                                              \\
\midrule
Bario Kelabit    &   (ukuɁ)&    təluh &&      bərat                                                                             \\
LA Kenyah        &   asu   & təlu     &&  baat                                                                                  \\
BB Berawan       &  acoh   & təloh    &&   pəkit                                                                                \\
Bintulu          &  asəw   & ləw      && vat\\
\lspbottomrule
\end{tabularx}

\begin{tabularx}{\textwidth}{lXXXl}
\lsptoprule
PNS              &  *tawa  & *kayu         \\
                 & ‘to laugh’& ‘wood, tree’    \\
\midrule
Bario Kelabit    & (riruh) &  kayuh        \\
LA Kenyah        &  pə-tawa& kayu          \\
BB Berawan       &  tavah  & kajuh         \\
Bintulu          &bə-taba  & kayəw         \\
\lspbottomrule
\end{tabularx}
\end{paperappendix}

\printbibliography[heading=subbibliography,notkeyword=this]

\end{document}
