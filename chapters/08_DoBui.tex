\documentclass[output=paper]{langscibook}
\ChapterDOI{10.5281/zenodo.15148176}
\author{Bien Dobui\orcid{}\affiliation{Université de Picardie Jules Verne, CERCLL-PraLing (UR UPJV 4283) and SeDyl (UMR 8202 CNRS)}}
\title{A model of non-modal phonation: Ballisticity in Otomanguean languages}
\abstract{Ballistic versus controlled syllables have long been posited in Amuzgo and Chinantec languages (Otomanguean). They have been treated as a feature specific only to these languages. Given phonetic and articulatory correlates of ballistic syllables, more recent studies have instead characterized the trait as laryngeal in nature and better understood as a realization of [spread glottis]. In this study, the trait is understood in relation to the distribution of other laryngeal contrasts, and in the context of morphophonological patterns showing that [spread glottis] is also morphemic. A [spread glottis] autosegment is proposed. This autosegment forms part of a three-way phonation contrast (spread glottis, constricted glottis, and modal voice) observed in these languages. A review of available data in light of \citegen{Hyman2006} parameter-based approach to prosodic typology shows that previous assertions that ‘ballisticity’ is a form of stress or a separate tone system should be abandoned.
\keywords{Amuzgo, Chinantec, ballisticity, phonation, prosody}}
\IfFileExists{../localcommands.tex}{
  \addbibresource{../localbibliography.bib}
  \usepackage{tabularx, multicol, multirow, longtable}
\usepackage{url}
\urlstyle{same}

\usepackage{orcidlink}
\definecolor{orcidlogocol}{cmyk}{0,0,0,1}
\RenewDocumentCommand{\LinkToORCIDinAffiliations}{ +m }
  {%
    \orcidlink{#1}\,%
  }
\SetupAffiliations{orcid placement=before}

\usepackage{siunitx}
\sisetup{detect-weight=true, detect-family=true, group-digits=none}

\usepackage{mathtools}
\usepackage{langsci-optional}
\usepackage{langsci-lgr}
\usepackage{langsci-gb4e}

\usepackage{stmaryrd}
\usepackage[capitalize]{cleveref}
\babelfont[macedonian]{rm}[Language=Macedonian,ItalicFont=LibertinusSerif-Italic.otf]{LibertinusSerif-Regular.otf}
\usepackage{eqparbox}
\usepackage[autostyle]{csquotes}
\usepackage[linguistics]{forest}

\usetikzlibrary{positioning, matrix}
\usepackage[glosses,inline]{leipzig}
\PassOptionsToPackage{xindy,toc,nopostdot}{glossaries}
\usepackage{glossary-inline}
\setglossarystyle{inline}
\makeglossaries

\usepackage{phonrule}
\usepackage{threeparttable}


\usepackage{textcomp,gensymb}


\usepackage[preservefont]{tipauni}

\usepackage[normalem]{ulem}

\usepackage{enumitem} %so lists aren't ugly
	
\usepackage{threeparttable}	%allows tables with tablenotes. note marks: †‡
	\makeatletter 
	\g@addto@macro\TPT@defaults{\footnotesize} 
	\makeatother

\usepackage{colortbl}
	\definecolor{Pink}{rgb}{0.96, 0.76, 0.76} 
	\definecolor{PaleBlue}{rgb}{0.67, 0.9, 0.93}
	\definecolor{carolinablue}{rgb}{0.6, 0.73, 0.89}
	\definecolor{goldenyellow}{rgb}{1.0, 0.87, 0.0}
	\definecolor{Orange}{rgb}{1.0, 0.66, 0.07}
	\definecolor{puce}{rgb}{0.8, 0.53, 0.6}
	\definecolor{turquoisegreen}{rgb}{0.63, 0.84, 0.71}


% add all extra packages you need to load to this file
\usepackage{langsci-textipa}
\usepackage{vowel}
\usepackage{textgreek}

% \usepackage{langsci-branding}
% \usepackage{subcaption}
\usepackage{subfigure}

\usepackage{tabto}


\usetikzlibrary{tikzmark}
\usepackage{pgfplots}


\newfontfamily\tibetan{NotoSerifTibetan-Regular.ttf}
\usepackage{langsci-branding}
\usepackage{hyphenat}

\usepackage{accents}

  \renewcommand{\lsChapterFooterSize}{\footnotesize}

\makeatletter
\let\thetitle\@title
\let\theauthor\@author
\makeatother

\newcommand{\togglepaper}[1][0]{
   \bibliography{../localbibliography}
   \papernote{\scriptsize\normalfont
     \theauthor.
     \titleTemp.
     To appear in:
     Natalia Kuznetsova, Cormac Anderson \& Shelece Easterday (ed.).
     Rarities in phonetics and phonology.tex.
     Berlin: Language Science Press. [preliminary page numbering]
   }
   \pagenumbering{roman}
   \setcounter{chapter}{#1}
   \addtocounter{chapter}{-1}
}

\newbool{bookcompile}
\booltrue{bookcompile}
\newcommand{\bookorchapter}[2]{\ifbool{bookcompile}{#1}{#2}}

\newcommand{\textarab}[1]{\RL{\arabicfont #1}}

\newcommand\mb[1]{\eqparbox[t]{examples}{#1}\hspace{1em}}
\newcommand\mbi[1]{\mb{#1}}
\newcommand{\twe}[3]{\mbi{#1}\eqparbox[t]{orths}{\emph{#2}}\hspace{1em}`#3'\hspace{1em}} % three-way example
\providecommand\glottocode[1]{[\href{https://glottolog.org/resource/languoid/id/#1}{#1}]}
\newcommand{\phonreal}[1]{\ensuremath{\llbracket}#1\ensuremath{\rrbracket}}

\DeclareRobustCommand\dash{\unskip\nobreak\thinspace\textendash\allowbreak\thinspace\ignorespaces}

\forestset{minus/.style={edge label={node[midway, left] {\ensuremath{-}\hspace*{2mm}}}},
plus/.style={edge label={node[midway, right] {\hspace*{2mm}\ensuremath{+}}}}}
\providecommand\ipa[1]{#1}


\newcommand{\tone}[1]{\textsuperscript{#1}}

\newcommand{\orthog}[1]{\textit{#1}}
\newcommand{\gloss}[1]{`#1'}

\newcommand{\glottolog}[1]{\texttt{\href{https://glottolog.org/resource/languoid/id/#1}{#1}}}

\newcolumntype{O}{>{\itshape }l<{}}
\newcolumntype{G}{>{`}l<{'}}

\newcounter{tabsubcounter}
\newcommand{\tablecounter}{\setcounter{tabsubcounter}{0}}
\newcommand{\TC}{\stepcounter{tabsubcounter}\alph{tabsubcounter}.}

\usetikzlibrary{chains,positioning,calc,decorations.markings}
\tikzset{
	seg/.style={text height=0.6em, text depth=0.3em},
	moraic-structure/.style={xscale=0.6,yscale=1.1, text height=0.65em,text depth=0.25em},
 }

%05_Culhane_Edwards
%%%%%%%%%%%%%%%%%%%%%%%%%%%%%%%%
%%	Symbols and Characters  	%%
%%%%%%%%%%%%%%%%%%%%%%%%%%%%%%%% αβσµ

\newcommand{\tl}{\char`~}						%middle tilde ~
\renewcommand{\Q}{\textquotesingle}		%straight apostrophe444
\newcommand{\ra}{→} 								%right arrow ->
\newcommand{\0}{∅} 									%zero symbol
\newcommand{\gap}{\textunderscore} 	%underscore
%\renewcommand{\j}{ʤ}								%dezh digraph
\newcommand{\syll}{σ}								%lowercase sigma medial form
\newcommand{\wrd}{ω}								%lowercase omega
\newcommand{\ft}{φ}									%lowercase phi
\newcommand{\gw}{gʷ}								%g with superscript w
\newcommand{\B}{β}									%voiced bilabial fricative
\newcommand{\hp}{\hphantom}					%space equal to width of argument
\newcommand{\it}{\textit}	%italics

%%%%%%%%%%%%%%%%%%%%%%%%%%%%%%%%
%%	Font Styles & Formatting	%%
%%%%%%%%%%%%%%%%%%%%%%%%%%%%%%%%

\definecolor{DarkBlue}{RGB}{0,0,130}										%dark blue colour
% \newcommand{\ve}[1]{\textcolor{DarkBlue}{\textit{#1}}}	%vernacular text
\newcommand{\ve}[1]{{\textit{#1}}}	%vernacular text
\definecolor{DarkRed}{RGB}{150,0,0}											%dark red colour
% \newcommand{\tbr}[1]{\textcolor{DarkRed}{\textbf{#1}}}	%Bold red text
\newcommand{\tbr}[1]{{\textbf{#1}}}	%Bold red text
%\renewcommand{\it}{\textit}																%italics
\newcommand{\tsc}{\textsc}															%small caps
\newcommand{\sub}{\textsubscript}												%subscript
\newcommand{\su}{\textsuperscript}											%superscript

%%%%%%%%%%%%%%%%%%%%%%%%%%%%%%%%%%%%%%%%%%%%%%%%%%%%
%% Tables %% Tables %% Tables %% Tables %% Tables %%
%%%%%%%%%%%%%%%%%%%%%%%%%%%%%%%%%%%%%%%%%%%%%%%%%%%%

% \newcommand{\mc}{\multicolumn}									%multicolumn
% \newcommand{\st}[1]{\setlength{\tabcolsep}{#1}}	%reduce column width in tables
%
%%%%%%%%%%%%%%%%%%%%%%%%%%%%%%%%
%%    Cross   References      %%
%%%%%%%%%%%%%%%%%%%%%%%%%%%%%%%%

% \def\Plus{\texttt{+}}
% \def\Minus{\texttt{-}}
% \newcommand{\GS}{ʔ}
% \def\SH{ʃ}
% \newcommand{\TSH}{ʧ}
% \def\ZH{ʒ}
% \def\DZH{ʤ}
% \def\:{ː}
% \def\UP{\textsuperscript}
% \def\rs{ʂ}
% \newcommand{\rn}{ɳ}
% \def\rt{ʈ}
% \def\tllr{ɺ}
% \newcommand{\Bb}{β}
% \def\Eps{ɛ}
% \def\Oo{ɔ}
% \def\Gm{ɣ}
% \def\NG{ŋ}
% \def\barU{ʉ}
\newcommand{\CM}{\ding{51}}
\newcommand{\XM}{\ding{53}}
% \newcommand{\tap}{ɾ}
% \def\darkL{ɫ}
% \def\schwa{ə}
%
% \def\BUL{\textbullet}


%%%%%%%%%%%%%%
%					%
%	Secondaries		%
%					%
%%%%%%%%%%%%%%
%	Post
\newcommand{\Post}[2]{#1\textsuperscript{#2}}
%	Pre
\newcommand{\Pre} [2] {\textsuperscript{#1}#2}
%	Undertilde
\newcommand{\utilde}[1]{\ensuremath{\smash{\underset{\mathclap{\sim}}{\text{#1}}}}}
%	Devoiced
% \newcommand{\VCLS}[1]{\textsubring{#1}}
%%%%%%%%%%%
%				%
%	Definitions		%
%	Markup		%
%				%
%%%%%%%%%%%
% \def\->{$\rightarrow$}
% \def\__{\underline{\hspace{1em}}}
\def\NoPoss{\cellcolor{gray!30}}

\newcommand{\VOICELESS}{\textsc{voiceless}}
\newcommand{\VOICED}{\textsc{voiced}}
\newcommand{\tablenote}[2][1]{\parbox{#1\textwidth}{\footnotesize\raggedright #2}}

\newcommand{\appref}[1]{Appendix~\ref{#1}}
\renewcommand{\sectref}[1]{Section~\ref{#1}}


\newcommand{\dobuibox}[5]{#1\\[-1.1em]
\hspace*{-.8cm}
 \begin{tabularx}{.9\textwidth}{@{}lQQ@{}}
       &  {oral} &  {nasal} \\
       \midrule
     {controlled} &\parbox[t]{4cm}{\raggedright  #2} & \parbox[t]{4cm}{\raggedright #3} \\
     \tablevspace
     {ballistic} &\parbox[t]{4cm}{\raggedright  #4} & \parbox[t]{4cm}{\raggedright  #5} \\
 \end{tabularx}
}

\newfontfamily\VdottildeFont{LibertinusVdottilde.otf}

\newcommand{\Vdottilde}{{\VdottildeFont V̰̣}}

% \renewcommand{\keywords}[1]{\textbf{#1}}
 
  %% hyphenation points for line breaks
%% Normally, automatic hyphenation in LaTeX is very good
%% If a word is mis-hyphenated, add it to this file
%%
%% add information to TeX file before \begin{document} with:
%% %% hyphenation points for line breaks
%% Normally, automatic hyphenation in LaTeX is very good
%% If a word is mis-hyphenated, add it to this file
%%
%% add information to TeX file before \begin{document} with:
%% %% hyphenation points for line breaks
%% Normally, automatic hyphenation in LaTeX is very good
%% If a word is mis-hyphenated, add it to this file
%%
%% add information to TeX file before \begin{document} with:
%% \include{localhyphenation}
\hyphenation{
    af-fri-cates
    al-ve-o-pal-a-tal
    Ama-nu-ban
    Ara-wak-an
    Árna-son
    Ber-ber
    can-di-dates
    Cam-er-oon
    Chi-nan-tec
    Chir-ko-va
    Crai-o-ve-a-nu
    di-chot-o-my
    Ec-ua-do-rian
    Ec-ua-dor
    elec-tro-glot-to-gra-phy
    Faro-ese
    Ike-ma
    Kuznet-sova
    Mes-kwa-ki
    Mio-ma-fo
    mono-mor-aic
    Ne-ca-xa
    Oto-man-gue-an
    par-a-digm
    post-as-pi-rat-ed
    post-as-pi-ra-tion
    pre-as-pi-rat-ed
    pre-as-pi-ra-tion
    pros-o-dic
    pros-o-dies
    re-con-struc-table
    Sheh-ret
    Svan-tes-son
    Ta-ras-can
    Tórs-havn
    Ural-ic
    epen-the-sis
    Anin-dil-yak-wa
    Mi-nyag
    Na-ka-ma
}

\hyphenation{
    af-fri-cates
    al-ve-o-pal-a-tal
    Ama-nu-ban
    Ara-wak-an
    Árna-son
    Ber-ber
    can-di-dates
    Cam-er-oon
    Chi-nan-tec
    Chir-ko-va
    Crai-o-ve-a-nu
    di-chot-o-my
    Ec-ua-do-rian
    Ec-ua-dor
    elec-tro-glot-to-gra-phy
    Faro-ese
    Ike-ma
    Kuznet-sova
    Mes-kwa-ki
    Mio-ma-fo
    mono-mor-aic
    Ne-ca-xa
    Oto-man-gue-an
    par-a-digm
    post-as-pi-rat-ed
    post-as-pi-ra-tion
    pre-as-pi-rat-ed
    pre-as-pi-ra-tion
    pros-o-dic
    pros-o-dies
    re-con-struc-table
    Sheh-ret
    Svan-tes-son
    Ta-ras-can
    Tórs-havn
    Ural-ic
    epen-the-sis
    Anin-dil-yak-wa
    Mi-nyag
    Na-ka-ma
}

\hyphenation{
    af-fri-cates
    al-ve-o-pal-a-tal
    Ama-nu-ban
    Ara-wak-an
    Árna-son
    Ber-ber
    can-di-dates
    Cam-er-oon
    Chi-nan-tec
    Chir-ko-va
    Crai-o-ve-a-nu
    di-chot-o-my
    Ec-ua-do-rian
    Ec-ua-dor
    elec-tro-glot-to-gra-phy
    Faro-ese
    Ike-ma
    Kuznet-sova
    Mes-kwa-ki
    Mio-ma-fo
    mono-mor-aic
    Ne-ca-xa
    Oto-man-gue-an
    par-a-digm
    post-as-pi-rat-ed
    post-as-pi-ra-tion
    pre-as-pi-rat-ed
    pre-as-pi-ra-tion
    pros-o-dic
    pros-o-dies
    re-con-struc-table
    Sheh-ret
    Svan-tes-son
    Ta-ras-can
    Tórs-havn
    Ural-ic
    epen-the-sis
    Anin-dil-yak-wa
    Mi-nyag
    Na-ka-ma
}
 
  \togglepaper[8]%%chapternumber
}{}

\begin{document}
\maketitle 
%\shorttitlerunninghead{}%%use this for an abridged title in the page headers


\section{Introduction}\label{sec:dobui:1}

A syllabic level contrast commonly referred to as ``ballistic'' vs ``controlled'' has been observed in a number of Otomanguean languages, namely in the varieties of two language groups: Amuzgo (ISO 639-3, Glottocode amuz1254; hereafter Am) and Chinantec (ISO 639-3, Glottocode chin1484; hereafter Ch). The term encompasses a number of stress-like acoustic and articulatory markers that can co-occur with every other lexical prosodic trait (contrastive tone, nasalization, laryngealization, also vowel length in the case of Ch). While ballisticity has been largely described instrumentally, phonological accounts are incomplete as they typically lack insight from other parts of the grammar.

The argument presented in this chapter takes into account language-specific syllable economy and morphological phenomena. Both language groups allow only a limited number of monosyllabic structural frames coupled with high prosodic complexity and a large number of morphological combinatory possibilities. All this results in complex morphophonological alternations of these prosodic inventories. Data presented from Am and Ch show that ballisticity is involved in morphological marking, with widest functionality in Am. Ballisticity contrastively marks both functional and lexical morphemes, and is morphologically motivated across different inflectional paradigms and, to a lesser degree, in word formation. In the Ch languages where the feature is attested, ballisticity is mostly restricted to lexical morphemes, though in some it also marks lexical verb classes and verbal agreement.

This chapter follows \citet{Silverman1994} and \citet{Zendejas2000} in their initial characterization of ballisticity as a phonological feature [spread glottis] (hereafter [sp]) and further discusses the following arguments. First, ballisticity is a phonologically independent phenomenon: it is not dependent on other features as it is cross-distributed with every other lexical prosodic feature. It is contrastive, unpredictable, and found in lexical minimal pairs. Second, the feature is morphologically bound: it marks certain morphological contrasts separately or in combination with other prosodic contrasts (tone, nasalization, vowel length in Ch). Third, coherency within prosodic typology is addressed using \citegen{Hyman2006} prototypical treatment of stress versus tone. This stricter definition of stress and tone allows us to dismiss ballisticity as an archetypal form of both categories while understanding how it is exploited in different languages. And finally, ballisticity is put into perspective with other laryngeal features in the discussed languages. The feature [sp] can be understood as entering the existing paradigm of laryngeal features to form a fully exploited set. In other words, phonation fully contrasts three types: modal voice, [constricted glottis], and [spread glottis], and ballisticity or [sp] is but one realization of this contrast. In order to theoretically interpret multiple and sometimes conflicting laryngeal features realized in one syllable, the laryngeal phenomena are interpreted as having different phonological statuses. In particular [sp] is a syllable-level autosegment rather than a segmentally attached feature as proposed in \citet{Silverman1994} and  \citet{Zendejas2000}.  

Data used in this study are from first- and second-hand sources. Two different first-hand corpora of the Amuzgo variety spoken in Xochistlahuaca (hereafter AMU) are used. One corpus was recorded in 2010–2013 with JS, a male speaker born in the 1970s, by the author and other members of the Endangered Language Alliance (hereafter ELA), a non-profit documenting language diversity in New York City. This corpus constitutes about 20 hours of elicited paradigms and prompted speech using stimuli like the \textit{Picture series on topological relations} by \citet{BowermanPederson1992}. The second corpus of two hours was recorded by the author in 2016, 2017 and 2021 with JP, a male speaker born in 1985. These corpora will be referred to using the initials of the speakers and the year of collection. Second-hand sources for the Amuzgo variety from San Pedro Amuzgos (SPA) and the different varieties of Chinantec are noted in full citation. The Chinantec corpus was compiled by \citet{Rensch1968} and consists of Rensch’s own fieldwork and the previous work of other linguists from the Summer Institute of Linguistics. The corpus was digitized in 2012 for the Meso-American MorphoPhonology (MAmP) project funded by the Institut Universitaire de France from 2009-2014.\footnote{Cf. http://jll.smallcodes.com/home.page}

The study is organized as follows. In \sectref{sec:dobui:2}, the feature known as ballisticity is described in detail. I discuss both how it manifests in Am and Ch and in other languages of the world and how it has been treated in the literature. Then, \sectref{sec:dobui:3} addresses the prosodic and morphological status of ballisticity in the Am and Ch languages for which the feature has been phonetically studied. In \sectref{sec:dobui:4}, different possible statuses of the feature are considered from a typological perspective, followed by a discussion of other laryngeal features in Am and Ch. At the end of \sectref{sec:dobui:4}, all the laryngeal features are briefly considered in an effort to model this contrastive system. Some remaining questions are also touched on. Finally, in \sectref{sec:dobui:5}, a short conclusion closes the study.

\section{What is ballisticity?}\label{sec:dobui:2}
\label{bkm:Ref102631913}
\subsection{Introduction}
Ballistic syllables have been described in Otomanguean studies by multiple authors (\citealt{Skinner1962}; \citealt{Merrifield1963,Merrifield1968}; \citealt{Bauernschmidt1965}; \citealt{Westley1971}; \citealt{Foris1973}, \citealt{Mugele1982}). Ballisticity has been characterized as a syllable level stress-like feature where an initial rapid surge over the nucleus then rapidly decays, followed by devoicing and a breathy release. Other observations include higher energy, fortis onset consonants, shorter duration and diverging tonal realizations, as compared to ``controlled'' syllables. All these features are also listed in \citet{DiCanioBennett2020}. Across the different Otomanguean languages in which ballisticity has been attested, descriptive accounts vary only slightly, mostly in relation to the distribution of ballistic and controlled syllables rather than to their phonetic characteristics.

Phonetic correlates of ballisticity have been analyzed for the Ch languages of Lalana \citep{Mugele1982}, Palantla (\citealt{MerrifieldEdmondson1999}) and Comaltepec (\citealt{Silverman1994,Silverman1997a}), and the Am languages AMU (\citealt{Zendejas2000}) and SPA \citep{Kim2011}. \figref{fig:dobui:1} presents spectrograms of controlled and ballistic syllables /we/ (M) ‘red’ and /we·/ (L) ‘two’ (from ELA, 2011). Ballisticity is noted as a syllable-final dot: /CV·/.

  
\begin{figure}
\subfigure[/we/ (M) ‘red’]{
\includegraphics[height=.4\textheight]{figures/DoBui-img001.png}
}
\subfigure[/we·/ (L) ‘two’]{
\includegraphics[height=.4\textheight]{figures/DoBui-img002.png}
}
\caption{Controlled /we/ (M) ‘red’ (top) and ballistic /we·/ (L) ‘two’ (bottom) (from ELA, 2011)}
\label{fig:dobui:1}
\end{figure}

\figref{fig:dobui:2} presents the same contrast in a more complex environment with /tsa̤ʔ/ ‘plant fibers; ash’ (top) and /tsa̤ʔ·/ ‘sponge’ (bottom) (from \citealt{JP2021}). Here the vowel is aspirated and the syllable ends in a glottal stop, which illustrates the prosodic complexity that ballisticity is part of. The annotation reflects the multiple realizations of laryngealization through two vowel slots, but this is not to indicate vowel length.

  
\begin{figure}
\subfigure[/tsa̤ʔ/ ‘plant fibers; ash’]{
\includegraphics[height=.4\textheight]{figures/DoBui-img003.png}
}
\subfigure[/tsa̤ʔ·/ ‘sponge’]{
\includegraphics[height=.4\textheight]{figures/DoBui-img004.png}
}
\caption{Controlled /tsa̤ʔ/ ‘plant fibers; ash’ (top) and /tsa̤ʔ·/ ‘sponge’ (bottom) (from \citealt{JP2021})}
\label{fig:dobui:2}
\end{figure}

Taken as a whole, results from these different studies affirm the impressionistic descriptions of the feature in the literature. Counterpart ``controlled'' syllables show steady initial surge, sustained voicing, stable formants throughout the syllable, and no final aperiodicity, which is typical of modal voicing. Articulatory studies of Lalana and Palantla Ch (respectively, \citealt{Mugele1984} and \citealt{MerrifieldEdmondson1999}) point to subglottal pressure as the major articulatory enhancer of ballistic syllables. This is contrary to \citet{Silverman1994} for whom laryngeal abduction is the major articulator.

In all the languages in question, ballisticity occurs widely across different structural syllable types. Depending on the language, these can be oral or nasal, open or closed syllables, with laryngealized or modal onsets, with long or short vowels. Ballisticity is realized with only a partial or a different inventory of lexical tones, as compared to the controlled pronunciation. Depending on the language, ballisticity is contrastive only in the stressed position or on any syllable of a word; it can be found in both free and bound morphemes. 

Phonological accounts for Chinantec \citep{Silverman1994} and Amuzgo (\citealt{Zendejas2000}) combine acoustic and articulatory evidence to interpret final frication and loss of voice on ballistic syllables as breathy vowels [Vh], phonologically represented as [spread glottis]. For both authors, other phonetic correlates are considered non-phonological, possibly serving to enhance the saliency of the trait in question. In SPA, \citet{Kim2011} finds differing tone sandhi patterns between ballistic tones and controlled tones as tentative evidence for the phonological reality of this contrast. However, she also calls into question the legitimacy of proposing entirely separate tone inventories. 

In morphology, ballistic syllables can mark lexical distinctions, morphological derivations, and serve as markers for animacy agreement and tense-aspect-modality. For example in Lealao Ch, \citet{Palancar2015} shows that ballistic syllables in the first personal plural future serve as a marker of verb class I. \citet{Dobui2018} notes that ballisticity marks animacy agreement in a class of adjectives in AMU. For the most part, ballisticity does not map one-to-one with derivational or inflectional values. Ballisticity often combines with other alternations (tone, nasality, laryngealization, length) in what \citet{Palancar2015} calls “prosodic inflection”, although this also holds for word formation processes. 

The examples below illustrate the ballisticity contrast in a variety of syllabic and morphological environments.\footnote{When possible, tones are represented with letters to avoid confusion between different traditions in Otomanguean glossing stylesː H is high, M is mid, L is low. Sequential tones are noted with the same divisions as the gloss, and contour tones are noted together, e.g. \textit{ka-hndá} (H-MH) ‘\textsc{class}-frog’ (AMU). In the examples using unverifiable data from second-hand sources, notation used by the respective authors is reproduced. Where relevant, stress is marked with an acute accent. Transcriptions of examples conform to IPA standards with the exception of breathy and creaky segments, which are written using the orthographic standard, i.e. \textit{ka-ʔnãʔ} (H-M) ‘funny’ rather than [ka-n̰ãʔ].}

\ea\label{ex:dobui:1}
{ Lexical contrast in AMU (\citealt{Bauernschmidt2010}: 8–9)}\\
\ea\label{ex:dobui:1a}
{Controlled}\\
\gll ka-ʔnãʔ (H-M)\\
     \textsc{class}-funny\\
\glt ‘funny’

\ex\label{ex:dobui:1b}
{Ballistic}\\
\gll ka-\textbf{ʔnã·} (H-M)\\
     \textsc{class}-be.tasty\\
\glt ‘be tasty’
\z
\z

\ea\label{ex:dobui:2}
{ Word formation in AMU \citep[314]{Bauernschmidt2010}}\\
\ea\label{ex:dobui:2a}
{Controlled}\\
  səiʔ (M)\\
\glt ‘flesh,~muscle, body’
\ex\label{ex:dobui:2b}
{Ballistic}\\
\textbf{səiʔ}· (M)\\
\glt ‘meat’
\z
\z

\newpage
\ea\label{ex:dobui:3}
{ Morphological agreement in AMU \citep{JP2017}}\\
\ea\label{ex:dobui:3a}
{Controlled}\\
\gll ka-hã  (H-M=H) \\
     \textsc{class}-yellow.\textsc{a}\textsc{n}\\
\glt ‘yellow’ (\textsc{an})
\ex\label{ex:dobui:3b}
{Ballistic}\\
\gll ka-\textbf{hã·} (H-M=H)\\
     \textsc{class}-yellow.\textsc{inan}\\
\glt ‘yellow’ (\textsc{inan})
\z
\z

\ea\label{ex:dobui:4}
{ Lexical contrast in Usila Ch (\citealt{SkinnerSkinner2000}: 269)}\\
\ea\label{ex:dobui:4a}
{ Controlled}\\
  pei\textsuperscript{1}\\
\glt ‘extremity, tip’
\ex\label{ex:dobui:4b}
{ Ballistic}\\
\textbf{pei¹·}\\
\glt ‘small, little’
\z
\z

\ea\label{ex:dobui:5}
{ Word formation in Usila Ch (\citealt{SkinnerSkinner2000}: 192)}\\
\ea\label{ex:dobui:5a}
{Controlled}\\
  lia⁴\\
\glt ‘like’ (\textsc{adv})
\ex\label{ex:dobui:5b}
{Ballistic}\\
\textbf{lia⁴·}\\
\glt ‘as well’ (\textsc{conj})
\z
\z

\ea\label{ex:dobui:6}
{ Morphological contrast in Usila Ch (\citealt{SkinnerSkinner2000}: 474)}\\
\ea\label{ex:dobui:6a}
{Controlled}\\
\gll a²-kwan³-i³\\
     \textsc{class}-dress-3\textsc{sgposs}\\
\glt ‘her dress’
\ex\label{ex:dobui:6b}
{Ballistic}\\
\gll a²-\textbf{kwan³⁻³·}\\
     \textsc{class}-dress.2\textsc{sgposs}\\
\glt ‘your dress’
\z
\z

These different accounts clarify observed acoustic and articulatory features, effects in tonal allophony, and distributions across morphosyntactic systems in the Otomanguean languages that display the feature. In yet other languages, both Otomanguean and not, the term “ballisticity” has been used to describe similar but ultimately different phenomena. A discussion of this literature helps to clarify the feature addressed in this study. 

\subsection{Ballisticity in other languages}

The terms ``ballistic'' versus ``controlled'' have also been employed for other Otomanguean languages and also for the Finno-Ugric and Scandinavian languages of Northern Europe. This section discusses these phenomena and why they ultimately differ from the ballistic feature found in Am and Ch.

For the Otomanguean Mezquital Otomi, \citet{Wallis1968} used the terms ``ballistic'' and ``controlled'' in order to “describe [other] distinctive features of the syllable” rather than the “contrastive syllable types” of Am and Ch (p. 78). The author described the feature as non-contrastive but active in word-level and sentence-level tone sandhi, where predictable combinations of tones result in final aspiration and tone allophony (higher high tones, lower low tones). \citegen{Palancar2013} account treats this same phenomenon as aspiration conditioned by intersyllabic environments. This chapter does not follow Wallis in employing the term “ballisticity”.

For Jalapa Mazatec, \citet{SilvermanEtAl1995} conducted phonetic studies on a set of minimal pairs suspected of contrasting ballisticity. They showed that these do not exhibit phonetic cues typical of ballisticity. Three phonetic correlates typical of ballisticity were examined: fortis consonants, surge and decay in intensity, and post-vocalic aspiration. All three showed no systematic differences between ballistic and controlled syllables. However, fundamental frequency was slightly higher and duration notably shorter in ballistic syllables. It was also noted that length is contrastive, but only in morphologically complex environments. This would leave a gap in simplex forms, which presumably do not contrast length. The authors tentatively proposed that ballistic/controlled forms fill this gap in simplex forms, contrasting ``ballistic'', or monomoraic, syllables with ``controlled'', or bimoraic, syllables. Correlates of what initially appear to be ballistic contrasts in Jalapa Mazatec are thus remapped onto the concept of length.

Neither Otomi nor Jalapa Mazatec seems to display the phonetic correlates or phonological structures corresponding to those linked to ballisticity in Am and Ch. Nevertheless, some correlates are similar, which might indicate different evolutionary paths of reconstructed final aspiration. 

The terms ``ballistic''/``controlled'' have also been used in literature on unrelated languages of Northern Europe, notably in Sámi \citep{Harms1975}, Estonian and Danish (\citealt{Kuznetsova2018} and the references therein). The terms are used to describe prosodic contrasts with accent-like functions and varying phonetic realizations that form a set of contrasts referred to as “balanced/controlled” versus “unbalanced/ballistic”. In a general sense, this is much like in Otomanguean languages: in both language phyla, ‘ballisticity’ involves perceived spikes in articulatory energy and non-prototypical prosodic functions (which are neither canonical tone nor stress, but something in between).

Some phonetic indicators from these and other European languages find parallels with ballisticity in Otomanguean languages (as described in \citealt{Morrison2019}): tone (in Swedish, Norwegian, Franconian), duration (Estonian), and laryngealization (Standard Danish), or all of the above (varieties of Scottish Gaelic). However, phonetic indicators of laryngealization in these languages are those typical of constricted glottis (vocal fold stiffening) rather than spread glottis, as is the case for ballisticity in Otomanguean languages. Additionally, the aforementioned contrasts in these languages functionally interact with stress and, in the Finno-Ugric languages, with duration, while in Otomanguean languages ballisticity is independent of both, as shown in \sectref{sec:dobui:3}.

\section{Phonological and morphological status of ballisticity}\label{sec:dobui:3}
\label{bkm:Ref124173421}
This section looks at the phonological and morphological status of ballisticity in AMU and Ch, the languages for which the trait has been clearly identified phonetically. 

\subsection{Xochistlahuaca Amuzgo}\label{sec:dobui:3.1}
\label{bkm:Ref102915118}
Ballisticity has the widest functionality in AMU. It can occur in all syllable types and can co-occur with all types of word-prosodic inventories, in both cases unpredictably. It can appear more than once in a morphosyntactic word, not only in the stressed position. It also regularly marks animacy agreement in a large class of adjectives, and irregularly marks operations of word formation. Before addressing the distribution of the feature, basic information on AMU phonology (based on \citealt{Dobui2018}) will be provided.

Segmental inventories for AMU are given in \tabref{tab:dobui:1} and \figref{tabfig:dobui:2}. Consonants contrast across five places of articulation. Marginal segments are in parentheses, allophones are in square brackets. All voiced segments (except /β/) can be breathy or creaky. In both tables, the phonemes conform to IPA, but in transcribed examples, I also employ the practical orthography used by some Amuzgo speaking communities, e.g. /m̤/ is written <hm>, and /ɲ̰/ is <ʔɲ>.


\begin{table}
\fittable{
\begin{tabular}{llllll}
\lsptoprule
& {\bfseries Bilabial} & {\bfseries Apico-dental} & {\bfseries Postalveolar} & {\bfseries Velar} & {\bfseries Glottal}\\
\midrule
{\bfseries Stop} & { (p)}  & { t} & {     tʲ} & { k kʷ} & { ʔ}\\
\tablevspace
{\bfseries Affricate} &  & { ts} & {   ʧ} &  & \\
\tablevspace
{\bfseries Fricative} & { (β)} & { s} & {   ʃ} &  & \\
\tablevspace
{\bfseries Nasal} & { m m̤ m̰} & { n [n ɲ̩ ŋ̩ nᵈ] n̤ n̰} & {       ɲ [ɲᵈʲ]     ɲ̤ ɲ̰} &  & \\
\tablevspace
{\bfseries Trill} &  & { r} &  &  & \\
\tablevspace
{\bfseries Tap} &  & { ɾ} &  &  & \\
\tablevspace
{\bfseries Lateral} &  & { l l̤ l̰} &  &  & \\
\tablevspace
{\bfseries Glide} & { w w̤ w̰} &  & {   j j̤ j̰} &  & \\
\lspbottomrule
\end{tabular}
}
\caption{Consonant inventory of AMU}
\label{tab:dobui:1}
\end{table}

\figref{tabfig:dobui:2} shows the seven contrastive oral vowels and five nasal vowels. All vowels can also be breathy or creaky, substantially increasing the number of contrastive values possible. As for voiced consonants, non-modal phonation found in examples will be transcribed using the orthographic conventions mentioned above, e.g. /e̤/ is <he> and /o̰\~{}  / is <ʔõ>\footnote{Note that the realization of the nasal vowel /õ/ is accompanied by an unreleased and voiceless labial closure and is phonetically transcribed as [ãwm̥].}.


\begin{figure}
\begin{tabular}{ccc}
{ i i̤ ḭ}  &  & {          u ṳ ṵ} \\
{      e e̤ ḛ ẽ ẽ̤ ḛ}  &  & { o o̤ o̰ õ õ̤ õ̰} \\
{ ɛ ɛ̤ ɛ̰ ɛ ɛ̤ ɛ̰} &  & {        ɔ ɔ ɔ̤ ɔ̰}\\
& { a a̤ a̰ ã ã̤ ã̰} &
\end{tabular}
\caption{Vowel inventory of AMU}
\label{tabfig:dobui:2}
\end{figure}

There are three register tones: H, M, L. The first two phonetically descend slightly, while the last one phonetically rises slightly. The contour tones are HL, which may rise slightly upward at the end, LM, and MH. The latter two both rise, as expected. A near-minimal tonal sextuple is given in \tabref{tab:dobui:3} (all data from \citet{Bauernschmidt2010} with page numbers given in the table).


\begin{table}
\begin{tabularx}{.66\textwidth}{Ql}
\lsptoprule
{ hnda (H)} & { ‘river’ (p. 121)}\\
\tablevspace
{ hnda· (M)}\newline
     {child.\textsc{3sg.poss}} & { ‘his/her child’ (p. 121)}\\
\tablevspace
{ ka-hnda· (L-L)}\newline
     \textsc{class}-expensive & { ‘expensive’ (p. 5)}\\
\tablevspace
{ hndaʔ (HL)} & { ‘ordered, organized’ (p. 123)}\\
\tablevspace
{ ka-hnda (H-MH)}\newline
     \textsc{anim}-frog & { ‘frog’ (p. 5)}\\
\tablevspace
{ hnda (LM)} & { ‘moment, time’ (p. 121)}\\
\lspbottomrule
\end{tabularx}
\caption{Minimal sextuple of tones in AMU}
\label{tab:dobui:3}
\end{table}

On ballistic syllables, the same tones are realized over a compressed duration, given the devoicing of the final portion of the vowel. Low ballistic tone does not show a final rise, and no ballistic equivalent to LM exists.

The maximal syllable structure in AMU is CCGVʔ where G is a glide and the final glottal stop is segmental. Minimal (V) and maximal (CCGVʔ) syllables are uncommon, while CV(ʔ) and CCV(ʔ) are most common. 

\subsubsection{Distribution of ballisticity in AMU}

In \xxref{ex:dobui:4V}{ex:dobui:4ChVʔ}, four syllable structures, V, Vʔ, CV, CVʔ, with lexically oral and nasal (or morphologically nasalized, see \citealt{Dobui2021} and on SPA  \citealt{CortézVázquez2016}) nuclei are crossed with the prosodic inventories of ballisticity and non-modal phonation, represented by \textit{h/ʔ} to the left of the vowel (all data from \citealt{Bauernschmidt2010}).\footnote{For better visibility, data sources in \xxref{ex:dobui:4V}{ex:dobui:4ChVʔ} are given with initials followed by page numbers: B for \citet{Bauernschmidt2010}, AP for  \citet{ApóstolPolanco2014}, and JG for  \citet{deJesúsGarcía2004}. For example, “AP6” should be understood as  \citet[6]{ApóstolPolanco2014}.} Ballisticity exists for every type of syllabic structure, including underpopulated types (V and Vʔ) and crosses with each prosodic feature. Where possible, minimal pairs are given to illustrate the contrastive reach of ballisticity. Most are monomorphemic, but in some cases it was not possible to find simplex examples. For example, in contrast to a ballistic and breathy nasal syllable with an initial stop onset \textit{tho\~{} }· (H) ‘bent thread/wire’ and ‘as well as’, I included a controlled counterpart \textit{t-hõ} (n.d. for tone) ‘s/he put’, the past marked form of \textit{ma-ho\~{} } ‘to put’ (M-H).



\eabox[-.5\baselineskip]{\label{ex:dobui:4V}
\dobuibox{V}
          {a (M)\newline‘well, good’ (adv) (B1)}
          {--}
          {\gll ma-kwaʔ=\textbf{a·}~(M-M=M)\\
               \textsc{prog.sg}-eat.\textsc{1sg=1sg}\\
               \glt ‘I am eating’ (AP6)}
          {\gll ma-kʷãʔ{\textasciitilde}\textbf{ã·}~(M-M{\textasciitilde}M)\\
               \textsc{prog.sg}-eat.3sg{\textasciitilde}3sg\\
               \glt ‘s/he is eating’ (AP7)}
}

\ea
\dobuibox{ʔV}
          { ʔa   (H)\\
            \glt ‘thick, viscous’ (B134)
          }
          {\textbf{ʔɛ·}  (L)\\
               \glt ‘because’ (B134)
          }
          {}
          {}
\z


\ea
\dobuibox{hV}
     {ho (H)\\
          \glt ‘place’ (B131)\\
     }
     {kʷi-laʔ-hõ=ɲᵈʲe~(M-H-MH=H)\\
          \textsc{prog}.\textsc{pl}-\textsc{caus}.\textsc{pl}-compare=\textsc{et.pl}\\
          \glt \mbox{‘they are comparing’ (JG737)}
     }
     {\textbf{ho·}~(H)\\
          \glt ‘for, towards’ (used with motion verbs)  (B131)
     }
     {\gll \textbf{hõ·}~(H)\\
          3\textsc{sghum}\\
          \glt ‘s/he’  (B131)
     }
\z


\ea  Vʔ      ---
\z


\ea
\dobuibox{ʔVʔ}
     {---}
     {---}
     {\textbf{ʔɛʔ·}~(L)\\
      \glt ‘ew’ (sound of disgust)  (B134)
      }
      {---}
\z

\newpage
\ea
\dobuibox{hVʔ}
          {haʔ~(M)\\
               \glt ‘heavy’ (B119)
          }
          {\gll hãʔ=ɲe~(H=H) \\
               dark=\textsc{te}\\
               \glt ‘dark, obscure’  (B119)
          }
          {\textbf{haʔ·} ʧiu~(L L)\\
               \glt ‘of course’  (B119)
          }
          {\gll tueʔ-\textbf{hãʔ·}{\textasciitilde}ã·~(M-H{\textasciitilde}M)\\
               \textsc{pst}.become-dark.\textsc{3sg} {\textasciitilde}\textsc{3sg} \\
               \glt ‘s/he fainted’  (B244)
          }
\z

\ea
\dobuibox{CV}
          {to~(H) \\
               \glt ‘girdle’  (B349)
          }
          {tõ~(H) \\
               \glt ‘knot (in thread, rope, hair)’  (B349)
          }
          {kwi-\textbf{to·}~(M-L)\\
               \glt ‘if by chance’  (B97)
          }
          {\textbf{tõ·}~(H) \\
               \glt ‘exactly’  (B349)
          }
\z


\ea
\dobuibox{CʔV}
     {\gll kʷi-tʔɔ~(M-LM) \\
          \textsc{prog.pl-pl}.answer\\
          \glt ‘they answer’  (JG494)
     }
     {\gll tʔõ~(H)\\
          \textsc{pst}.sprout.3\textsc{sg}\\
          \glt ‘s/he sprouted’(B78)
     }
     {ka-\textbf{tʔo·}~(H-M) \\
          \glt ‘truncated, stumpy’  (B31)
     }
     {kʷi-\textbf{tʔõ·}~(M-L) \\
          \glt ‘to spread out’  (B96)
     }
\z


\ea
\dobuibox{ChV}
     {\gll t-ha-kwa\~{}~(L-M)\\
          \textsc{pst}-go.call.someone.3\textsc{sg}\\
          \glt ‘s/he went to call someone’  (AP208)
     }
     {\gll  t-hõ~(MH)\\
          \textsc{pst}-reach.into.something.3\textsc{sg}\\
          \glt ‘s/he reached into something’  (JG185)
     }
     {\gll \textbf{thɔ·}~(L)\\
               \textsc{pst}.go.1\textsc{sg}\\
               \glt ‘I went’  (JG186)
     }
     {\textbf{thõ·}~(H) \\
          \glt ‘bent thread/wire’,\glt ‘as well as’  (B345)
     }
\z
\clearpage
\ea
\dobuibox{CVʔ}
     {toʔ~(H) \\
          \glt ‘be full, busy’  (B349)
     }
     {tõʔ~(H)\\
          \glt ‘head of garlic; main rib of a leaf; bundle (of leaves, fruit, etc.)’  (B350)
     }
     {\textbf{toʔ·}~(M)\\
          \glt ‘garbage, waste (noun); rotten, decayed’  (adjective)  (B349)
     }
     {\textbf{ntõʔ·}~(HL)\\
          \glt ‘oven’  (B290)
     }
\z

\ea
\dobuibox{CʔVʔ}
     {\gll kʷi-wi-tʔaʔ~(M-H-HL) \\
          \textsc{prog}-become-full\\
          \glt ‘to accumulate’  (B114)
     }
     {\gll kʷi-tʔõʔ~(M-L) \\
          \textsc{prog}-scatter\\
          \glt ‘to scatter, to disperse’  (B96)
     }
     {\gll ma-\textbf{wʔɔʔ·}~(M-M)\\
          \textsc{prog.sg}-answer.2\textsc{sg}.2\textsc{sg}\\
          \glt ‘you answer’  (JG494)
     }
     {\gll \textbf{mʔãʔ·}~(L)\\
          \textsc{pl.det} (B254)\\
     }
\z

\ea
\dobuibox{ChVʔ \label{ex:dobui:4ChVʔ}}
     {thoʔ~(HL) \\
          \glt ‘measurement unit from thumb to little finger’  (B345)
     }
     {thõʔ~(MH) \\
          \glt ‘together, unified, in agreement’  (B346)
     }
     {\textbf{thɔʔ·}+su~(M+H) \\
          \glt ‘grinding stone’  (B346)
     }
     {\textbf{thaⁿʔ·+ɲᵈʲo}~(M+H)\\
          \glt ‘lip’  (B344)
     }
\z



% Wide distribution across syllable types and combinations with different prosodic inventories show that ballisticity is independent of other prosodic contrasts. Still, minimal pairs rarely contrast solely in ballisticity, rather lexical contrast is often enhanced by other described prosodic contrasts.

\tabref{tab:dobui:summary1} summarizes the contrasts seen in examples \xxref{ex:dobui:4V}{ex:dobui:4ChVʔ}, where empty cells correspond to forms that were not found. Wide distribution across syllable types and combinations with different prosodic
inventories show that ballisticity is independent of other prosodic contrasts Still, minimal pairs rarely contrast solely in ballisticity; rather lexical contrast is often enhanced by other described prosodic contrasts.

\begin{table}
\caption{A summary of ballisticity distribution across syllable and phonation types in AMU}
\label{tab:dobui:summary1}
\begin{tabularx}{\textwidth}{lCCCC}
\lsptoprule
                                        & \multicolumn{2}{c}{controlled} & \multicolumn{2}{c}{ballistic}\\
\cmidrule(r){2-3}\cmidrule(l){4-5}
Syllable structure  and phonation types & oral & nasal & oral & nasal\\
\midrule
V                                       & +   &    --   &   +   &   +   \\
ʔV                                      & +   &    --   &   +   &   --   \\
hV                                      & +   &    +   &   +   &   +   \\
Vʔ                                      & --   &    --   &   --   &  --   \\
ʔVʔ                                     & --   &    --   &   +   &   --   \\
hVʔ                                     & +   &    +   &   +   &   +   \\
CV                                      & +   &    +   &   +   &   +   \\
CʔV                                     & +   &    +   &   +   &   +   \\
ChV                                     & +   &    +   &   +   &   +   \\
CVʔ                                     & +   &    +   &   +   &   +   \\
CʔVʔ                                    & +   &    +   &   +   &   +   \\
ChVʔ                                    & +   &    +   &   +   &   +   \\
\lspbottomrule
\end{tabularx}
\end{table}


As also seen in \xxref{ex:dobui:4V}{ex:dobui:4ChVʔ}, ballisticity marks both functional and lexical, and bound and free morphemes. It is also involved in morphological marking across different inflectional classes and, to a lesser degree, in word formation.

All morphemes are monosyllabic.\footnote{Although
     a monomorphemic “sesquisyllable” (literally one-and-a-half syllables, i.e. a major syllable and a minor dependent syllable) consisting of a syllabic sonorant before an occlusive onset of a following major syllable (e.g. n̩.CV: \textit{n̩təĩʔ}· \textit{(HL)} ‘adobe’) is frequently found.}
Phonological words are often multisyllabic with multiple class markers or tense-aspect-mood markers in pre-stem position, possibly multimorphemic stems, and multiple suffixes and/or enclitics in postposition. In the prosodic word, stress is fixed on the stem-final syllable, which is “redundantly marked by differences of pitch allophones and of duration of voicing in the syllable nuclei” \citep[472]{Bauernschmidt1965}. Syllables preceding and following the stem have reduced vocalic and tonal inventories. The examples in \xxref{extab:dobui:6start}{extab:dobui:6end} show ballisticity on root syllables, as seen in
% the row
\REF{extab:dobui:6free}
% (7)
which contains free lexical roots and
% the row
% (9)
\REF{extab:dobui:6deriv}
with lexically derived stems. It can also occur in the bound morphemes in pre- and postposition of the stem, as seen in the bound morphemes in
% rows (8)
\REF{extab:dobui:6bound}
and
% (10)
\REF{extab:dobui:6agreement}
where it marks agreement.

Morphologically, contrastive ballisticity is observed both as lexical and inflectional word prosody.
In
% (7)
\REF{extab:dobui:6free}, ballisticity differentiates two lexical roots.
In
% (8)
\REF{extab:dobui:6bound}, ballisticity in combination with tone differentiates between grammatical morphemes like 1\textsc{pl} inclusive versus 1\textsc{pl} exclusive.
In
% (9)
\REF{extab:dobui:6deriv}, ballisticity distinguishes between ‘flesh’ and ‘meat’, and between ‘well’ and ‘geyser’. However, semantically related minimal pairs like these are rare and the form-to-function derivational value of ballisticity is inconsistent.
In
% (10)
\REF{extab:dobui:6agreement}, ballisticity marks an adjectival stem to accord with an inanimate noun. Contrary to the examples
in
%(9)
\REF{extab:dobui:6deriv}, inanimacy marking with ballisticity is productive in a large class of adjectives consistently marked for animacy agreement.
In
%(10)
\REF{extab:dobui:6agreement}, subject agreement for the second person singular is also regularly marked in a verb class. In this same example, ballisticity marks only the syllable of the stem, with no spreading to adjacent syllables. Similarly,
in
%(10)
\REF{extab:dobui:6agreement}, syllables mismatch for ballisticity within the same prosodic word domain, which shows that no ballistic spread or harmony occurs.

% Distribution of ballistic vs. controlled syllables within the prosodic word in AMU (JP, 2017)


\ea\label{extab:dobui:6start}\label{extab:dobui:6free} Free morphemes
     \ea controlled\\
          wɛ~(M)\\
          \glt  ‘red’
     \ex ballistic \\
          \textbf{wɛ}·~(L)\\
          \glt  ‘two’\\
     \z
\z
\ea Bound morphemes \label{extab:dobui:6bound}\\
     \ea controlled\\
          =ja~(L)\\
          1\textsc{pl.incl} \\
     \ex ballistic\\
          \textbf{=ja}·~(LH)\\
          \textsc{1pl.excl}\\
     \z
\z

\ea Derivation\label{extab:dobui:6deriv}
     \ea
          \ea controlled\\
                seiʔ~(M)\\
               \glt ‘flesh,~muscle, body’
          \ex  ballistic \\
               \textbf{seiʔ·}~(M)\\
               \glt ‘meat’\\
          \z
     \ex
          \ea  controlled\\
               tsuiʔ~(M)\\
               \glt ‘(water) well’
          \ex  ballistic \\
               \textbf{tsuiʔ·}~(M)\\
               \glt ‘geyser’\\
          \z
     \z
\z
\ea Agreement\label{extab:dobui:6end}\label{extab:dobui:6agreement}
     \ea
          \ea controlled\\
               \gll ka-ʧi·~(H-H) ka\textbf{-}hã~(H-M)\\
                    \textsc{anim}-eagle \textsc{class}\textsc{\textsubscript{adj}}-yellow.\textsc{an>}\\
               \glt ‘the eagle is yellow’
          \ex ballistic\\
               \gll  lja~(H) ka-\textbf{hã·}~(H-M)\\
                    dress \textsc{class}\textsc{\textsubscript{adj}}-yellow.\textsc{inan}\\
               \glt ‘the dress is yellow’\\
          \z
     \ex
          \ea controlled\\
               \gll  ma-kʷãʔ{\textasciitilde}ã~(M-M{\textasciitilde}M)\\
                    \textsc{prog.sg}-eat.3\textsc{sg.hum}{\textasciitilde}3\textsc{sg.hum}\\
               \glt  ‘s/he is eating’
          \ex ballistic\\
               \gll ma-\textbf{kʷaʔ·}~(M-HM)\\
               \textsc{prog.sg}-eat.2\textsc{sg}\\
               \glt ‘you are eating’\\
          \z
     \z
\z



Ballisticity can occur once, more than once, or not at all at the word level. In examples \xxref{ex:dobui:11}{ex:dobui:12}, stress co-occurs with both ballistic and controlled syllables. In the compound word for ‘ball’ in \REF{ex:dobui:11}, the root for ‘fruit’ is ballistic while the root for ‘rubber’ remains controlled. The Spanish loan for ‘chair’ ends in a ballistic syllable, which might correspond to the word-final stress in the original language. The root for ‘warm’ appears independently in \REF{ex:dobui:12a} and derived as a reflexive verb in \REF{ex:dobui:12b}, with no change in ballisticity. The data elicited by the \citet{ELA2012} come from the \textit{Picture series on topological relations} by \citet{BowermanPederson1992}.

\ea\label{ex:dobui:11}
{\label{bkm:Ref124427360}\citet{ELA2012}}\\
\glll nti~(HM) \textbf{tɛ·}+hnᵈʲoʔ~(L+LM) na~(M) khɛ~(L) su\textbf{lɛ·}~{(H M)}\\
     nti       \textbf{tɛ·}+hnᵈʲóʔ        na     khɛ     su\textbf{lɛ·}~{(HM L+LM M L H M)}\\
     \textsc{exist}.inside fruit-rubber \textsc{comp} under chair\\
\glt ‘The ball is under the chair.’
\z

\ea\label{ex:dobui:12}
\ea\label{ex:dobui:12a}
{\citet[270]{Bauernschmidt2010}}\\
\glll  nda-tio~(H-M) \textbf{wi·} (L)\\
       nda-tió \textbf{wi·} (H-M L)\\
     water warm\\
\glt ‘warm water’

\ex\label{ex:dobui:12b}
{\label{bkm:Ref101351927}\citet{ELA2012}}\\
\glll tio-ʧó~(M-HM) ma-tsəi-\textbf{wi}\textbf{·}=ne (M-M-L=M) \\
   tio-ʧó        ma-tsəi-\textbf{wí}\textbf{·}=ne {(M-HM M-M-L=M)}\\
     \textsc{class}\textsc{\textsubscript{hum}}-boy \textsc{prog}.\textsc{sg}-\textsc{caus}-warm=\textsc{et}\\
\glt ‘A boy is warming himself.’
\z
\z

In general, words where all syllables are controlled are common while words with ballistic syllables are a minority, although multisyllabic words where all syllables are ballistic do exist, as in \REF{ex:dobui:13}. This distribution stems from the fact that multisyllabic words are more likely to have class markers or preverbs. Such morphological units were once fully lexical but have become grammaticalized and have lost the prosodic contrast of ballisticity. The latter process indicates that controlled syllables are in general default, or unmarked, in the language. Compare the word for ‘fruit’, as it appears as a full content word containing a ballistic syllable in \REF{ex:dobui:14a}, to its function as a class marker in \REF{ex:dobui:14b}, where it is no longer ballistic. In \REF{ex:dobui:11} above, given the ballisticity of the root for ‘fruit’ the whole word is considered a compound with two full content lexical roots. Ballisticity loss through grammaticalization is not consistent for all noun class markers, which are often still analyzable as content words. For example, the root \textit{ju}\textbf{·} (M) ‘person’ is ballistic both as an independent word and as a noun class marker, as in \REF{ex:dobui:13}. Inverse cases where morphemes become ballistic through grammaticalization have not been attested.

\ea\label{ex:dobui:13}
{\label{bkm:Ref101355886} \citet{ELA2012}}\\
\glll  \textbf{ju·=sʔa·}~(L=L) ma-ʔma~(M-MH) \textbf{hnõ·} (L)\\
       \textbf{ju·=sʔa·}       ma-ʔma        \textbf{hnõ·} {(L=L M-MH L)}\\
     \textsc{class}\textsc{\textsubscript{hum}}=man \textsc{prog.sg}-smoke.3\textsc{sg} cigarette\\
\glt ‘The man is smoking a cigarette.’
\z

\ea\label{ex:dobui:14}
{\citet{ELA2012}}\\
\ea\label{ex:dobui:14a}
\glll \textbf{tɛ·}~(L) ntʲha=na~(ML=M) ʃkẽ~(L) tsʔõ (LM)\\
      \textbf{tɛ·}     ntʲha=na        ʃkẽ     tsʔõ {(L ML=M L LM)}\\
     fruit hang=\textsc{inan} head.\textsc{3sgposs} tree\\
\glt ‘A piece of fruit hangs from the tree.’
\ex\label{ex:dobui:14b}
\glll  tɛ-mansana~(L-HLM) nti~(H) lansa (HB)\\
       tɛ-mansana         nti     lansa {(L-HLM H HB)}\\
     \textsc{class}\textsc{\textsubscript{fruit}}\textbf{\textit{-}}apple inside arrow\\
\glt ‘Inside of the apple there is an arrow.’
\z
\z

To summarize the characteristics of ballisticity in AMU, this feature:

\begin{itemize}
\item[(1)] is unpredictably distributed across different syllable structures;
\item[(2)] co-occurs with all types of laryngeal specifications and all other types of word-prosodic inventories;
\item[(3)] can occur in stressed or unstressed syllables;
\item[(4)] is found on bound and free morphemes;
\item[(5)] can occur more than once in a word;
\item[(6)] can contrast lexical or grammatical minimal pairs (and contexts) alone or in combination with tone, nasality, or phonation;
\item[(7)] and, regular form-to-function morphemic value is only found in animacy marking in a class of adjectives.
\end{itemize}

\subsection{Chinantec languages}

Of the 23 Chinantec varieties present in the corpus by \citet{Rensch1968}, ballisticity is attested in lexical roots in nine varieties across all five language groups: I-Northern highlands (in the town of Sochiapan), II-Transition zone (Palantla and Tepetotutla), III-Lowlands (Ozumacín), IV-Southern Lowlands (Latani, Lalana, Lealao), and V-Mountain highlands (Quiotepec and Comaltepec). For the remaining 14 varieties (which are part of language groups I, III and V) further data and study is needed to establish whether the ballisticity contrast is active. 

The consonantal inventory for Proto-Chinantec is given below in \tabref{tab:dobui:7}. For vowels, \citet[11]{Rensch1989} reconstructed the following: *i, *e, *a, *u, * ɨ, and *ə, in addition to several diphthongs in Chinantec. The tonal inventory includes *H, *L, *HL, *LH, and *HLH. Vowel length, ballisticity and vowel nasalization are also reconstructed. 

\begin{table}
\begin{tabularx}{\textwidth}{lXXXl}
\lsptoprule
  & Bilabial & Alveolar & Velar & Glottal\\
\midrule
{\bfseries Stop} & { *p}  & { *t} & { *k   *kʷ} & { *ʔ}\\
& { *b} & { *d} & { *g    *gʷ} & \\
\tablevspace
Fricative &  & { *s} &  & { *h}\\
\tablevspace
Nasal & { *m} & { *n} & { *ŋ} & \\
\tablevspace
Approximant & { *w} & { *l} & { *j} & \\
\lspbottomrule
\end{tabularx}
\caption{\label{tab:dobui:7}. Reconstructed consonants in Proto-Chinantec\\
(adapted from \citealt{Rensch1989}: 11)}
\end{table}

In those modern Ch languages where the feature is attested, ballisticity is also widely and unpredictably distributed across syllable structures and co-occurs with other prosodic inventories, such as tone, nasalization and laryngealization of sonorants and vowels, and vowel length. Syllable shapes in almost all Ch languages are usually CV or CGV, where G is a glide or a vocalic element of a diphthong. In simplex morphemes, across the different languages codas may always have a glottal stop, sometimes a glide, and a “post-syllabic nasal” consonant in Lalana, Temextitlán, Yolox, Ozumacín \citep{Rensch1968}.\footnote{In some Ch languages, a final glottal fricative is cited in coda position in the literature. For these same languages, ballisticity is not mentioned (see \citealt{Rensch1968}). This may serve as an indication of potentially undetermined ballisticity.} 

\subsection{Distribution of ballisticity in Ch languages}%\todo{why the abbreviation Ch?}

Examples \xxref{extab:dobui:8V}{extab:dobui:8CVʔ} show the distribution of the ballistic vs. controlled contrast across different syllable structures with short vowels in Comaltepec Ch (all data from \citealt{AndersenEtAl2021}).\footnote{In
     the data below, verbs should not be considered as infinitives, e.g. examples like \textit{hee}\textbf{·} (LM) \textit{kiaʔr}\textbf{·} (M) \textsc{inan} ‘to accompany’ are given with a final rhotic, which is a verbal formant. Verbal stems can be specific to plural subjects and distinguish for inanimate vs. animate, which is noted as \textsc{inan/an}, e.g., \textit{hiʉʔ}{{·}} (L) \textsc{inan} \textsc{pl} ‘to fall off’.
} The syllable structures given for this variety parallel those found in AMU as much as possible, but structures like CʔV or ChV are not attested for this variety.



\ea
\dobuibox{V \label{extab:dobui:8V}}
          {a (LH) \textsc{inan} \\
               \glt ‘to contain (liquid)’ (p. 2)
          }
          {ĩ (L) \textsc{an}\\
               \glt definite article (p. 58)
          }
          {\textbf{a·} (L) \\
               \glt ‘to fill’ (p. 2)
          }
          {\textbf{ã·} (H)\\
               \glt ‘worm’ (p. 2)
          }
\z

\ea
\dobuibox{ʔV}
     {ʔuɨ (H)\\
          \glt ‘black widow spider’ (p. 56)
     }
     {---}
     {---}
     {---}
\z


\ea
\dobuibox{hV}
     {hi (L)\\
          \glt ‘book’ (p. 73)
     }
     {hã (LM) \\
          \glt ‘foam’ (p. 68)
     }
     {\textbf{ha·} (H)  \\
               interrogative (p. 68)
     }
     {---}
\z


\ea
\dobuibox{Vʔ}
     {\gll aʔ~(LM) \\
          heart.2\textsc{sg.poss}\\
          \glt ‘your heart’ (p. 3)
     }
     {---}
     {---}
     {---}
\z


\ea
\dobuibox{ʔVʔ}
     {ʔuiʔ (H)\\
          \glt ‘flake’ (p. 56)
     }
     {---}
     {\gll \textbf{ʔiʔ·}~(L)\\
          \textsc{inan} \\
          \glt ‘to liquify’ (p. 56)
     }
     {\textbf{ʔuĩʔ·} (L)\\
     \textsc{inan}\\
               \glt ‘to get ground (specific to maize)’ (p. 56)
     }
\z

\ea
\dobuibox{hVʔ}
     {hɨʔ (L)\\
          \glt ‘light’ (p. 77)
     }
     {hĩʔ (L) \\
          \glt ‘orange (fruit)’ (p. 76)
     }
     {\gll \textbf{hiʉʔ·}~(L)\\
          \textsc{inan}~\textsc{pl}\\
          \glt ‘to fall off’ (p. 76)
     }
     {\textbf{hĩʔr·}~(L) \\
          \glt ‘to be alive’ (p. 77)
     }
\z

\newpage
\ea
\dobuibox{ʔLV}
     {ʔla~(LM)\\
     \textsc{inan} \\
          \glt ‘to bounce, pull’ (p. 118)
     }
     {---}
     {\textbf{ʔli·} (H) \\
          \glt ‘arch’ (p. 120)
     }
     {---}
\z


\ea
     \dobuibox{hLV}
     {hlɨ (HL) \textsc{inan} \\
          \glt ‘to be covered’ (p. 79)
     }
     {hmi (LH)\\
          \glt ‘skunk’ (p. 82)
     }
     {\textbf{hlʉ·} (L) \textsc{inan}\\
          \glt ‘welt’ (p. 80)
     }
     {\textbf{hma·} (L) \textsc{inan}\\
          \glt ‘to be wide’ (p. 80)
     }
\z



\ea
\dobuibox{hLVʔ}
     {hleʔr (L) \textsc{an} \\
          \glt ‘to tremble’ (p. 79)
     }
     {hmangʔ (M)\\
          \glt ‘pure’ (p. 80)
     }
     {\textbf{hloʔ·} (H) \textsc{inan}\\
          \glt ‘to be good, precious, beautiful’ (p. 79)
     }
     {\textbf{hmaʔ·} (M)\\
          \glt ‘things or animals distributed’ (\textsc{adv}) (p. 80)
     }
\z


\ea
\dobuibox{ʔLVʔ}
     {ʔleʔ (L) \textsc{inan}\\
          \glt ‘to decompose’ (p. 119)
     }
     {ʔleiñʔ (LM) \textsc{an}\\
          \glt ‘to push’ (p. 119)
     }
     {\textbf{ʔleʔ·} (LM)\\
          \glt ‘group’ (p. 120)
     }
     {\textbf{ʔnaʔ·} (H)\\
          \glt ‘disgusting, evil’ (p. 143)
     }
\z

\ea
\dobuibox{CV}
          {tu (L)\\
               \glt ‘chicken’ (p. 184)
          }
          {tũ (LM) \\
               \glt ‘guitar’   (p. 184)
          }
          {\textbf{tʉr·} (L) \textsc{inan}\\
               \glt ‘to abandon, leave’ (p. 186)
          }
          {\textbf{tũ·} (M)\\
               \glt ‘two’  (p. 184)
          }
\z

\newpage
\ea
\dobuibox{CVʔ \label{extab:dobui:8CVʔ}}
     {toʔ (L)\\
          \glt ‘shrimp’ (p. 183)
     }
     {tõʔr (LM)\\
          \glt ‘to mend something’ (p. 183)
     }
     {\textbf{toʔ·} (L) \textsc{inan} \textsc{sg} \\
               \glt ‘to fall’ (p. 183)
     }
     {\textbf{tõʔ·} (L) \\
          \glt ‘type of parakeet’ (p. 183)
     }
\z

Examples \xxref{extab:dobui:9V}{extab:dobui:9CVʔ} show the distribution of the ballistic vs. controlled contrast across different syllable structures with long vowels in Comaltepec Ch (\citealt{AndersenEtAl2021}).

\ea
\dobuibox{{V}\label{extab:dobui:9V}}
     {ee (L)\\
               \glt ‘early (in the morning)’ (p. 35)
     }
     {---}
     {\textbf{ee·} (LH)\\
          \glt ‘girdle, tape’ (p. 35)
     }
     {\textbf{ãã·} (LM) \\
          \glt ‘bridge’ (p. 2)
     }
\z

\ea
\dobuibox{ʔV}
     {---}
     {ʔuɨɨ (L) koo (L)\\
          \glt ‘strip, thread’ (p. 56)
     }
     {\textbf{ʔuɨɨ·} (H)\\
     \textsc{inan} \\
          \glt ‘to ache’ (p. 56)
     }
     {\textbf{ʔuĩĩ·} (M)\\
          \glt ‘far’ (p. 56)
     }
\z


\ea
\dobuibox{hV}
     {hee (LH) \\
          \glt ‘between, in’ (p. 70)\\
     hii (LH) \\
          \glt ‘tuber’ (p. 74)
     }
     {haaiñ (L) \\
          \textsc{exist} (p. 68) \\
     hĩĩ (L) \\
          \glt ‘year, season’ (p. 74)
     }
     {\textbf{hee·} (LM) kiaʔr{\textbf{·}} (M) \textsc{inan} \\
               \glt ‘to accompany’ (p. 70)
     }
     {\textbf{haaiñ·} (L) \textsc{an}\\
               \glt ‘to break a limb’ (p. 68)\\
     \textbf{hĩĩ·} (L)\\
               \glt ‘mud’ (p. 74)\\
     }
\z

\ea Vʔ      ---
\z

\ea ʔVʔ     ---
\z

\ea
\dobuibox{hVʔ}
     {---}
     {---}
     {\textbf{hɨɨʔ·} (LH) \textsc{inan} \\
          \glt ‘to be unfertile’ (p. 78)
     }
     {\textbf{hĩĩʔ·} (LH)\\
          \glt ‘only’ (p. 74)
     }
\z


\ea
\dobuibox{ʔLV}
     {ʔlee (L)\\
               \glt ‘powder’ (p. 119)
     }
     {---}
     {\textbf{ʔlee·} (M)\\
          \glt ‘soldier’ (p. 119)
     }
     {---}
\z

\ea
\dobuibox{hLV}
     {hlɨɨ (M) \textsc{an}\\
          ‘larva’ (p. 79)\\
     }
     {hmee (L) \textsc{inan} \\
          \glt ‘to stink’ (p. 81)
     }
     {\textbf{hlee·} (L) \textsc{inan}\\
          \glt ‘to tremble’ (p. 78)
     }
     {\textbf{hliiñ·} (L) \textsc{inan}\\
          \glt ‘to shuck, husk’ (p. 79)
     }
\z

\ea
\dobuibox{ʔLVʔ}
     {---}
     {ʔlɨiñʔ (H) \textsc{an} \\
          \glt ‘to be evil’ (p. 120)
     }
     {\textbf{ʔlooʔ·} (LH)\\
          \glt ‘cockroach’  (p. 121)
     }
     {\textbf{ʔleeiñʔ·} (LH) \textsc{an}\\
          \glt ‘to earn salary/pay’  (p. 119)
     }
\z

\ea
\dobuibox{hLVʔ}
     {---}
     {hmɨiñʔ (LM)\\
          \textsc{an}\\
          ‘to urinate’ (p. 88)
     }
     {\textbf{hlɨɨʔ·} (LH)\\
          \glt ‘stick, slit of firewood’ (p. 79)
     }
     {\textbf{hlúuiñʔ·} (LH)\\
     \textsc{an}\\
          \glt ‘to be lame’ (p. 80)
     }
\z

\newpage
\ea
\dobuibox{CV}
     {too (L) \\
          \glt ‘hole’ (p. 182)
     }
     {tõõ (LH)\\
     \textsc{inan}. \\
          \glt ‘to stand vertically’ (p. 182)
     }
     {\textbf{too·} (L) \\
          \glt ‘grinding stone’ (p. 182)
     }
     {\textbf{tõõ·} (ML)\\
          \glt ‘thorn’ (p. 182)
     }
\z



\ea
\dobuibox{\label{extab:dobui:9CVʔ} \textbf{CVʔ}}
     {---}
     {tuiñʔ (L)\\ \textsc{an}\\
          \glt ‘to be hunchbacked’ (p. 185)
     }
     {\textbf{tooʔ·} (LH) \textsc{noun} \\
          \glt ‘drop’ (p. 183)
     }
     {\textbf{tɨɨ} \textbf{ngʔ·} (LH)\\
          \glt ‘edge, shore’ (p. 181)
     }
\z



For the two sets of examples \xxref{extab:dobui:8V}{extab:dobui:8CVʔ} and \xxref{extab:dobui:9V}{extab:dobui:9CVʔ}, true minimal pairs and monomorphemic words are given where possible. \tabref{tab:dobui:summary2} summarizes these contrasts, where empty cells correspond to forms that were not found. Some syllable types are underpopulated or unpopulated. For example, ʔVʔ only exists with short vowels, and inversely ʔV is more commonly found with long vowels. For the Vʔ type, only one (multimorphemic) form was found.


\begin{table}
\caption{A summary of ballisticity distribution across syllable and phonation types in Comaltepec Ch}
\label{tab:dobui:summary2}
\fittable{
\begin{tabular}{l cc cc cc cc}
\lsptoprule
& \multicolumn{4}{c}{controlled} & \multicolumn{4}{c}{ballistic}\\
\cmidrule(r){2-5}\cmidrule(l){6-9}
Syllable structure  and phonation types & V &Ṽ &VV &ṼṼ &V &Ṽ &VV &ṼṼ \\
\midrule
V                                       & + & + & + & --& + & + & + & + \\
ʔV                                      & + & --& --& + & --& --& + & + \\
hV                                      & + & + & + & + & + & --& + & + \\
Vʔ                                      & + & --& --& --& --& - & --& --\\
ʔVʔ                                     & + & --& --& --& + & + & + & + \\
hVʔ                                     & + & + & --& --& + & + & + & + \\
ʔLV                                     & + & --& + & --& + & --& + & --\\
ʔLVʔ                                    & + & + & --& + & + & + & + & + \\
hLV                                     & + & + & + & + & + & + & + & + \\
hLVʔ                                    & + & + & --& + & + & + & + & + \\
CV                                      & + & + & + & + & + & + & + & + \\
CVʔ                                     & + & + & --& + & + & + & + & + \\
\lspbottomrule
\end{tabular}
}
\end{table}








Tonal inventories and their relation to ballisticity vary across Ch languages. In general, these languages have between three and five level tones, and three or more contour tones (\citealt{Rensch1968}; \citealt{Suárez1983}). For some languages, equivalent tonal inventories are attested both in ballistic and controlled syllables, e.g. in Palantla \citep{Merrifield1968}. In others, e.g. in Ozumacín \citep{Rupp2012}, fewer tones combine with ballistic syllables. The latter case appears to be more common. This is also the case for AMU ballistic syllables, which have one contour tone (LM) less than controlled syllables. On the other hand, in a Ch language Quiotepec, the LM contour occurs \textit{only} on ballistic syllables (\citealt{GardnerMerrifield1990}). While a reduced set of tonal contrasts in most other Ch languages and in AMU lend ballisticity a marked quality, in Quiotepec the inverse might be the case. A comparison of tone and ballistic interaction across these two language groups is given in \sectref{sec:dobui:3.3}.

Ballisticity is largely restricted to lexical morphemes, but it can also mark lexical verbal classes and verbal agreement. In \tabref{tab:dobui:10}, a review of data from existing literature\footnote{The data in \tabref{tab:dobui:10} is reproduced from a variety of sources with varying annotation conventions. Superscript numbers represent tones, where 1 is a high tone. Data from \citet{Rupp2012} reports tones as a system of diacritics.} shows that in every group of Ch, ballistic lexical roots are found (see the first column). As in AMU, morphological functions of ballisticity are also observed in Ch, but word formation (in the second column) is rare. In the third column, ballisticity marks verbal subject, nominal possession, or animacy agreement, alone or in combination with other prosodic contrasts, e.g. tone. In this column, the example for Quiotepec shows an independent first person singular subject pronoun following the volitive mood marker, which is controlled. In the absence of an independent subject pronoun, first person singular is marked on the mood marker with ballisticity and tonal alternation. As seen from the final column, TAM marking with ballisticity is attested in Palantla, Ozumacín, and Quiotepec, though in combination with a tonal change or a segmental change, as is the case for AMU (see the stem for ‘castrate’ \textit{kw̰a̰·} (L) and its plural marked form \textit{tw̰a̰} (LM)). Ballisticity in Lealao also occurs on verb stems marked for the 1\textsc{pl}.\textsc{fut} at a rate of 1.4\% in \citegen[14]{Palancar2015} quantitative review, meaning presence of ballisticity indicates the verb most likely does not belong to class I verbs, acting as a sort of negative identifier.


\begin{sidewaystable}
\caption{Distribution of ballisticity within the prosodic word in Chinantec varieties}
\label{tab:dobui:10}
\begin{tabularx}{\textwidth}{p{3.5cm}p{3.5cm}p{3.2cm}Qp{2.5cm}}
\lsptoprule
& {\bfseries Free morphemes} & {\bfseries Word formation} & {\bfseries Agreement} & {\bfseries TAM}\\
\midrule
´{\bfseries I – Usila}\newline { (\citealt{SkinnerSkinner2000}: 269, 192, 474)}
     & { pei¹\newline ‘extremity, tip’}\newline~\newline
     {\bfseries pei¹·}\newline { ‘small, little’}
          & { lia⁴}\newline { ‘like’}\newline~\newline
          {\bfseries lia⁴·}\newline { ‘as well’}
               & a²-kwan³-i³\newline  \textsc{class}-dress-3\textsc{sg.poss}\newline ‘her dress’\newline~\newline
               { a²-\textbf{kwan³⁻³·}}\newline \textsc{class}-dress.2\textsc{sg.poss}\newline ‘your dress’ & \\
\tablevspace
{\bfseries II – Palantla}\newline{ (\citealt{Merrifield1963,Merrifield1968})}
     & { li³}\newline { ‘tepejilote\newline (type of palm tree)’}\newline~\newline
     {\bfseries li³·}\newline { ‘flower’}
          &
               &
                    & { ka² gwe²-dʒa}\newline { ‘he slept’}\newline~\newline
                    { \textbf{gwe¹·}-dʒa}\newline {‘he will sleep’}\\
\tablevspace
{\bfseries III – Ozumacín}\newline { \citep[32-33, 235, 246]{Rupp2012}}
     & { d͡soo (mid)}\newline { ‘fault, disease’}\newline~\newline
          { \textbf{d͡soo·} (mid)}\newline{ ‘right’}
          &
               & { lle-y (mid)}\newline {head-3.}\textsc{poss}\newline {‘his/her/their head’}\newline~\newline
               \textbf{lle-n·} (low) \newline head-1\textsc{sg.poss}\newline ‘my head’
                    &  jɛɛy \newline (mid-rising)\newline  \textsc{prog}.look \newline ‘be looking’\newline~\newline
                    \textbf{jɛɛy·} (high)\newline \textsc{pst}.look\newline ‘looked’\\
\midrule
\end{tabularx}
\end{sidewaystable}
\begin{sidewaystable}
\begin{tabularx}{\textwidth}{p{3.5cm}p{3.5cm}lQp{3cm}}
\lsptoprule
& {\bfseries Free morphemes} & {\bfseries Word formation} & {\bfseries Agreement} & {\bfseries TAM}\\
\midrule
{\bfseries IV – Lalana}\newline { \citep[35]{Rensch1989}}
     & { tu²}\newline ‘belly button’\newline~\newline
          \bfseries tu·³\newline ‘tube’
          &
               &
                    & \\
\tablevspace
{\bfseries IV – Lealao}\newline (\citealt{Rensch1968}; \citealt{RuppRupp1996})
     & { ta³}\newline { ‘ladder’}\newline~\newline
          {\bfseries ta³·}\newline { ‘banana’}
          &
               & ma³-tiaʔ³\newline \textsc{perf}-fall.\textsc{inan}\newline ‘(it) has fallen’\newline~\newline
                ma³-\textbf{tiaʔ³·}\newline \textsc{<perf}-fall.3\textsc{sg.an}\newline ‘(s/he) has fallen’ & \\
\tablevspace
{\bfseries V – Quiotepec}\newline (\citealt{Rensch1989}: 35; \citealt{Robbins1968}: 91; \citealt{GardnerMerrifield1990})
     & { tu²}\newline  ‘belly button’\newline~\newline
      \textbf{tu²³·}\newline  ‘tube’
          &
               & nĩ·³ duʔ³ hnã²\newline go.1\textsc{sg} \textsc{vol} 1\textsc{sg}\newline  ‘I intend to go’\newline~\newline
               nĩ·³ \textbf{du¹²}\textbf{·}\newline go.1\textsc{sg} \textsc{vol}.1\textsc{sg}\newline ‘I intend to go’
                    &  ni³-ʔlaʔ¹\newline \textsc{pst}-push\newline ‘pushed’\newline~\newline
                         ni³ \textbf{ʔlaa¹·} ʔa\newline ‘you will push’\\
     \lspbottomrule
\end{tabularx}
\end{sidewaystable}

In Ch, stems are usually monomorphemic and monosyllabic, although pre- and post-positional functional morphemes may attach to create multimorphemic and multisyllabic words. Only one stem morpheme can be stressed (as marked with an acute accent). On this structure, ballisticity is described as contrastive only in stressed syllables, meaning only one syllable may be ballistic in the prosodic word (\citealt{Merrifield1963}; \citealt{Rensch1968}; \citealt{RuppRupp1996}; \citealt{Silverman2006}). However, functional morphemes (which do not carry stress) with lexically specified ballisticity do exist, e.g. in Lealao: \textit{-a²·} 1\textsc{sg} (\citealt{RuppRupp1996}) and in Ozumacín: \textit{ta}{{·}} (L) ‘by’ \citep{Rupp2012}. The former morpheme remains unchanged when attached to a lexically ballistic verbal stem, as in \REF{ex:dobui:15}. On the other hand, some other lexically ballistic verbal stems may become controlled if this morpheme is attached to them; compare \REF{ex:dobui:16a} with \REF{ex:dobui:16b}.

\ea\label{ex:dobui:15}
{\label{bkm:Ref102939767}\citet[10]{Palancar2015}}\\
{Ballisticity can occur more than once in a prosodic word} \\
\gll \textit{ʔi²-\textbf{hee²}\textbf{·}-\textbf{a\textsuperscript{4}}\textbf{·}}\\
     \textsc{fut}-run.over.something/one-1\textsc{sg}\\
\glt ‘I will run over something/one (animate).’
\z

\ea\label{ex:dobui:16}

{\label{bkm:Ref102939769}\citet[3]{Palancar2015}}\\
\ea\label{ex:dobui:16a}
{\label{bkm:Ref102939772}Lexically ballistic verbal stem}\\
     \textit{ʔi\textsuperscript{4}}\textit{-}\textbf{\textit{la yʔ}}\textbf{\textit{\textsuperscript{4}}}\textbf{\textit{·} }\\
\glt \textsc{fut}-get.an.entity.down \\
‘will get an (animate) entity down’

\ex\label{ex:dobui:16b}
{\label{bkm:Ref102939779}Lexically ballistic verbal stem with \textsc{1sg} marking}\\
     \textit{ʔi²}\textit{-la ʔ}\textit{\textsuperscript{42}}\textit{-}\textbf{\textit{a}}\textbf{\textit{²}}\textbf{\textit{·}}\\
     \textsc{fut}-get.an.entity.down-1\textsc{sg}\\
\glt ‘I will get an (animate) entity down.’
\z
\z

Additionally, in compounds, more than one lexical stem can be ballistic, as shown in \REF{ex:dobui:17}. Theoretically only one syllable can carry stress even in multimorphemic words (\citealt{GardnerMerrifield1990}), but in the examples below, stress is not marked in the sources.

\ea\label{ex:dobui:17}
\ea
{Ozumacín Ch (\citealt{Rupp2012}: 89, 40)}\\
     \textbf{\textit{lɑ·}}+\textbf{\textit{hʊʊ·}} (L+L) ‘all’\\
     \textbf{\textit{gyi}}·+\textbf{\textit{hwɨɨ}}{{·}} (M+H) ‘hell’\\

\ex
{Lealao Ch (\citealt{RuppRupp1996}: 208)}\\
     \textbf{\textit{la}}{{·}}+\textbf{\textit{ɲu}}{{·}} (H+M) ‘house with a grass roof’\\
\z
\z

These examples seem to contest the descriptions of ballisticity as necessarily related to stress both because ballisticity can occur more than once in a prosodic word (as in \ref{ex:dobui:15} and \ref{ex:dobui:17}), and because it is possible for a ballistic syllable to occur outside a stressed position (as in \ref{ex:dobui:23}). Further study of the distribution of ballisticity in Ch is needed to clarify the trait’s possible domains and whether secondary stress could be postulated.

To summarize, in Chinantec, the ballistic contrast:

\begin{itemize}
\item[(1)] is found mostly on the stressed syllable, but in some languages grammatical morphemes maintain lexical ballisticity in an unstressed position;
\item[(2)] occurs unpredictably on all syllable types;
\item[(3)] co-occurs with all types of laryngeal specifications and all other types of word-prosodic inventories;
\item[(4)] is lexically contrastive;
\item[(5)] irregularly marks animacy and subject agreement on verbs;
\item[(6)] is an index marker for at least one lexical verb class in Lealao.
\end{itemize}

\subsection{Comparative summary of ballisticity in Amuzgo and Chinantec}\label{sec:dobui:3.3}
\label{bkm:Ref124452992}
This section reviews and compares the properties of ballisticity in Am and Ch.

A synthesis of the data from AMU and Ch is given in \tabref{tab:dobui:11}.


\begin{table}
\begin{tabularx}{\textwidth}{lXc}
\lsptoprule
& { AMU} & { Ch}\\
\midrule
 Widely distributed across syllable structures           &  \langscicheckmark &  \langscicheckmark\\
 Co-occurs with other word-prosodic inventories         &  \langscicheckmark &  \langscicheckmark\\
 Occurs on stressed and unstressed syllables            &  \langscicheckmark &  (\langscicheckmark)\\
 Lexically contrastive on bound and lexical morphemes   &  \langscicheckmark &  (\langscicheckmark)\\
 Occurs more than once in a word                        &  \langscicheckmark &  (\langscicheckmark)\\
 Morphologically active                                 &  \langscicheckmark &  \langscicheckmark\\
\lspbottomrule
\end{tabularx}
\caption{Properties of ballisticity in AMU and Ch}
\label{tab:dobui:11}
\end{table}

\tabref{tab:dobui:11} shows that the trait has very similar functional properties in both language groups, despite some deviations (indicated by check marks in parentheses) in those Ch languages where the feature is restricted to lexical morphemes or occurs only once in the prosodic word. Such examples that contradict the mainstream tendencies have been found in Lealao (Group IV) and Ozumacín (Group III). On the other hand, no ballistic syllables have been observed outside the stressed position and more than once in the prosodic word in Palantla (Group II). For now, I will continue to consider that in general, the two language groups differ in the distribution of ballisticity across the prosodic word. While ballisticity is generally not contrastive outside of stressed positions in the word in Ch, its distribution in AMU is freer, and the trait can occur outside of stressed positions and more than once.

Other differences between Ch and AMU involve cross-distributions of the contrast with tonal inventories, although these vary from language to language rather than between the two language groups. As observed in the introduction, ballisticity has been considered in the literature both a syllabic level contrast and a sub-feature of tone: two separate tonal inventories, one controlled and one ballistic, have been proposed. In \tabref{tab:dobui:12}, level and contour tones in controlled and ballistic syllables are given for the Amuzgo varieties of San Pedro Amuzgos (SPA) and Xochistlahuaca (AMU) and in all five language groups of Chinantec. To facilitate comparison, I map the original notation found in the literature (given in parentheses) to the H, M, L convention. Tones that do not occur in both ballistic and controlled syllables are italicized.


\begin{table}
\begin{tabularx}{\textwidth}{l  QQQQ}
\lsptoprule
\phantom{MMM} & \multicolumn{2}{c}{{Controlled}} & \multicolumn{2}{c}{{Ballistic}}\\
\cmidrule(r){2-3}\cmidrule(l){4-5}
& {Level}  & {Contour}  & {Level}  & {Contour} \\
\midrule
\multicolumn{5}{l}{\bfseries Amuzgo varieties} \\
\midrule
\multicolumn{5}{l}{SPA \citep{Smith-StarkFermin-Tapia1984}}\\
& {\textit{H}, {M, L}\newline (5, 34, 12)} & {\textit{MH}\newline (35)} & {M, L\newline (3, 1)} & {\textit{HM, HL}\newline (53, 31)}\\

\tablevspace
\multicolumn{4}{l}{AMU \citep{Bauernschmidt1965}}\\
& {H, M, L\newline (1, 2, 3)} & {HL, MH, \textit{LM}\newline (13, 21, 32)} & {H, M, L\newline (1, 2, 3)} & {HL, MH\newline (13, 21)}\\
\midrule
\multicolumn{5}{l}{\bfseries Chinantec varieties}\\\midrule
\multicolumn{4}{l}{Group I: Tlacoatzintepec \citep{Thelin1980}}\\

& {H, MH, M, L\newline (1, 2, 3, 4)} & {ML, LMH\newline (34, 42)} & {H, MH, M, L\newline (1, 2, 3, 4)} & {ML, LMH\newline (34, 42)}\\

\tablevspace
\multicolumn{4}{l}{Group II: Palantla \citep{Merrifield1963}} \\

& {H, M, L\newline (3, 2, 1)} & {LH, LM, HL\newline (13, 12, 31)} & {H, M, L\newline (3, 2, 1)} & {LH, LM, HL\newline (13, 12, 31)}\\

\tablevspace
\multicolumn{4}{l}{Group III: Ozumacín \citep{Rupp2012}}\\

& {H, M, L
\newline ({{alto, medio, bajo}})}
& \textit{HH, MH, LH
\newline ({alto ascendente, medio asc., bajo asc.})}
& {H, M, L
\newline ({{alto, medio, bajo}})} & \\

\tablevspace
\multicolumn{4}{l}{Group IV: Lealao \citep{Rupp1990}}\\
& {High H, H, M, L\newline (1, 2, 3, 4)} & {\textit{MH, LH}\newline (32, 42)} & {High H, H, M, L\newline (1, 2, 3, 4)} & \\

\tablevspace
\multicolumn{4}{l}{\quad Group V: Quiotepec \citep{GardnerMerrifield1990}}\\
& {H, M, L} & {LH, ML, MH} & {H, M, L} & {\textit{LM}, LH, ML, MH}\\
\lspbottomrule
\end{tabularx}
\caption{Tonal inventories of selected Am and Ch varieties}
\label{tab:dobui:12}
\end{table}

For almost all the reviewed languages, tones are equally contrastive in ballistic and controlled syllables, although there are some notable distributional differences. In SPA, high tone does not occur with ballistic syllables. Elsewhere, all level tone inventories are found in both controlled and ballistic syllables. Contour tones vary slightly more, at times co-occurring more often with controlled syllables than with ballistic syllables. For example, the LM contour tone in AMU is not found in ballistic syllables. More striking examples are Lealao and Ozumacín, where no contour tones co-occur with ballistic syllables. The inverse, where more contour tones co-occur with ballistic syllables than with the controlled ones, can also be true, as in SPA and Quiotepec. In Sochiapan, MH and HL are noted as being perceived by speakers as ballistic in CV syllable structures, but as controlled in CVʔ. The remaining contour tone LM is perceived by speakers as both controlled and ballistic in free variation for both CV and CVʔ syllables \citep{Foris1973}.

\newpage
\section{Discussion}\label{sec:dobui:4}
\largerpage
In this section, a typological account of ballisticity as word prosody is first presented. Then other realizations of laryngeal features in AMU and Ch are analyzed in order to reach a wider understanding of laryngeals as a system in these languages.

\subsection{Prosodic typology}\label{sec:dobui:4.1}

Traditionally ballisticity has been referred to as a form of stress, thus syllabically anchored, and with separate sets of tone inventories. However, \citet{Silverman1994} and  \citet{Zendejas2000} analyze ballisticity as a segmentally anchored floating feature “spread glottis” [sp]. Consequently, I will address the question of whether or not [sp] as an autosegment is a form of stress, and why it cannot be understood as a different set of tones. 

\hspace*{-3.3pt}In word-prosodic typology, a number of types of word prosody from the world’s languages are “analytically indeterminate”, leading \citet{Hyman2006} to rule out the tripartite division of tone, pitch accent, and stress. Instead, the author recommends a parameter-based approach for analyzing prosodic systems over a fixed word-prosodic taxonomy. Under this approach, stress and tone are at opposing ends of a privative scale of characteristics by which word prosodic systems can be compared (p. 229--231):

\begin{itemize}
\item[(1)] stress is syntagmatic while tone is paradigmatic;
\item[(2)] stress is both obligatory (i.e. every lexical word has at least one stressed syllable) and culminative (i.e. every lexical word has at most one main stressed syllable);
\item[(3)] tone is primarily defined by pitch movement, can be non-obligatory e.g. in inventories with few tonal contrasts, and can attach to a number of tone-bearing units.
\end{itemize}

Despite phonetic correlates similar to stress, this approach allows us to rule out [sp] as an archetypal form of stress, given that in both Ch and AMU, [sp] is not obligatory. On the other hand, in some Chinantec languages [sp] appears culminative because it is restricted to the stressed position of the prosodic word. The Ch word most often consists of one lexical root surrounded by prefixes and suffixes, with stress hosted by one syllable of the lexical root. It is then unsurprising that [sp] most often appears in stress position, but not exclusively. Still, the distribution of the feature in the two language groups indicates differences in the dependency domains. In AMU and some Ch languages, [sp] is morpheme-dependent, occurring multiple times in the word domain. 

\largerpage
While [sp] is clearly not stress in AMU and in at least some Ch languages, whether or not it is a form of tone is less clear. In both languages, the sets of tone have also traditionally included the ballistic parameter with so-called ballistic syllable tones usually mirroring the controlled syllable tones. \citet{Kim2011} suggests considering ballisticity as a “high“ register of tones versus a “low” register (the latter corresponding to ``controlled''), wherein ballisticity is realized as a result of conditioned tone alternations. This is an interesting idea that only captures part of the picture. Considering ballisticity to be uniquely tone-based leaves out any understanding of final devoicing in relation to other laryngeal features, and does not address the morphological functions observed in both languages. However, if tone and [sp] are taken as independent of one another, as I have done here, then effects on tone realization, e.g. shortening of modal voicing, can be attributed to [sp] rather than the other way around.

Still, if taken as altogether different tonemes, the tones with ballisticity could be thought to engage [sp] or final devoicing as a means for perceptual enhancement, as an articulatory consequence, or even as an inherent feature of tone in these languages. This configuration is not without precedent in the languages of the world. In White Hmong (Hmong-Mien), breathy and creaky voice co-occur with one lexical tone each. \citet{Garellek2013} find that while creaky voice is unnecessary for speakers to identify the “low-falling” tone, breathy voice is necessary for the perception of “mid-to-high” tone. This indicates that breathiness is an inherent phonetic feature of this second tone. The authors point out that in other laryngeally complex languages, where phonation and tone combine, the perception of some tones can be dependent on phonation (like breathiness in White Hmong), while for other tones pitch is the main cue (as with the low-falling tone and its optional creakiness in White Hmong). Other combinations of tones and phonations exist as well, where two-way or three-way phonation contrasts can be exploited in different ways in tone realization. 

The case of White Hmong and the other languages discussed in \citet{Garellek2013} differ, however, from Am and Ch. The former languages exploit phonation in only a restricted set of tones and contours. There are no full or nearly full sets of tonal melodies contrasting only in phonation, as is clearly the case with Am and Ch, which have tonal inventories that can be both accompanied and unaccompanied by [sp]. The two sets of tones are either equivalent or near equivalent in number. This helps to clarify that [sp] would not be better understood as tone, or as a property of tone. The feature [sp] acts as a privative and additional parameter applied in parallel to the tonal set. 

This view of [sp] and tone interaction is only in line with the part of \citegen{Kim2011} proposal that “ballistic tones” should not be considered as a separate set of tones. Following the proposal put forward in the present study, what Kim calls “register” can be understood as the co-occurrence of separate features [sp] and (most) tones, rather than as an interdependent property. 

More similarities can be found between [sp] and independent word-prosodic features like the Danish stød or Udihe (Tungusic) glottalization. As described in \citet[42--44]{Kuznetsova2022}, the features are independent of pitch accent, although some phonetic effects like pitch dip can occur. The features are widely distributed across vowel and syllable types, are lexicalized and unpredictable, and also have morphemic expression. All these are also true for [sp]. Contrary to [sp], however, these laryngeal word-prosodic features are realizations of [constricted glottis]. Also, the distribution of glottalization in Udihe is restricted to long vowels, and the frequency of occurrence for the respective features in Danish and Udihe differ. 

Turning back to Hyman’s approach to prosodic typology, let us note that tone is defined as paradigmatic and featural, and a language with tone is “one in which an indication of pitch enters into the lexical realization of at least some morphemes” (2006: 229). Under this definition, ballisticity in Am approaches prototypical tone in that it is paradigmatically applicable to every syllable. On the other hand, Ch appears to present ballisticity syntagmatically as ‘word tone’ where the word is a major tone-bearing unit. Hyman’s typology characterizes this latter type under tone systems with a syntagmatic quality typical of stress, rather than of prototypical tone.

Under \citegen{Hyman2006} approach, [sp] has prototypical characteristics of neither stress nor tone, though in Am it displays tone-like properties, while in Ch it displays stress-like properties. Ballisticity is thus prosodic by its functional properties but it is not really stress or tone. From here, I tentatively follow the autosegmental interpretation. As intended by the author, Hyman’s parameter-based approach to prototypical definitions of stress and tone allows for a detailed discussion of possible properties of the feature leading to a more nuanced understanding.

\subsection{Laryngeal prosodies and phonemes}\label{sec:dobui:4.2}
\label{bkm:Ref133780325}
The previous typological discussion of ballisticity shows that this feature is independent of tone and stress. Given that its major indicator is glottal abduction typical of aspiration or breathy voice, this section looks at other realizations of laryngeal features found in Am and Ch that co-occur with ballisticity. 

\subsubsection{Laryngeal features in AMU}

For the inventories of the described languages, I have asserted that vowels can be breathy or creaky and sonorants aspirated or glottalized (cf. \sectref{sec:dobui:3.1}). This is a departure from previous accounts of /h/ and /ʔ/ as full phonemes (\citealt{Bauernschmidt1965,ApóstolPolanco2014}). I analyze these instead as non-segmental instances of non-modal phonation realized on voiced segments, except for /ʔ/ when it is found in coda position. For example, a form noted as \textit{ɳhã} (L) ‘here’ \citep[302]{Bauernschmidt2010} is phonologically represented in this chapter as /ɳã̤/. Similarly, aspirated sonorants noted as e.g. \textit{hɲᵈʲa} M) ‘chest, thorax’ \citep[127]{Bauernschmidt2010} are phonologically represented as /ɲ̤a/.\footnote{Forms like this are represented as /ɲ̤a/ (and not */ɲ̤ᵈʲa/) given that post-nasal occlusion is an allophonic alternation that blocks nasal assimilation of oral vowels, a process called “nasal shielding” (see \citealt{Dobui2021}).} The reasons for such an interpretation are outlined below.

First, both /h/ and /ʔ/ are always found before voiced consonants (sonorants) and vowels and never before stops. Upon observation of this same distribution in Jalapa Mazatec (another Otomanguean language), \citet{GolstonKehrein1998} propose that this distribution shows a dependence on voicing, one better understood as non-modal phonation (laryngealization\slash aspiration) on vowels and sonorants. This distribution also coincides with what \citet{Silverman1997b} calls “phasing”. Languages that display phasing have complex prosodic inventories that combine contrastive tone, nasalization, and non-modal phonation. Given the high load of articulatory targets and sometimes contradictory demands on articulation, targets are often sequenced in time, or “phased”. In these languages, non-modal phonation is at least partially asynchronous of voicing, occurring to the left of vowels before modal voice, which is needed for production of contrastive tone. 

Second, there is language specific evidence in the phonology and morphology of Am for the non-segmental interpretation of laryngeals that are found before voiced segments. For example, in Am phonology, both laryngeals are unable to block an automatic alternation where high front /i/ undergoes diphthongization after front consonants. For instance, \textit{ti} (L) ‘bud’ is realized as [təi]. This same effect is illustrated in \REF{ex:dobui:18}, where the lexical verb stem in \REF{ex:dobui:18a} is marked for past tense by an alveolar stop prefix /t/ in \REF{ex:dobui:18b}, also triggering initial consonant loss. The high front /i/ in \REF{ex:dobui:18a} is diphthongized in \REF{ex:dobui:18b} after prefixation of /t/ rather than surfacing as \textit{*}[ti̤ʔ]. Under a phonemic analysis of the laryngeal in \REF{ex:dobui:18a} as \textit{kʷ}\textit{hiʔ} rather than \textit{kʷ}\textit{i̤ʔ}, it would be difficult to explain why \textit{h} is unable to block diphthongization in \REF{ex:dobui:18b}. The inability of laryngeals to block this automatic alternation supports an interpretation of laryngeals as non-phonemic.

\ea\label{ex:dobui:18}
{ \citet[565]{DeJesúsGarcía2004}}\\
\ea\label{ex:dobui:18a}
\gll {/k}{ʷ}{i̤ʔ/ (M)}\\
  {[k}{ʷ}{i̤ʔ]}\\
     ‘choose’\\
\ex\label{ex:dobui:18b}
  {/ti̤ʔ/} {(M)\\
{}[tə̤i̤ʔ]} \\
 \textsc{pst-}{choose}\\
 \z
 \z

More evidence is found in the morphophonology of nominal plurals. When the plural prefix /n/ attaches to a nominal stem with two consonants in the onset, a language-internal rule against C\textsubscript{1}C\textsubscript{2}C\textsubscript{3} initial syllables (unless C\textsubscript{3} is a glide) triggers deletion of C\textsubscript{1}, as in the morpheme for ‘knee’ in \REF{ex:dobui:19}.

\ea\label{ex:dobui:19}
{ \label{bkm:Ref111916202}\citet{JP2021}}\\
\ea
  \textbf{ʃ}tʲoʔ· (M) \\
\glt ‘knee’
\ex
\gll ka-\textbf{ɲ}tʲoʔ·~(M-M)\\
     \textsc{anim}-\textsc{pl}-knee\\
\glt ‘knees’
\z
\z

This is not the case in Cʔ/h initial stems. After prefixation, the sequences CCʔV and CChV are permitted, as in \REF{ex:dobui:20}, which again indicates that neither laryngeal is segmental and thus does not comport a violation of *C\textsubscript{1}C\textsubscript{2}C\textsubscript{3}.

\ea\label{ex:dobui:20}
{ \label{bkm:Ref111916220}\citet{JP2021}}\\
\ea
  \textbf{s}tʔəĩ (H) \\
\glt ‘vulture’
\ex
\gll ka²-\textbf{n}tʔəĩ~(H)\\
     \textsc{anim}-\textsc{pl}.vulture\\
\glt ‘vultures’
\ex
  shã·~(HL)\\
\glt ‘lark’
\ex
\gll n-shã·~(HL)\\
     \textsc{pl}-lark\\
\glt ‘larks’
\z
\z

From the above evidence, these laryngeals cannot be analyzed as full consonantal segments contrary to how they have been treated in previous literature. However, other possible analyses remain. Laryngeals could be (i) part of the nucleus, i.e. \textit{CV̤}, \textit{CV̰ } or (ii) part of the consonant, i.e. as unisegmental \textit{Cˀ}, \textit{Cʰ}. In other words, is \textit{shã· (HL)} ‘lark’ underlyingly /sã̤/ or /sʰã/? 

Again, alternations seen in the morphophonology show these laryngeal features to be dependent on adjacent voiced segments, as described in (i), and not secondary articulations of consonants, as in (ii). In \REF{ex:dobui:18b} above, when the past tense marker /t/ is prefixed to this class of stems, the initial consonant elides but aspiration remains. The same behavior is found across all stems of this class (i.e. where elision occurs) that have laryngeal features. \textit{CʔV} and \textit{ChV} always become \textit{t-ʔV} and \textit{t-hV} when marked by the past prefix and never *\textit{t-V}. This means that Cʔ and Ch do not act unisegmentally. Moreover, the laryngeals realized after prefixation are indeed those of the stem and not of the aspect marker itself, which is /t/ and not i.e. *\textit{tˀ} or *\textit{tʰ}. This can be seen when \textit{ku+tʲe·} (M+M) ‘distribute’ becomes \textit{tu+tʲe·} after past tense marking, and not e.g. *\textit{tˀu+tʲe·}. The morphophonology of plural nominals also provides more evidence that \textit{Cʔ} and \textit{Ch} are not unisegmental. For example, \textit{tsʔã} (H) ‘tail’ becomes \textit{nʔã} ‘tails’ and not *\textit{nã} (see also \ref{ex:dobui:19} and \ref{ex:dobui:20}). These patterns hold across both paradigms.

There remains the question of whether in laryngeal-initial syllable shapes (i.e. \textit{h(L)V} and \textit{ʔ(L)V}, where L is a sonorant) glottals are segmental or voice-dependent features. In the morphophonology of past prefixation, another inflectional class of these syllable shapes is marked by t-prefixation without any change to the stem (as opposed to \ref{ex:dobui:18}). Thus in \REF{ex:dobui:21a}, the laryngeal features are maintained after prefixation of the past marker, much like in other stems of different shapes. Examples of this kind are rare but indicate that laryngeals are a dependent part of the adjacent voiced segment. In \REF{ex:dobui:21b}, however, the laryngeal feature is not realized when the nominal plural marker /n/ is prefixed.

\ea\label{ex:dobui:21}
\ea\label{ex:dobui:21a}
{\label{bkm:Ref124497611}\citet[476, 491]{DeJesúsGarcía2004}}\\
\gll t-huʔ·(L)\\
     \textsc{pst}-expel; throw away\\
\gll t-ʔma\\
     \textsc{pst}-smoke\\

\ex\label{ex:dobui:21b}
{\label{bkm:Ref124497613}\citet[6]{Bauernschmidt2010}}\\
ka²-hnõ~(M-H)\\
     \textsc{class}-owl\\

ka²-nõ~(M-H)\\
     \textsc{class}-\textsc{pl}.owl\\
\z
\z

There are multiple interpretations for the alternation in \REF{ex:dobui:21b}. One is that the plural prefix /n/ triggers elision of initial unisegmental /n̤/, leaving behind only the nasal prefix: \textit{n-n̤V} → \textit{n-V}. A second interpretation of \REF{ex:dobui:21b} is that a phonetic effect resulting from concatenation of /n-n̤V/ or /n-hnV/ results in loss of [spread glottis] or \textit{h} in an intervocalic environment, followed by elision. A third alternative is that the ban against CCC triggers loss of \textit{h} as in: \textit{n-hnV} → \textit{n-nV} → \textit{n-V}, as above in \REF{ex:dobui:20}.

All cases appear to call upon resolution of two identical consonants surfacing. This effect is seen with other prefixes in the language. For example, the verbal stem \textit{mẽĩʔ}{{·}} (L) hit.3\textsc{sg} takes the plural verb prefix /t/ and surfaces as \textbf{\textit{t-}}\textit{mẽĩʔ} (LM) \textbf{\textsc{pl}}\textsc{-}hit.3\textsc{pl} (\citealt{DeJesúsGarcía2004}: 505). But t-initial stems like \textit{tãʔ} (L) \textsc{leave.3}\textsc{sg} surface unchanged after t-prefixation as \textit{t}\textit{ãʔ} (L) \textit{<}\textbf{\textsc{pl}}.leave.3\textsc{pl>} (\citealt{DeJesúsGarcía2004}: 498).

The first interpretation would support the argument that \textit{hn} is underlyingly \mbox{/n̤/}. The laryngeal feature does not surface because it is dependent on the stem-initial nasal sonorant that itself is elided to avoid homorganic CC. The second and third interpretations require a number of steps to derive surface forms. These do not rule out \textit{h} as phonemic given that this segment could have been deleted as an initial C, an effect that is found elsewhere in nominal plurals. In this particular environment, the status of initial laryngeals is unclear. Still the most succinct interpretation (i.e. implicating fewer rules) is the first one. It is also the one supporting the analysis of laryngeals proposed here. Other alternatives would require more study. In any case, they also demand a more elaborate analysis that assumes that initial and non-initial laryngeals should be treated differently. 

When found to the left of voicing, laryngeals are taken not as phonemes but rather as realizations of [spread glottis] and [constricted glottis], meaning instances of non-modal phonation. Still, for ease of exposition this chapter notes [h] and [ʔ] in transcription throughout the text. However, when either of them is found to the left of a sonorant or vowel, they respectively indicate aspiration and laryngealization, or breathiness and creakiness.

The final type of laryngeal feature is [ʔ] in the stem coda position, i.e. to the right of voicing. I analyze this final /ʔ/ as a phoneme in AMU, although one with a highly restricted distribution, i.e. occurring uniquely in the coda. Evidence of its status as segmental can be found in verbal subject marking. The first person singular is realized as \textit{=a·} after stems with \textit{ʔ} in final position like \REF{ex:dobui:22}. After stems without \textit{ʔ} in final position like \REF{ex:dobui:22b}, the realization of the first person singular is \textit{=ja·}. In \REF{ex:dobui:23}, the glottal-final stem is marked by the second person plural marker =\textit{joʔ·}.

\clearpage
\ea\label{ex:dobui:22}
{\citet[6, 56]{ApóstolPolanco2014}}\\
\ea\label{ex:dobui:22a}
\gll ma-kʷa\textbf{ʔ}·=a·~(M-H=M) \\
     \textsc{prog.sg}-eat.\textsc{1sg=1sg}\\
\glt ‘I am eating.’\\
\ex\label{ex:dobui:22b}
\gll ma-tʲhɛ·=\textbf{j}a·~(M-H=M) \\
     \textsc{prog.sg}-cut.\textsc{1sg=1sg} \\
\glt ‘I am cutting.’\\
\z
\z

\ea\label{ex:dobui:23}
{\citet[7]{ApóstolPolanco2014}}\\
\gll kʷi-kʷa\textbf{ʔ}·=joʔ·~(M-L=M)\\
     \textsc{prog.pl}-eat.\textsc{2pl}=2\textsc{pl}\\
\glt ‘You (pl.) are eating.’
\z

These data can be taken to show that a final glottal is indeed phonemic, rather than a laryngeal feature dependent on the preceding vowel. This interpretation depends on the underlying representation of the first person singular as /a·/ and not e.g. as /ja·/. If /ja·/ were the underlying representation, one might assume a rule against CVC.CV words that triggers alternation of /ja·/ to /a·/ in favor of CVC.V. However, forms like the one in \REF{ex:dobui:23} show that there is no problem for glottal final stems to be followed by another consonant word-internally, i.e. in CVC.CV. Therefore, the first person singular marker is interpreted as /a·/. In turn, this indicates a word-level well-formedness rule preventing CV.V from surfacing in \REF{ex:dobui:22b} (e.g. *\textit{ma-tʲ}\textit{hɛ·=a·}). The form [ja·] is realized instead, forming CV.CV. This means that the final glottal is phonemic, as any other interpretation, i.e. \textit{CV̰ } or \textit{CVˀ} instead of \textit{CVʔ}, means that \REF{ex:dobui:22a} would be an illicit word form: *CV̰.V (*\textit{kʷa̰·=a·}) or *CVˀ.V (*\textit{kʷaˀ·=a·}) rather than a case of CVC.V (\textit{kʷa}\textbf{\textit{ʔ}}\textit{·=a·}), as interpreted here.

\subsubsection{Laryngeal features in Ch}

In Chinantec, glottal fricatives and stops are also transcribed in the literature as consonants preceding sonorants or vowels. Glottal stops are also transcribed in coda position. Earlier accounts (\citealt{Andersen1989}; \citealt{Westley1991}; \citealt{Macaulay1999}) have contrasted three types of word-initial sonorants: “the plain series, the voiceless series, and the glottalized series” \citep[76]{Macaulay1999}. These resemble non-modal phonations described above for AMU and their phasing before the modal voicing period. They can be phonologically represented as in \REF{ex:dobui:24}. \label{bkm:Ref133676897}Additionally, this voicing-dependent distribution of laryngeals is reconstructed for Proto\hyp Chinantec (\citealt{Rensch1968,Rensch1989}).

\ea\label{ex:dobui:24}
{Italicized forms from Ojitlán \citep[76]{Macaulay1999}}\\
  ʔmĩ¹ã² → {/m̰ĩ¹ã²/} \\
  ‘to sew’
\z
\ea
  ʔja² → {/j̰a²/} \\
\glt ‘to griddle’
\z

No other onsets with \textit{h} and \textit{ʔ} (i.e. \textit{CʔV} or \textit{ChV}) are attested in e.g. Ozumacín \citep{Rupp2012}, Lealao (\citealt{RuppRupp1996}), or Comaltepec \citep{AndersenEtAl2021}. Additionally, all other prevocalic positions are either single consonants or consonants and glides. The latter are considered by \citet[10]{Rupp2012} to be part of diphthongs. Considering \textit{h} and \textit{ʔ} as voicing-dependent in Ch rather than as full segments also has the advantage of eliminating the only apparent initial consonant clusters in Ch, like those in \REF{ex:dobui:24}.

As for glottal stops found in coda position, observations from noun possession and verb subject inflection point to their phonemic status. In the variety from Ozumacín, possession of unalienable nouns is marked by segmental and prosodic suffixes (in bold in \ref{ex:dobui:25}). First person possession is marked by \textit{-n,} [sp], and low tone. Second person is marked by a final glottal stop, and third person is marked by \textit{-y} [i]. Unalienable nouns are body parts or nouns considered close to humans. They do not surface without possession, meaning no unmarked forms exist. In \REF{ex:dobui:26}, subject is marked on the verbs ‘to fry (inanimate things)’ and ‘to collect (inanimate things)’ with similar forms. First person subject is \textit{n-}, second person is the glottal stop, and third person is \textit{-y.}

\ea\label{ex:dobui:25}
{\label{bkm:Ref133742628} \citet[235--236]{Rupp2012}}\\
\ea\label{ex:dobui:25a} lle\textbf{n}\textbf{·} \textbf{(L)} {‘my head’} \\
lle\textbf{ʔ} (M) {‘your (}\textsc{sg}{/}\textsc{pl}{) head’}\\
lle\textbf{y} (M) {‘his/her/their head’}\\
\ex\label{ex:dobui:25b} tuʔ\textbf{n·} \textbf{(L)} {‘my stomach’}\\
tu\textbf{ʔ}· (H) {‘your (}\textsc{sg}{/}\textsc{pl}{) stomach’}\\
tu\textbf{y}ʔ· (H) {‘his/her/their stomach’} \\
\z
\z

\ea\label{ex:dobui:26}
{\label{bkm:Ref133775649}\citet[248]{Rupp2012}}\\
\ea\label{ex:dobui:26a} tʃɪɪ\textbf{n} (MH) {‘I fry (inanimate) things.’}\\
tʃɪɪ\textbf{ʔ}{·} (M) {‘You (}\textsc{sg}{/}\textsc{pl}{) fry (inanimate) things.’}\\
tʃɪɪ\textbf{y} (MH) {‘S/he/they fry (inanimate) things.’}\\
\ex\label{ex:dobui:26b} hɪɪʔ\textbf{n} (M) {‘I collect (inanimate) things.’}\\
hɪɪ\textbf{ʔ} (M) {‘You (}\textsc{sg}{/}\textsc{pl}{) collect (inanimate) things.’}\\
hɪɪ\textbf{y}ʔ (MH) {‘S/he/they collect (inanimate) things.’}\\
\z
\z

\citet[236]{Rupp2012} observes that while the third person possessive suffix attaches stem-finally in coda-less stems (\ref{ex:dobui:25a} and \ref{ex:dobui:26a}), in a lexically glottal final stem (\ref{ex:dobui:25b} and \ref{ex:dobui:26b}), \textit{-y} attaches to the left of the glottal stop. One way to account for this is to look at the syllable shapes that are realized. In \REF{ex:dobui:25b} and \REF{ex:dobui:26b}, \textit{-y}, which is realized as [i], cannot attach in final position, as this would result in multiple syllables: * \textit{hɪɪʔy} i.e. CVCV. By maintaining the righthand alignment of the glottal stop, a licit CVVC surfaces instead. This points to the phonemic status of the glottal stop.

However, in the first person examples in \REF{ex:dobui:25b} and \REF{ex:dobui:26b}, the final glottals are treated differently than in the third person. In the first person examples, the \textit{-n} suffix attaches without further segmental change to the stem instead of the glottal stop staying right-aligned. Under the understanding that Ozumacín has post-syllabic nasals \citep[30]{Rensch1968}, the morpheme \textit{-n} is not homosyllabic when attaching to glottal-final stems. While these forms do not trouble the analysis of /ʔ/ as phonemic, neither do they bolster the argument.

At the very least, glottal stops in coda position are treated as phonemes in third person possession in unalienable nouns and subject marking on verbs. Similarly to AMU, the non-segmental laryngeals to the left of voiced segments are still transcribed below as \textit{h} and \textit{ʔ}, following tradition. 

\subsection{Phonological status of ballisticity}

This section addresses how ballisticity fits in with the realizations of laryngeal features discussed in \sectref{sec:dobui:4.2} and outlines possible directions for further research.

It is clear that ballisticity, or [sp], acts differently than the laryngeal features explored in \sectref{sec:dobui:4.2}, despite all these traits being laryngeally-based. First, [sp] is realized voicing-finally, while the laryngeal features described in \sectref{sec:dobui:4.2} are realized pre-vocalically as \textit{hV/L} and \textit{ʔV/L} in Ch, and as \textit{(C)hV/L} and \textit{(C)ʔV/L} in Am. Both sets are dependent on voiced segments, and the prevocalic laryngeal features are treated as voicing features.

Contrary to Silverman and Herrera Zendejas, however, [sp] should be treated as attaching to the syllable rather than to a vowel segment. The difference in its status from that of other laryngeal features is seen in how it is realized. Contrary to the other features, ballisticity combines with tonal realization. As previously noted, it is also linked with fortis consonants, which indicates a syllabic level of application. Importantly, ballisticity co-occurs with the other laryngeal features. Under a segmental understanding of ballisticity, it should not be able to attach to the same segment which already carries a contradictory laryngeal value. A syllabic understanding of ballisticity resolves this problem. 

By generalizing the different distributions and statuses of laryngeal features into an abstract model, their entire inventory can be thought of as a three-way phonation contrast somewhat symmetrically distributed to the left and the right of voicing. For both language groups, laryngeal features that attach to the left of voicing are left-aligned. For AMU, this is an alignment either to a sonorant or a vowel position, but never to both at the same time. For Ch, laryngeal features are truly left-aligned, given that \textit{Ch/ʔV} never occur. For both languages, non-modal phonation on the left edge is privative and paradigmatic: only one realization can occur at a time. 

On the right-edge in both language groups, [sp] and glottal stop are right-aligned, with the specification that [sp] is voicing-dependent and syllabic and the glottal stop is segmental. Contrary to the other laryngeal features, these two types can co-occur in both languages. Analyzing their individual statuses as non-segmental for [sp], and segmental for [ʔ] accounts for this non-privative distribution.

The modeling of the laryngeal features as two sets, one realized on the left edge and another on the right edge, accounts for their co-occurrence in one syllable and syntagmatic distribution. And finally, the absence of any laryngeal feature constitutes modal phonation, the third value in the three-way contrast. 

Some open questions remain, in particular how to reach a full theoretical understanding of laryngeals in these languages, where prosodic inventories are so rich. An area of inquiry that remains to be further explored is the relation of ballisticity to contrastive length suggested for Jalapa Mazatec in \citet{SilvermanEtAl1995} and explored acoustically for AMU in \citet{ApóstolPolancoforthcoming}. Contrastive length is found in Ch, but not in Am, according to the literature. This raises the question for Ch of how ballisticity interacts with short and long syllables, and whether in Am ballisticity also bears some special relation to duration.

\section{Conclusion}\label{sec:dobui:5}

The study brings together a wide set of data across a number of Chinantec and Amuzgo languages to establish the phonological and morphological properties of ‘ballisticity,’ a trait thought to be specific to these Otomanguean languages. Previous phonetic characterizations of the feature give acoustic aperiodicity and articulatory abduction as its primary correlates, which makes it better understood as a laryngeal feature [sp]. This study combines this insight together with an interpretation of other language-specific laryngeal features. It tentatively suggests that both languages have a three-way modal-breathy-creaky contrast. Under this contrast, laryngeal features are fully contrasted in the positions to the left and to the right of voicing, although the features have different phonological statuses. The ``ballistic'' feature is but one expression of a richly exploited set of laryngeal features in these languages.

\section{Abbreviations}

\begin{tabularx}{.45\textwidth}{lQ}
= &  clitic\\
{\textasciitilde} &  reduplicated morpheme\\
\textsc{adv} &  adverbe\\
Am & Amuzgo language group\\
AMU & Amuzgo variety from Xochistlahuaca, Guerrero\\
\textsc{an} &  animate \\
\textsc{anim} &  animate class marker\\
C & consonant\\
\textsc{caus} &  causative\\
Ch & Chinantec language group \\
\textsc{class} &  class marker\\
\textsc{comp} &  complementizer\\
\textsc{conj} &  conjunction\\
ELA & Endangered Language Alliance\\
\textsc{et} &  extended theme\\
\textsc{excl} &  exclusive\\
\textsc{exist} &  existential pronoun\\
\end{tabularx}
\begin{tabularx}{.45\textwidth}{lQ}
\textsc{fruit} &  fruit noun class marker\\
\textsc{fut} &  future\\
G & glide\\
H & high tone\\
\textsc{hum} &  human\\
\textsc{inan} &  inanimate\\
\textsc{incl} &  inclusive\\
L & low tone\\
M & mid tone\\
\textsc{perf} &  perfective\\
\textsc{pl} &  plural\\
\textsc{poss} &  possessive\\
\textsc{prog} &  progressive\\
\textsc{pst} &  past\\
\textsc{sg} &  singular\\
{}[sp] & spread glottis\\
SPA & Amuzgo variety from San Pedro Amuzgos\\
V & vowel\\
\textsc{vol} &  volitive mood\\
\\
\end{tabularx}

\section*{Acknowledgments} 

My appreciation to the editors for their very helpful notes and revisions. This study benefitted enormously from the anonymous reviewers. Also thank you to Jair Apóstol, Yuni Kim and the Endangered Language Alliance for their support. Any errors are solely mine.

\sloppy\printbibliography[heading=subbibliography,notkeyword=this]
\end{document} 

